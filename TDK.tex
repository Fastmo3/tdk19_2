\documentclass[12pt,a4paper]{article}
\usepackage[utf8]{inputenc}
\usepackage[magyar]{babel}
\usepackage[T1]{fontenc}
\usepackage{amsmath}
\usepackage{amsfonts}
\usepackage{amssymb}
\usepackage[left=2cm,right=2cm,top=2cm,bottom=2cm]{geometry}
\date{\vspace{-5ex}}
%TODO a petri háló írása legyen konzisztens
\title{Implementing a Petri net visualizer \& editor}%LEcserélni az actual title-re!!
\begin{document}
\maketitle
\section{Introduction}
A dolgozat célja egy működő Petri háló  megjelenítő, szerkesztő %(és futtató?)
alkalmazás fejlesztése, és bemutatása. %TODO: (folyt)
\section{Petri net}
\subsection{Overview}
A Petri net egy matematikai leírómodell elosztott rendszerek bemutatására. A modellt Carl Adam Petri készítette. A modell nagyon hasonlít a programozók körében elterjedt folyamat ábrára. A háló irányított élekből, helyekből és átmenetekből (\textsl{mint elemek}) áll. Az élek csak két különböző típusú elem között állhatnak. A helyeken pontok, ún. tokenek állhatnak. A tokenek csak diszkrét számban fordulhatnak elő egy helyen, és a token átvitele atomi folyamat, azaz nem félbeszakítható. A tokenek elláthatóak attribútummal is ilyen esetben a tokeneket "kiszínezzük" és színezett petri hálóról beszélünk. (ld. 2.2.) %TODO (LINK!)

Formálisan  a petri háló egy $$(P,T,F)$$ rendezett hármas, ahol 
\begin{itemize}
\item $P$ egy véges \textsl{hely} halmaz ,
\item $T$ egy véges átmenet halmaz,
\item $F\subseteq (P\times T\cup T\times P)$ pedig egy folyamat reláció
\end{itemize}

\subsection{Colored Petri net}
\section{BPEL}
A BPEL (Buisness Process Execution Language)üzleti folyamatok végrehajtó nyelve. Az OASIS által kezelt XML alapú szabványt használ. 
\section{The Application}
\subsection{Overview}
\subsection{Usage}
\end{document}