%TODO Add bpel control elements, 10 example from core std /w pics, annot, and bpel desgn pictograms. Non core elements only w/ annotation

\chapter{A BPEL és folyamatainak bemutatása} 

A BPEL (Buisness Process Execution Language)üzleti folyamatok végrehajtó nyelve \cite{saraswathi2013oracle}.
Az OASIS által kezelt XML alapú szabványt használ. 
A dokumentum felépítésében egy XML dokumentum, mely a WS-BPEL szabvány szerint validált. A BPEL programok szerkezetét célszerű egy mintával áttekinteni, a jobb megértéshez. %TODO link: http://docs.oasis-open.org/wsbpel/2.0/OS/wsbpel-v2.0-OS.html chapter 5 section 1
Vegyük a következő példát.  \textsl{Adott egy online rendelést felvevő cég. A cég egy automata segítségével generál számlákat A számla az ár kiszámítása, a futár kiválasztása, és a szükséges termelés ütemezése után kerül kiállításra.} A lépéseket az alábbi ábra mutatja be: %TODO insert fig1.png 
Az ábrán a téglalapok egy rész processzt jelentenek. Az egy blokkban különállóak pedig konkurens proceszeket. A szaggatott vonal szekvenciát jelöl, míg a teli/sima pedig vezérlő linkek, a konkurens processzek szinkronizációját, várakoztatását lehet velük megoldani. Az ábra nem képez átíratot, csak mint egy standard érthető vizualizáció segíti a megértést. 

A program következő része egy WSDL szabvány ami a portot adja meg a processz számára. %TODO link to appendix
\newpage


 
 A kódban szereplő \texttt{<partnerLinks>} tartalmaz mindent (így közvetetten mindenkit) amik kapcsolatba kerül a processzel. Az elnevezés tükrözi a résztvevő partit, valamint a résztvevő feladatát, szándékát. A \texttt{<variables>} a változókat tartalmazza, míg a \texttt{<faultHandlers>} a hibakezelőket. A hibakezelés egy try-catch-finally "hibakezelőblokk" helyett az XML mentalitását tükröző módon kerül lekezelésre, a handlerek által. A kód többi része a processz standard definíciójához tartozik. A példák alapján elmondható, hogy a program a következő struktúra szerint épül fel.
\begin{itemize}
\item \textbf{Definíció: } a processz neve, névtere és különféle sémahívások, majd bővítmények, importok
\item \textbf{PartnerLinkek: } A megfelelő partnerek hozzáadása attribútumokkal. 
\item \textbf{Változók}
\item \textbf{Hibakezelők}
\item\textbf{Eseménykezelők}
\end{itemize}
