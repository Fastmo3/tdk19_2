\chapter{BPEL folyamatok}

A BPEL (Buisness Process Execution Language)üzleti folyamatok végrehajtó nyelve \cite{saraswathi2013oracle}.
Az OASIS által kezelt XML alapú szabványt használ. 
A dokumentum felépítésében egy XML dokumentum, mely a WS-BPEL szabvány szerint validált. A BPEL programok szerkezetét célszerű egy mintával áttekinteni, a jobb megértéshez. %TODO link: http://docs.oasis-open.org/wsbpel/2.0/OS/wsbpel-v2.0-OS.html chapter 5 section 1
Vegyük a következő példát.  \textsl{Adott egy online rendelést felvevő cég. A cég egy automata segítségével generál számlákat A számla az ár kiszámítása, a futár kiválasztása, és a szükséges termelés ütemezése után kerül kiállításra.} A lépéseket az alábbi ábra mutatja be: %TODO insert fig1.png 
Az ábrán a téglalapok egy rész processzt jelentenek. Az egy blokkban különállóak pedig konkurens proceszeket. A szaggatott vonal szekvenciát jelöl, míg a teli/sima pedig vezérlő linkek, a konkurens processzek szinkronizációját, várakoztatását lehet velük megoldani. Az ábra nem képez átíratot, csak mint egy standard érthető vizualizáció segíti a megértést. 

A program következő része egy WSDL szabvány ami a portot adja meg a processz számára:
\newpage

\begin{verbatim}
<wsdl:definitions
   targetNamespace="http://manufacturing.org/wsdl/purchase"
   xmlns:sns="http://manufacturing.org/xsd/purchase"
   xmlns:pos="http://manufacturing.org/wsdl/purchase"
   xmlns:wsdl="http://schemas.xmlsoap.org/wsdl/"
   xmlns:plnk="http://docs.oasis-open.org/wsbpel/2.0/plnktype"
   xmlns:xsd="http://www.w3.org/2001/XMLSchema">
   
   <wsdl:types>
      <xsd:schema>
         <xsd:import namespace="http://manufacturing.org/xsd/purchase"
          schemaLocation="http://manufacturing.org/xsd/purchase.xsd" />
      </xsd:schema>
   </wsdl:types> 

   <wsdl:message name="POMessage">
      <wsdl:part name="customerInfo" type="sns:customerInfoType" />
      <wsdl:part name="purchaseOrder" type="sns:purchaseOrderType" />
   </wsdl:message>

   <wsdl:message name="InvMessage">
      <wsdl:part name="IVC" type="sns:InvoiceType" />
   </wsdl:message>

   <wsdl:message name="orderFaultType">
     <wsdl:part name="problemInfo" element="sns:OrderFault"/>
   </wsdl:message>

   <wsdl:message name="shippingRequestMessage">
      <wsdl:part name="customerInfo" element="sns:customerInfo" />
   </wsdl:message>

   <wsdl:message name="shippingInfoMessage">
      <wsdl:part name="shippingInfo" element="sns:shippingInfo" />
   </wsdl:message>

   <wsdl:message name="scheduleMessage">
      <wsdl:part name="schedule" element="sns:scheduleInfo" />
   </wsdl:message> 

   <!-- portTypes supported by the purchase order process -->

   <wsdl:portType name="purchaseOrderPT">
      <wsdl:operation name="sendPurchaseOrder">
         <wsdl:input message="pos:POMessage" />
         <wsdl:output message="pos:InvMessage" />
         <wsdl:fault name="cannotCompleteOrder"
            message="pos:orderFaultType" />
      </wsdl:operation>
   </wsdl:portType>
   
   <wsdl:portType name="invoiceCallbackPT">
      <wsdl:operation name="sendInvoice">
         <wsdl:input message="pos:InvMessage" />
      </wsdl:operation>
   </wsdl:portType>

   <wsdl:portType name="shippingCallbackPT">
      <wsdl:operation name="sendSchedule">
         <wsdl:input message="pos:scheduleMessage" />
      </wsdl:operation>
   </wsdl:portType>

   <!-- portType supported by the invoice services -->

   <wsdl:portType name="computePricePT">
      <wsdl:operation name="initiatePriceCalculation">
         <wsdl:input message="pos:POMessage" />
      </wsdl:operation>      
      
      <wsdl:operation name="sendShippingPrice">
         <wsdl:input message="pos:shippingInfoMessage" />
      </wsdl:operation>
   </wsdl:portType>

   <!-- portType supported by the shipping service -->

   <wsdl:portType name="shippingPT">
      <wsdl:operation name="requestShipping">
         <wsdl:input message="pos:shippingRequestMessage" />
         <wsdl:output message="pos:shippingInfoMessage" />
         <wsdl:fault name="cannotCompleteOrder"
            message="pos:orderFaultType" />
      </wsdl:operation>
   </wsdl:portType>

   <!-- portType supported by the production scheduling process -->

   <wsdl:portType name="schedulingPT">
      <wsdl:operation name="requestProductionScheduling">
         <wsdl:input message="pos:POMessage" />
      </wsdl:operation>
      
      <wsdl:operation name="sendShippingSchedule">
         <wsdl:input message="pos:scheduleMessage" />
      </wsdl:operation>
   </wsdl:portType> 

   <plnk:partnerLinkType name="purchasingLT">
      <plnk:role name="purchaseService"
         portType="pos:purchaseOrderPT" />
   </plnk:partnerLinkType> 

   <plnk:partnerLinkType name="invoicingLT">
      <plnk:role name="invoiceService"
         portType="pos:computePricePT" />
      <plnk:role name="invoiceRequester"
         portType="pos:invoiceCallbackPT" />
   </plnk:partnerLinkType> 

   <plnk:partnerLinkType name="shippingLT">
      <plnk:role name="shippingService"
         portType="pos:shippingPT" />
      <plnk:role name="shippingRequester"
         portType="pos:shippingCallbackPT" />
   </plnk:partnerLinkType> 

   <plnk:partnerLinkType name="schedulingLT">
      <plnk:role name="schedulingService"
         portType="pos:schedulingPT" />
   </plnk:partnerLinkType>
</wsdl:definitions>
\end{verbatim}

A szükséges portok és linkek definiálása után most a rendelés processzét definiáljuk hasonlóképp. 
\begin{verbatim}
<process name="purchaseOrderProcess"
   targetNamespace="http://example.com/ws-bp/purchase"
   xmlns="http://docs.oasis-open.org/wsbpel/2.0/process/executable"
   xmlns:lns="http://manufacturing.org/wsdl/purchase"> 
   
   <documentation xml:lang="EN">
      A simple example of a WS-BPEL process for handling a purchase
      order.
   </documentation> 

   <partnerLinks>
      <partnerLink name="purchasing"
         partnerLinkType="lns:purchasingLT" myRole="purchaseService" />
      <partnerLink name="invoicing" partnerLinkType="lns:invoicingLT"
         myRole="invoiceRequester" partnerRole="invoiceService" />
      <partnerLink name="shipping" partnerLinkType="lns:shippingLT"
         myRole="shippingRequester" partnerRole="shippingService" />
      <partnerLink name="scheduling"
         partnerLinkType="lns:schedulingLT"
         partnerRole="schedulingService" />
   </partnerLinks> 

   <variables>
      <variable name="PO" messageType="lns:POMessage" />
      <variable name="Invoice" messageType="lns:InvMessage" />
      <variable name="shippingRequest"
         messageType="lns:shippingRequestMessage" />
      <variable name="shippingInfo"
         messageType="lns:shippingInfoMessage" />
      <variable name="shippingSchedule"
         messageType="lns:scheduleMessage" />
   </variables> 

   <faultHandlers>
      <catch faultName="lns:cannotCompleteOrder"
         faultVariable="POFault"
         faultMessageType="lns:orderFaultType">
         <reply partnerLink="purchasing"
            portType="lns:purchaseOrderPT"
            operation="sendPurchaseOrder" variable="POFault"
            faultName="cannotCompleteOrder" />
      </catch>
   </faultHandlers> 

   <sequence>
      <receive partnerLink="purchasing" portType="lns:purchaseOrderPT"
         operation="sendPurchaseOrder" variable="PO"
         createInstance="yes">
         <documentation>Receive Purchase Order</documentation>
      </receive>

      <flow>
         <documentation>
            A parallel flow to handle shipping, invoicing and
            scheduling
         </documentation>
         <links>
            <link name="ship-to-invoice" />
            <link name="ship-to-scheduling" />
         </links>
         <sequence>
            <assign>
               <copy>
                  <from>$PO.customerInfo</from>
                  <to>$shippingRequest.customerInfo</to>
               </copy>
            </assign>
            <invoke partnerLink="shipping" portType="lns:shippingPT"
               operation="requestShipping"
               inputVariable="shippingRequest"
               outputVariable="shippingInfo">
               <documentation>Decide On Shipper</documentation>
               <sources>
                  <source linkName="ship-to-invoice" />
               </sources>
            </invoke>
            <receive partnerLink="shipping"
               portType="lns:shippingCallbackPT"
               operation="sendSchedule" variable="shippingSchedule">
               <documentation>Arrange Logistics</documentation>
               <sources>
                  <source linkName="ship-to-scheduling" />
               </sources>
            </receive>
         </sequence>
         <sequence>
            <invoke partnerLink="invoicing"
               portType="lns:computePricePT"
               operation="initiatePriceCalculation"
               inputVariable="PO">
               <documentation>
                  Initial Price Calculation
               </documentation>
            </invoke>
            <invoke partnerLink="invoicing"
               portType="lns:computePricePT"
               operation="sendShippingPrice"
               inputVariable="shippingInfo">
               <documentation>
                  Complete Price Calculation
               </documentation>
               <targets>
                  <target linkName="ship-to-invoice" />
               </targets>
            </invoke>
            <receive partnerLink="invoicing"
               portType="lns:invoiceCallbackPT"
               operation="sendInvoice" variable="Invoice" />
         </sequence>
         <sequence>
            <invoke partnerLink="scheduling"
               portType="lns:schedulingPT"
               operation="requestProductionScheduling"
               inputVariable="PO">
               <documentation>
                  Initiate Production Scheduling
               </documentation>
            </invoke>
            <invoke partnerLink="scheduling"
               portType="lns:schedulingPT"
               operation="sendShippingSchedule"
               inputVariable="shippingSchedule">
               <documentation>
                  Complete Production Scheduling
               </documentation>
               <targets>
                  <target linkName="ship-to-scheduling" />
               </targets>
            </invoke>
         </sequence>
      </flow>
      
      <reply partnerLink="purchasing" portType="lns:purchaseOrderPT"
         operation="sendPurchaseOrder" variable="Invoice">
         <documentation>Invoice Processing</documentation>
      </reply>
   </sequence>
</process>
\end{verbatim}
 
 A kódban szereplő \texttt{<partnerLinks>} tartalmaz mindent (így közvetetten mindenkit) amik kapcsolatba kerül a processzel. Az elnevezés tükrözi a résztvevő partit, valamint a résztvevő feladatát, szándékát. A \texttt{<variables>} a változókat tartalmazza, míg a \texttt{<faultHandlers>} a hibakezelőket. A hibakezelés egy try-catch-finally "hibakezelőblokk" helyett az XML mentalitását tükröző módon kerül lekezelésre, a handlerek által. A kód többi része a processz standard definíciójához tartozik. A példák alapján elmondható, hogy a program a következő struktúra szerint épül fel.
\begin{itemize}
\item \textbf{Definíció: } a processz neve, névtere és különféle sémahívások, majd bővítmények, importok
\item \textbf{PartnerLinkek: } A megfelelő partnerek hozzáadása attribútumokkal. 
\item \textbf{Változók}
\item \textbf{Hibakezelők}
\item\textbf{Eseménykezelők}
\end{itemize}
