\chapter{Az üzleti folyamatok elemeinek leképzése}

%TODO Ide kellene felsorolni, és részletesen leírni, hogy a BPEL egyes elemeinek milyen Petri-háló feleltethető meg.

%TODO Megnézni, hogy az egyes elemek esetében milyen alternatívák lennének a leképzésre!
\section{A leképzés menete}
A BPEL nyelv az egymással együttműködő modulok működését szimulálja. A BPEL  modellben a folyamat lépéseit tevékenységnek \textsl{activity} nevezzük. A tevékenységek lehetnek elemiek és összetettek. Tipikus elemi tevékenységek:
\begin{itemize}
\item más funkció meghívása,
\item üzenet küldés/fogadás,
\item változók módosítása,
\item várakozás, szinkronizálás.
\end{itemize}
Az elemi tevékenységekből összetett folyamatok építhetőek fel szekvencia, párhuzamosítás és ciklus elemek segítségével.

A BPEL szabvány tevékenység készletének bemutatásánál csak a legfontosabb elemekre térünk most ki. A szabvány részletei a
http://docs.oasis-open.org/wsbpel/2.0/OS/wsbpel-v2.0-OS.html weboldalról érhetőek el. 

A BPEL definíció egyértelműsíti a folyamat kezdetét és inicializálja is a megfelelő paraméterekkel. A Petri háló erre nem tér ki, ezért bevezetünk egy opcionális \textit{START} trnazíciót. Ennek feladata, a megfelelő tokenek legenerálása a folyamat indításához. A folyamat zárása szintén egyértelmű a BPEL szabványban. Petri háló kapcsán beszélhetünk a háló leállásáról, ha már semmilyen hely és tranzíció nem produkál új tokent, nem várakozik és nem nyel el tokent. A köztes elemek leképzése általánosan nem bonyolult, de megvizsgálhatók alternatív leképzési módok. 

(Jelölésrendszer: A rendszer a helyeket P-vel, míg a tranzíciókat T-vel indexeli. A helyek után listázza az adott helyen levő tokeneket. A jelenleg lépésben levő tranzíciók pedig színes kitöltést kapnak. )

\section{\texttt{<receive>}}

\texttt{<receive>} A recieve egy megfelelő üzenet után engedi a folyamatot továbbhaladni, így várakoztatáshoz használható. A \texttt{<receive>} típusú XML elemekhez tartozó sémaleírás:\\
\begin{verbatim}
<receive partnerLink="NCName"
   portType="QName"?
   operation="NCName"
   variable="BPELVariableName"?
   createInstance="yes|no"?
   messageExchange="NCName"?
   standard-attributes>
   standard-elements
   <correlations>?
      <correlation set="NCName" initiate="yes|join|no"? />+
   </correlations>
   <fromParts>?
      <fromPart part="NCName" toVariable="BPELVariableName" />+
   </fromParts>
</receive>
\end{verbatim}
A hálóban ezt egy tranzicióval könnyedén megoldhatjuk, hiszen csak egy specifikus üzenet token kell a továbblépéshez, és a többit addig az előző helyen parkoltatja. 

\section{\texttt{<reply>}}

\texttt{<reply>} A reply egy üzenetküldő elem, ami \texttt{<receive>; <onMessage>;<onEvent>} események után léphet akcióba. 
\begin{verbatim}
<reply partnerLink="NCName"
   portType="QName"?
   operation="NCName"
   variable="BPELVariableName"?
   faultName="QName"?
   messageExchange="NCName"?
   standard-attributes>
   standard-elements
   <correlations>?
      <correlation set="NCName" initiate="yes|join|no"? />+
   </correlations>
   <toParts>?
      <toPart part="NCName" fromVariable="BPELVariableName" />+
   </toParts>
</reply>
\end{verbatim}
A lekezelése az előző példával analóg módon, annyi különbséggel, hogy a tokenek nem parkolnak, hanem tovább mennek és a tranzíció csak akkor generál új tokent ha üzenetet kap egy ágról.

\section{\texttt{<invoke>}}

\texttt{<invoke>} Egy BPEL vagy épp egy webszolgáltatás meghívására szolgál és definiálja a szolgáltatás feladatát is. 
\begin{verbatim}
<invoke partnerLink="NCName"
   portType="QName"?
   operation="NCName"
   inputVariable="BPELVariableName"?
   outputVariable="BPELVariableName"?>
   <catch faultName="QName"? … >*
  	activity
   </catch>
   <toParts>?
      <toPart part="NCName" fromVariable="BPELVariableNm"/>+
   </toParts>
   <fromParts>?
      <fromPart part="NCName" toVariable="BPELVariableNm"/>+
   </fromParts>
</invoke>

\end{verbatim}
A grafikus megjelenítése \aref{fig:invoke}. ábrán látható.

\begin{figure}[h!]
\centering
\includegraphics[scale=0.6]{images/invoke.png}
\caption{Az \texttt{invoke} grafikus jelölése az Oracle BPEL designer-ben}
\label{fig:invoke}
\end{figure}

Használatát a programrészek újrafelhasználhatósága indokolja, valamint az átláthatósági alapelvek. Például, az ábrán látható CCvalidation használható ATM-es pénz felvét, egyenleglekérdezés, vagy egyéb ATM nél végezhető művelet során. 

\begin{figure}[h!]
\centering
\includegraphics[scale=0.6]{images/invokenet.png}
\caption{Az \texttt{invoke} leképzése petri hálóra}
\label{fig:invoke}
\end{figure}

Leképzés során ügyelni kell arra, hogy az \texttt{invoke} paramétereinek megfelelő tokenek keletkezzenek és legyenek átadva a részháló start elemének.
\section{\texttt{<assign>}}

\texttt{<assign>} Egy változó értékadására szolgáló esemény. Ellentétben egy imperatív értékadással egy \texttt{assign} blokkban bármennyi értékadás, másolás történhet, amíg azt a kliens kezelni tudja, így logikailag egy egységbe zárja a műveleteket.  
\begin{verbatim}
<assign validate="yes|no"? standard-attributes>
   (
   <copy keepSrcElementName="yes|no"? 
  	from-spec
  	to-spec
   </copy>
</assign>
\end{verbatim}
Az érték hozzárendelése nagyon egszerűen átírható egy tranzicióra ami a megfelelő tokenek színét módosítja. 
A grafikus megjelenítése \aref{fig:assign}. ábrán látható.
A színmódosítás egyszerűen a token nevének átírását jelenti. 

\begin{figure}[h!]
\centering
\includegraphics[scale=1]{images/assign.png}
\caption{Az \texttt{assign} grafikus jelölése az Oracle BPEL designer-ben}
\label{fig:assign}
\end{figure}

\section{\texttt{<validate>}}

\texttt{<validate>} Egy sémára validálja az XML (BPEL) állományt. 
\begin{verbatim}
<validate variables="BPELVariableNames" standard-attributes>
   standard-elements
</validate>
\end{verbatim}

\section{\texttt{<throw>}}

Egy rész processzen belül fault generálására szolgál. 
\begin{verbatim}
<throw faultName="QName"
   faultVariable="BPELVariableName"?
   standard-attributes>
   standard-elements
</throw>
\end{verbatim}
Nagyon egyszerűen egy \textit{fault} tokent generáló tranzició komponens. Explicit hálórésze nincs, hanem a megfelelő inputtokenek megléte vagy hiánya generálja egy tranzició során. 

\section{\texttt{<wait>}}

\texttt{<wait>} Időre vonatkoztatva várakoztat. Például 5000 tick vagy 14:00:23 (hh:mm:ss)
\begin{verbatim} 
<wait standard-attributes>
   standard-elements
   (
   <for expressionLanguage="anyURI"?>duration-expr</for>
   |
   <until expressionLanguage="anyURI"?>deadline-expr</until>
   )
</wait>
\end{verbatim}
Megadható egy részhálóval ami valójában egy oszcillátor és a megfelelő iteráció után folytat tokent küld. 

\section{\texttt{<empty>}}

\texttt{<empty>} No-op (\textit{no operations}) esemény szinkronizációra szolgál.
\begin{verbatim}
<empty standard-attributes>
   standard-elements
</empty>
\end{verbatim}
Beiktatható egy semleges tranzició és hely.

\section{\texttt{<sequence>}}

\texttt{<sequence>} Sorozatot ad meg.
\begin{verbatim}
<sequence standard-attributes>
   standard-elements
   activity+
</sequence>
\end{verbatim}
Egyszerűen csak tranziciók és helyek összefűzése. 

\section{\texttt{<if>}}

\texttt{<if>} Standard kétirányú elágazás. Logikai XPATH kifejezést vár. 
\begin{verbatim}
<if standard-attributes>
   standard-elements
   <condition expressionLanguage="anyURI"?>bool-expr</condition>
   activity
   <elseif>*
      <condition expressionLanguage="anyURI"?>bool-expr</condition>
      activity
   </elseif>
   <else>?
      activity
   </else>
</if>
\end{verbatim}
Egy tranzició, mely tokenek függvényében más felé küldi tovább, vagy generál tokeneket. Analóg módon egy \textit{Switch-Case} elágazás is definiálható vele.
A grafikus megjelenítése \aref{fig:if}. ábrán látható.

\begin{figure}[h!]
\centering
\includegraphics[scale=1]{images/if.png}
\caption{Az \texttt{if} grafikus jelölése az Oracle BPEL designer-ben}
\label{fig:if}
\end{figure}

\section{\texttt{<while>}}

\texttt{<while>} elöltesztelő ciklus. Végre hajt amíg az iterációs feltétel igaznak értékelődik ki. 
\begin{verbatim}
<while standard-attributes>
   standard-elements
   <condition expressionLanguage="anyURI"?>bool-expr</condition>
   activity
</while>
\end{verbatim}
Egy tranzició, mely token függvényében a folyamat egy korábbi pontjára csatol vissza, vagy éppen egy későbbire, a feltétel hamis logikai állapota esetén. A feltétel persze egy színes token jelenléte, vagy tokenek száma is lehet. 

\section{\texttt{<repeatUntil>}}

\texttt{<repeatUntil>} Egy hátultesztelő ciklusnak feleltethető amely akkor enged tovább, ha a feltétel igaz. 
\begin{verbatim}
<repeatUntil standard-attributes>
   standard-elements
   activity
   <condition expressionLanguage="anyURI"?>bool-expr</condition>
</repeatUntil>
\end{verbatim}
A \texttt{while}-al analóg módon megadható.

\section{\texttt{<forEach>}}

Végig iterál a gyerekelemeken. Megadható párhuzamos feldolgozás is. Egy \textit{Complete condition} segítségével megadható egy break utasítás ami kilép a ciklusból. 
\begin{verbatim}
<forEach counterName="BPELVariableName" parallel="yes|no"
   standard-attributes>
   standard-elements
   <startCounterValue expressionLanguage="anyURI"?>
      unsigned-integer-expression
   </startCounterValue>
   <finalCounterValue expressionLanguage="anyURI"?>
      unsigned-integer-expression
   </finalCounterValue>
   <completionCondition>?
      <branches expressionLanguage="anyURI"?
         successfulBranchesOnly="yes|no"?>?
         unsigned-integer-expression
      </branches>
   </completionCondition>
   <scope ...>...</scope>
</forEach>
\end{verbatim}
Egyszerű loop utasítás, azonban párhuzamosítás esetén a részhálóból megfelelő példányszámot generáltatunk. 

\section{\texttt{<pick>}}

Üzenetek várására vagy időtúllépés eseményre figyel. Ezek bármelyike a szubprocessz végrehajtásához vezet. 
\begin{verbatim}
<pick createInstance="yes|no"? standard-attributes>
   standard-elements
   <onMessage partnerLink="NCName"
      portType="QName"?
      operation="NCName"
      variable="BPELVariableName"?
      messageExchange="NCName"?>+
      <correlations>?
         <correlation set="NCName" initiate="yes|join|no"? />+
      </correlations>
      <fromParts>?
         <fromPart part="NCName" toVariable="BPELVariableName" />+
      </fromParts>
      activity
   </onMessage>
   <onAlarm>*
      (
      <for expressionLanguage="anyURI"?>duration-expr</for>
      |
      <until expressionLanguage="anyURI"?>deadline-expr</until>
      )
      activity
   </onAlarm>
</pick>
\end{verbatim}
A grafikus megjelenítése \aref{fig:pick}. ábrán látható.

\begin{figure}[h!]
\centering
\includegraphics[scale=1]{images/pick.png}
\caption{A \texttt{pick} grafikus jelölése az Oracle BPEL designer-ben}
\label{fig:pick}
\end{figure}

\section{\texttt{<flow>}}

Konkurens elemek deklarálására szolgál. Linkek segítségével megadható függőségi viszony a gyerekek között. 
\begin{verbatim}
<flow standard-attributes>
   standard-elements
   <links>?
      <link name="NCName" />+
   </links>
   activity+
</flow>
\end{verbatim} 
A grafikus megjelenítése \aref{fig:flow}. ábrán látható.

\begin{figure}[h!]
\centering
\includegraphics[scale=1]{images/flow.png}
\caption{A \texttt{flow} grafikus jelölése az Oracle BPEL designer-ben}
\label{fig:flow}
\end{figure}

\section{\texttt{<scope>}}

A gyerek elemek hatókörét lehet vele szabályozni.
\begin{verbatim}
<scope isolated="yes|no"? exitOnStandardFault="yes|no"?
   standard-attributes>
   standard-elements
   <partnerLinks>?
      ... see above under <process> for syntax ...
   </partnerLinks>
   <messageExchanges>?
      ... see above under <process> for syntax ...
   </messageExchanges>
   <variables>?
      ... see above under <process> for syntax ...
   </variables>
   <correlationSets>?
      ... see above under <process> for syntax ...
   </correlationSets>
   <faultHandlers>?
      ... see above under <process> for syntax ...
   </faultHandlers>
   <compensationHandler>?
      ...
   </compensationHandler>
   <terminationHandler>?
      ...
   </terminationHandler>
   <eventHandlers>?
      ... see above under <process> for syntax ...
   </eventHandlers>
   activity
</scope>
\end{verbatim}
Nem generál új elemet, csak a láthatósági, azaz visszacsatolási elemeket adja meg. 

\section{\texttt{<compensateScope>}}

\texttt{<compensateScope>}
%TODO ???
\begin{verbatim}
<compensateScope target="NCName" standard-attributes>
   standard-elements
</compensateScope>
\end{verbatim}

\section{\texttt{<compensate>}}

\texttt{<compensate>}
%TODO ???
\begin{verbatim}
<compensate standard-attributes>
   standard-elements
</compensate>
\end{verbatim}

\section{\texttt{<rethrow>}}

\texttt{<rethrow>}
%TODO ???
\begin{verbatim}
<rethrow standard-attributes>
   standard-elements
</rethrow>
\end{verbatim}

\section{\texttt{<extensionActivity>}}

\texttt{<extensionActivity>}
%TODO ???
\begin{verbatim}
<extensionActivity>
   <anyElementQName standard-attributes>
      standard-elements
   </anyElementQName>
</extensionActivity>
\end{verbatim}
