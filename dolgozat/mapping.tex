\chapter{Az üzleti folyamatok elemeinek leképzése}

%TODO Ide kellene felsorolni, és részletesen leírni, hogy a BPEL egyes elemeinek milyen Petri-háló feleltethető meg.

%TODO Megnézni, hogy az egyes elemek esetében milyen alternatívák lennének a leképzésre!
\section{A leképzés menete}
A leképzéskor elsődlegesen az aktivitás elemei kerülnek a figyelem középpontjába, a séma és különböző definíciók csak másodlagos helyet kapnak. Ez azért lényeges, mert a séma nem egy dinamikus folyamatot ír le, hanem az adott folyamat és eleminek tulajdonságát. Mivel a Petri-háló egy folyamat személtetésére lett létrehozva ezért főleg az aktív elemek a szignifikánsak a konverzió során, tehát leképzés során az XML validációhoz szükséges statikus elemeket ignoráljuk.  Az alábbi elemeknek feleltetünk meg egy egy részhálót:
\section{\texttt{<receive>}}
\texttt{<receive>} A recieve egy megfelelő üzenet után engedi a folyamatot továbbhaladni, így várakoztatáshoz használható. A \texttt{<receive>} teljes terjedelmében:\\
\begin{verbatim}
<receive partnerLink="NCName"
   portType="QName"?
   operation="NCName"
   variable="BPELVariableName"?
   createInstance="yes|no"?
   messageExchange="NCName"?
   standard-attributes>
   standard-elements
   <correlations>?
      <correlation set="NCName" initiate="yes|join|no"? />+
   </correlations>
   <fromParts>?
      <fromPart part="NCName" toVariable="BPELVariableName" />+
   </fromParts>
</receive>
\end{verbatim}
A hálóban ezt egy tranzicióval könnyedén megoldhatjuk, hisz csak egy specifikus msg token kell a továbblépéshez, és a többit addig az előző helyen parkoltatja. 

\section{\texttt{<reply>}}
\texttt{<reply>} A reply egy üzenetküldő elem, ami \texttt{<receive>; <onMessage>;<onEvent>} események után léphet akcióba. 
\begin{verbatim}
<reply partnerLink="NCName"
   portType="QName"?
   operation="NCName"
   variable="BPELVariableName"?
   faultName="QName"?
   messageExchange="NCName"?
   standard-attributes>
   standard-elements
   <correlations>?
      <correlation set="NCName" initiate="yes|join|no"? />+
   </correlations>
   <toParts>?
      <toPart part="NCName" fromVariable="BPELVariableName" />+
   </toParts>
</reply>
\end{verbatim}
A lekezelése az előző példával analóg módon, annyi különbséggel, hogy a tokenek nem parkolnak, hanem tovább mennek és a tranzíció csak akkor generál új tokent ha üzenetet kap egy ágról.

\section{\texttt{<invoke>}}
\texttt{<invoke>} Egy Bpel vagy épp egy webszolgáltatás meghívására szolgál és definiálja a szolgáltatás feladatát is. 
\begin{verbatim}
<invoke partnerLink="NCName"
   portType="QName"?
   operation="NCName"
   inputVariable="BPELVariableName"?
   outputVariable="BPELVariableName"?
   standard-attributes>
   standard-elements
   <correlations>?
      <correlation set="NCName" initiate="yes|join|no"?
         pattern="request|response|request-response"? />+
   </correlations>
   <catch faultName="QName"?
      faultVariable="BPELVariableName"?
      faultMessageType="QName"?
      faultElement="QName"?>*
      activity
   </catch>
   <catchAll>?
      activity
   </catchAll>
   <compensationHandler>?
      activity
   </compensationHandler>
   <toParts>?
      <toPart part="NCName" fromVariable="BPELVariableName" />+
   </toParts>
   <fromParts>?
      <fromPart part="NCName" toVariable="BPELVariableName" />+
   </fromParts>
</invoke>
\end{verbatim}
%TODO ebben nem vbagyok biztos hogy hogy lehet. Átad egy start tokent egy részhálónak??

\section{\texttt{<assign>}}
\texttt{<assign>} Egy változó értékadására szolgáló esemény. Ellentétben egy imperatív értékadással egy assign blokkban bármennyi értékadás, másolás történhet, amíg azt a kliens kezelni tudja, így logikailag egy egységbe zárja a műveleteket.  
\begin{verbatim}
<assign validate="yes|no"? standard-attributes>
   standard-elements
   (
   <copy keepSrcElementName="yes|no"? ignoreMissingFromData="yes|no"?>
      from-spec
      to-spec
   </copy>
   |
   <extensionAssignOperation>
      assign-element-of-other-namespace
   </extensionAssignOperation>
   )+
</assign>
\end{verbatim}
Az assign nagyon egszerűen átírható egy tranzicióra ami a megfelelő tokenek színét módosítja. 

\section{\texttt{<validate>}}
\texttt{<validate>} Egy sémára validálja az XML (BPEL) file-t. 
\begin{verbatim}
<validate variables="BPELVariableNames" standard-attributes>
   standard-elements
</validate>
\end{verbatim}
\texttt{<throw>}Egy rész processzen belül fault generálására szolgál. 
\begin{verbatim}
<throw faultName="QName"
   faultVariable="BPELVariableName"?
   standard-attributes>
   standard-elements
</throw>
\end{verbatim}
Nagyon egyszerűen egy fault tokent generáló tranzició komponens. Explicit hálórésze nincs, hanem a megfelelő inputtokenek megléte vagy hiánya generálja egy tranzició során. 

\section{\texttt{<wait>}}
\texttt{<wait>} Időre vonatkoztatva várakoztat. Például 5000 tick vagy 14:00:23 (hh:mm:ss)
\begin{verbatim} 
<wait standard-attributes>
   standard-elements
   (
   <for expressionLanguage="anyURI"?>duration-expr</for>
   |
   <until expressionLanguage="anyURI"?>deadline-expr</until>
   )
</wait>
\end{verbatim}
Megadható egy részhálóval ami valójában egy oszcillátor és a megfelelő iteráció után folytat tokent küld. 

\section{\texttt{<empty>}}
\texttt{<empty>} No-op (no operations) event szinkronizációra szolgál.
\begin{verbatim}
<empty standard-attributes>
   standard-elements
</empty>
\end{verbatim}
Beiktatható egy semleges tranzició és hely.

\section{\texttt{<sequence>} }
\texttt{<sequence>} Sorozatot ad meg.
\begin{verbatim}
<sequence standard-attributes>
   standard-elements
   activity+
</sequence>
\end{verbatim} Egyszerűen csak tranziciók és helyek összefűzése. 

\section{\texttt{<if>}}
\texttt{<if>} Standard kétirányú elágazás. Logikai XPATH kifejezést vár. 
\begin{verbatim}
<if standard-attributes>
   standard-elements
   <condition expressionLanguage="anyURI"?>bool-expr</condition>
   activity
   <elseif>*
      <condition expressionLanguage="anyURI"?>bool-expr</condition>
      activity
   </elseif>
   <else>?
      activity
   </else>
</if>
\end{verbatim}
Egy tranzició, mely tokenek függvényében más felé küldi tovább, vagy generál tokeneket. Analóg módon egy Swithc Case elágazás is definiálható vele.

\section{\texttt{<while>}}
\texttt{<while>} While loop. Végre hajt amíg az iterációs feltétel igaznak értékelődik ki. 
\begin{verbatim}
<while standard-attributes>
   standard-elements
   <condition expressionLanguage="anyURI"?>bool-expr</condition>
   activity
</while>
\end{verbatim}
Egy tranzició, mely token függvényében a folyamat egy korábbi pontjára csatol vissza, vagy ép egy későbbire, a feltétel hamis logikai állapota esetén. A feltétel persze egy színes token jelenléte, vagy tokenek száma is lehet. 

\section{\texttt{<repeatUntil>}}
\texttt{<repeatUntil>} Egy do-while ciklusnak feleltethető annyi kivétellel, hogy akkor enged tovább, ha a feltétel igaz. 
\begin{verbatim}
<repeatUntil standard-attributes>
   standard-elements
   activity
   <condition expressionLanguage="anyURI"?>bool-expr</condition>
</repeatUntil>
\end{verbatim}
Az előzővel analóg módon megadható

\section{\texttt{<forEach>}}
\texttt{<forEach>} A kezdeti változó és a végváltozó különbsége +1 szer iteráltatja a gyerek elemet. Megadható párhuzamos feldolgozás is. Egy Complete condition segítségével megadható egy break utasítás ami kilép a forEachből. 
\begin{verbatim}
<forEach counterName="BPELVariableName" parallel="yes|no"
   standard-attributes>
   standard-elements
   <startCounterValue expressionLanguage="anyURI"?>
      unsigned-integer-expression
   </startCounterValue>
   <finalCounterValue expressionLanguage="anyURI"?>
      unsigned-integer-expression
   </finalCounterValue>
   <completionCondition>?
      <branches expressionLanguage="anyURI"?
         successfulBranchesOnly="yes|no"?>?
         unsigned-integer-expression
      </branches>
   </completionCondition>
   <scope ...>...</scope>
</forEach>
\end{verbatim} Egyszerű loop utasítás, azonban párhuzamosítás esetén a részhálóból megfelelő példányszámot generáltatunk. 

\section{\texttt{<pick>}}
\texttt{<pick>}Üzenetek várására vagy timeout eseményre figyel. Ezek bármelyike a szubprocessz végrehajtásához vezet. 
\begin{verbatim}
<pick createInstance="yes|no"? standard-attributes>
   standard-elements
   <onMessage partnerLink="NCName"
      portType="QName"?
      operation="NCName"
      variable="BPELVariableName"?
      messageExchange="NCName"?>+
      <correlations>?
         <correlation set="NCName" initiate="yes|join|no"? />+
      </correlations>
      <fromParts>?
         <fromPart part="NCName" toVariable="BPELVariableName" />+
      </fromParts>
      activity
   </onMessage>
   <onAlarm>*
      (
      <for expressionLanguage="anyURI"?>duration-expr</for>
      |
      <until expressionLanguage="anyURI"?>deadline-expr</until>
      )
      activity
   </onAlarm>
</pick>
\end{verbatim}

\section{\texttt{<flow>}}
\texttt{<flow>} konkurens elemek deklarálására szolgál. Linkek segítségével megadható függőségi viszony a gyerekek között. 
\begin{verbatim}
<flow standard-attributes>
   standard-elements
   <links>?
      <link name="NCName" />+
   </links>
   activity+
</flow>
\end{verbatim} 

\section{\texttt{<scope>}}
\texttt{<scope>} A gyerek elemek scope-ját lehet vele szabályozni.
\begin{verbatim}
<scope isolated="yes|no"? exitOnStandardFault="yes|no"?
   standard-attributes>
   standard-elements
   <partnerLinks>?
      ... see above under <process> for syntax ...
   </partnerLinks>
   <messageExchanges>?
      ... see above under <process> for syntax ...
   </messageExchanges>
   <variables>?
      ... see above under <process> for syntax ...
   </variables>
   <correlationSets>?
      ... see above under <process> for syntax ...
   </correlationSets>
   <faultHandlers>?
      ... see above under <process> for syntax ...
   </faultHandlers>
   <compensationHandler>?
      ...
   </compensationHandler>
   <terminationHandler>?
      ...
   </terminationHandler>
   <eventHandlers>?
      ... see above under <process> for syntax ...
   </eventHandlers>
   activity
</scope>
\end{verbatim}
Nem generál új elemet, csak a láthatósági, azaz visszacsatolási elemeket adja meg. 

\section{\texttt{<compensateScope>}}
\texttt{<compensateScope>}
\begin{verbatim}
<compensateScope target="NCName" standard-attributes>
   standard-elements
</compensateScope>
\end{verbatim}

\section{\texttt{<compensate>}}
\texttt{<compensate>}
\begin{verbatim}
<compensate standard-attributes>
   standard-elements
</compensate>
\end{verbatim}

\section{\texttt{<rethrow>}}
\texttt{<rethrow>}
\begin{verbatim}
<rethrow standard-attributes>
   standard-elements
</rethrow>
\end{verbatim}

\section{\texttt{<extensionActivity>}}
\texttt{<extensionActivity>}
\begin{verbatim}
<extensionActivity>
   <anyElementQName standard-attributes>
      standard-elements
   </anyElementQName>
</extensionActivity>
\end{verbatim}

