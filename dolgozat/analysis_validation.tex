\chapter{A hálón végezhető elemzések}
\section{Háló korlátosság és puffer kapacitási ellenőrzés}

A háló egy adott helye akkor tekinthető korlátos (\textit{bounded}) helynek, ha bármely jelölésnél a tokenek száma az adott helynél nem megy egy adott korlát fölé. A Petri háló korlátos, ha minden helye korlátos hely.
A háló korlátossága az egyik leggyakoribb és legfontosabb minőségi jellemzője a Petri hálóknak. 

A háló alap tulajdonságainak, beleértve a korlátosságának az elemzésére több módszer is létezik, melyek közül kiemelhető a
\begin{itemize}
\item komponens/elérhetőségi gráf elemzése (SCC),
\item dekompozíciós módszerek.
\end{itemize}

A mátrix reprezentáció esetén transzformációs mátrixok segítségével írják fel a Petri háló dinamikáját. Az alapstruktúra az ú.n. incidencia M mátrixban kerül megadásra, melynek elemei az alábbi jelentéssel bírnak: $A_{ij}=a^+_{ij}-a^-_{ij}$, ahol 
\begin{itemize}
\item $a^+_{ij}: $ az élerősség az i. tranzícióból a j. kimeneti hely felé
\item $a^-_{ij}$ az élerősség az i. tranzícióhoz a j. bemeneti hely felől.
\end{itemize}

A mátrix alapvetően a tokenek számának a változását mutatja az egyes tranzició átmenetek esetére. A  Petri háló működési alapegyenlete a következő alakban adható meg: 
$$M_k=M_{k-1}+ Au_k,$$
ahol $M_k$ jelöli a háló markereinek (tokenek) státuszát a $k.$ lépésben. Az $u$ vektor a helyek tüzelési státuszt írja le. 

A fenti modellen alapuló elérhetőség vizsgálatok felhasználhatóak a korlátosság elemzésére \cite{murata1989petri}.
A kapcsolódó  egyenletek hatékony, lineáris programozási megoldása szintén ismert \cite{lasserre1989using}.

%TODO itt van egy kép, nem tudom lehet e direktben cite-olni????

\section{Saját modell}

Modellünkben korlátosságnak egy folyam-gráf megközelítését dolgoztuk ki.  A hálóban a következő típusú helyeket definiáljuk:
\begin{itemize}
\item forrás hely,
\item nyelő hely,
\item köztes hely.
\end{itemize}
Feltesszük, hogy csak a köztes helyeken lehet tokeneket tárolni, csak ott vannak pufferek. A hálóban az élekhez egy $I_x$ token áramlás erősséget definiálunk, ahol x jelöli az él indexét. A forrás helyekhez egy $Q_x$ forrás erősség indexet adunk meg. A hálóban az alábbi kapacitás korlátokat vezetjük be:
\begin{itemize}
\item $C_x:$ az x. tranzíció maximális erőssége,
\item $C_y:$ az y. nyelő maximális folyam erőssége.
\end{itemize}

A tranzícióknál a bemenő élek vonatkozásában kétféle működési módot értelmezünk:
\begin{itemize}
\item AND-mód: akkor van tüzelés, ha minden bejövő élnél megvan az al elvárt tokenszám, van szinkron,
\item OR-mód: akkor van tüzelés, ha megjelenik valamely bemeneten egy token, nincs szinkron.
\end{itemize}
A hálóban az alábbi megkötések élnek a folyamerősségekre:
\begin{itemize}
\item forrás helyek esetén: $\sum_y I_y=Q_x$ ($y$: kimenő élek),
\item nyelő helyek esetén: $\sum_y I_y \leq C_x$ ($y$: bejövő élek),
\item belső helyek esetén: 
$\sum_{y(ki)} I_y\leq \sum_{y(be)} I_y$,
\item tranziciók esetén: 
\begin{itemize}
\item $\sum_y I_y\leq C_x$ ($y:$ bejövő élek),
\item $\sum_{y(ki)} I_y = \sum_{y(be)} I_y$,
\item $\forall$ kimenő $x,y$ élre $I_x=I_y$.
\end{itemize}
\end{itemize}
Az AND típusú tranzakciók esetén még ezen felül teljesül, hogy $\forall$ bejövő $x,y$ élre:\\
$I_x = I_y$.

Az egyes belső helyeken a pufferbe áramló tokenek  eredő intentitása:
$$F= \sum_{x(\text{belső hely})} \left( \sum_{y(\text{x bejövő él})} I_y - \sum_{y(\text{x kimenő él})} I_y \right)$$
Az $F$ függvény 0 értéke esetén nincs szükség belső pufferre.

A fenti feladatot egy LP programozási feladatnak is tekinthető, ahol a változók az élek $I_x$ nem negatív intenzitásai és a célfüggvény:
$$F\Rightarrow \min$$ alakú.

\section{Számítási implementációja}

A számítások Python programozási nyelven kerültek implementálásra a \texttt{PuLP} nevű függvénykönyvtár és a \texttt{networkx} segítségével.

\begin{python}
import networkx as nx
import matplotlib.pyplot as plt
import pulp as p

class cl_place:
    def __init__(self):
        self.id = -1
        self.inputs = []
        self.outputs = []
        self.tokens = []
        self.Q = 0
        self.border = 0


class cl_transition:
    def __init__(self):
        self.id = -1
        self.inputs = []
        self.outputs = []
        self.C = 0
        self.mode = ' '

class cl_link:
    def __init__(self):
        self.id = -1
        self.input = -1
        self.output = -1
        self.alfa = 0
        self.inner = 0

class cl_petri_net:
    def __init__(self,Plist,Tlist, Elist):
        self.NP = len(Plist)
        self.NT = len(Tlist)
        self.NN = self.NP + self.NT
            
        self.places = []
        for i in range(self.NP):
            po = cl_place()
            po.id = i + 1
            po.Q = Plist[i]
            if Plist[i] > 0:
                po.border = 1
            if Plist[i] < 0:
                po.border = 2            
            self.places.append(po)
            
       
        self.transitions = []
        for i in range(len(Tlist)):
            tr = cl_transition()
            tr.id = self.NP + 1 + i
            tr.C = Tlist[i][0]
            tr.mode = Tlist[i][1]
            self.transitions.append(tr)
        
        self.nodes_dict = dict()
        for pp in self.places:
            self.nodes_dict[pp.id] = pp
        for tt in self.transitions:
            self.nodes_dict[tt.id] = tt
        
        self.links = []
        for i in range(len(Elist)):
            ll = cl_link()
            ll.id = i
            ll.input = Elist[i][0]
            ll.output = Elist[i][1]
            if ll.input > self.NP:
                db = 0
                for j in range(len(Elist)):
                    if Elist[j][0] == ll.input:
                        db = db + 1
                ll.alfa = db
                ll.inner = 1
                for j in range(len(self.places)):
                    if self.places[j].id == ll.output:
                        if self.places[j].border != 0:
                            ll.inner = 0
            else:
                db = 0
                for j in range(len(Elist)):
                    if Elist[j][1] == ll.output:
                        db = db + 1
                ll.alfa = db
                ll.inner = 1
                for j in range(len(self.places)):
                    if self.places[j].id == ll.input:
                        if self.places[j].border != 0:
                            ll.inner = 0
            
            self.links.append(ll)
            print (ll.id," : ", ll.input,"->" , ll.output, " : ", ll.alfa)
        
        self.edges_dict = dict()
        for ee in self.links:
            self.edges_dict[ee.id] = ee
        
        for pp in self.places:
            for ee in self.links:
                if ee.input == pp.id:
                    pp.outputs.append(ee.id)
                if ee.output == pp.id:
                    pp.inputs.append(ee.id)
        for tr in self.transitions:
            for ee in self.links:
                if ee.input == tr.id:
                    tr.outputs.append(ee.id)
                if ee.output == tr.id:
                    tr.inputs.append(ee.id)
    
    
    def draw_net(self):
        graph = []
        for ll in self.links:
            graph.append((ll.input, ll.output))
        ncols = []
        mp = 0
        for ll in self.links:
            fnd = 0
            for i in range(mp):
                if ncols[i] == ll.input:
                    fnd = 1
            if fnd == 0:
                ncols.append(ll.input)
                mp = mp + 1
            fnd = 0
            for i in range(mp):
                if ncols[i] == ll.output:
                    fnd = 1
            if fnd == 0:
                ncols.append(ll.output)
                mp = mp + 1
        for i in range (len(ncols)):
            if ncols[i] > self.NP:
                if self.nodes_dict[ncols[i]].mode == 'A':
                    ncols[i] = 'lime'
                else:
                    ncols[i] = 'olivedrab'
            else:
                ncols[i] = 'gold'
            
        labs = dict()
        for x in self.places:
            labs[x.id] = str(x.id) + ":" + str(x.Q)
        for x in self.transitions:
            labs[x.id] = str(x.id) + ":" + str(x.C)

        G=nx.DiGraph()
        for edge in graph:
            G.add_edge(edge[0], edge[1])
        graph_pos = nx.circular_layout(G)
        #graph_pos = nx.spring_layout(G)
    
        nx.draw_networkx_nodes(G, graph_pos,node_size=800, node_color = ncols, node_shape = 's' )
        nx.draw_networkx_edges(G, graph_pos,arrowsize = 30)
        nx.draw_networkx_labels(G, graph_pos, font_size=12, font_family='sans-serif',
        labels = labs)
        plt.show()

    def test_cap (self):
        prob = p.LpProblem("Petri net Problem",p.LpMinimize)

        tr_items = [e.id for e in self.links]
        tr_vars = p.LpVariable.dicts("Iv",tr_items,lowBound=0,cat='Continuous')
        print (tr_vars)
        
        costs = dict()
        for i in tr_items:
            costs[i] = 0
        for pp in self.places:
            if pp.border == 0:
                for ll in pp.inputs:
                    costs[ll] = costs[ll] + 1
                for ll in pp.outputs:
                    costs[ll] = costs[ll] - 1

        prob += p.lpSum([costs[i]*tr_vars[i] for i in tr_items])
        #print ("costs")
        #print (costs)
        
        #print ("weights")
        for pl in self.places:
            wgts = dict()
            cnts = 0
            for i in tr_items:
                wgts[i] = 0
            #print (wgts)
            #print (cnts)
            if pl.border == 0:
                for e in pl.inputs:
                    wgts[e] = wgts[e] + 1
                for e in pl.outputs:
                    wgts[e] = wgts[e] - 1
                cnts = 0
                prob += p.lpSum([wgts[i]*tr_vars[i] for i in tr_items]) >= cnts
            if pl.border == 1:
                for e in pl.outputs:
                    wgts[e] = wgts[e] + 1
                cnts = pl.Q
                prob += p.lpSum([wgts[i]*tr_vars[i] for i in tr_items]) == cnts
            if pl.border == 2:
                for e in pl.inputs:
                    wgts[e] = wgts[e] + 1
                cnts = -pl.Q
                prob += p.lpSum([wgts[i]*tr_vars[i] for i in tr_items]) <= cnts

        for tr in self.transitions:
            wgts = dict()
            cnts = 0
            for i in tr_items:
                wgts[i] = 0
            for e in tr.inputs:
                wgts[e] = wgts[e] + 1
            for e in tr.outputs:
                wgts[e] = wgts[e] - 1
            prob += p.lpSum([wgts[i]*tr_vars[i] for i in tr_items]) == 0
            
            for i in tr_items:
                wgts[i] = 0
            for e in tr.inputs:
                wgts[e] = wgts[e] + 1
            cnts = tr.C
            prob += p.lpSum([wgts[i]*tr_vars[i] for i in tr_items]) <= cnts

            if tr.mode == 'A':
                for e in range(1,len(tr.inputs)):
                    e1 = tr.inputs[0]
                    e2 = tr.inputs[e]
                    for i in tr_items:
                        wgts[i] = 0
                    wgts[e1] =  1
                    wgts[e2] =  -1
                    prob += p.lpSum([wgts[i]*tr_vars[i] for i in tr_items]) == 0
                for e in range(1,len(tr.outputs)):
                    e1 = tr.outputs[0]
                    e2 = tr.outputs[e]
                    for i in tr_items:
                        wgts[i] = 0
                    wgts[e1] =  1
                    wgts[e2] =  -1
                    prob += p.lpSum([wgts[i]*tr_vars[i] for i in tr_items]) == 0
            if tr.mode == 'O':
                for e in range(1,len(tr.outputs)):
                    e1 = tr.outputs[0]
                    e2 = tr.outputs[e]
                    for i in tr_items:
                        wgts[i] = 0
                    wgts[e1] =  1
                    wgts[e2] =  -1
                    prob += p.lpSum([wgts[i]*tr_vars[i] for i in tr_items]) == 0
                    
            
        print ("----------------")
        print (prob)
        prob.solve()
        print ("===============")
        print (p.LpStatus[prob.status])
        QV = 0
        for var in tr_vars:
            QV = QV + tr_vars[var].varValue*costs[var]
            print (tr_vars[var],":",tr_vars[var].varValue)
        print ("Cost=", QV)
\end{python}

\section{Mintafeladat}

Vegyünk egy 4 helyből és 2 tranzícióból álló rendszert. A helyekből egy nyelő, egy forrás és kettő belső hely. A rendszerben 6 él van az ábrán megadott módon.  A gráfban a sárga elem a helyeket, zöld a tranzíciókat jelöli. A csomópont elemben az első jel a hely kódja, a második a kapcsolódó kapacitás érték. A modellben a kisebb indexű tranzíció AND tulajdonságú, a másik OR tulajdonságú. 

A rendszerben 6 (nem negatív) változó jelenik meg:  $\{0: Iv_0, 1: Iv_1, 2: Iv_2, 3: Iv_3, 4: Iv_4, 5: Iv_5\}$

Az élek indexelése:
\begin{align*}
0  :  1 \rightarrow 5  \\
1  :  5 \rightarrow 2  \\
2  :  2 \rightarrow 6  \\
3  :  6 \rightarrow 3  \\
4  :  3 \rightarrow 5  \\
5  :  6 \rightarrow 4 
\end{align*}

A rendszerhez az alábbi egyenlőtlenségek kapcsolódnak:

\begin{center}
\begin{tabular}{rll}
$C_1$ &: $Iv_0$ &$= 20$ \\
$C_2$ &: $Iv_1 - Iv_2$ &$\geq 0$\\
$C_3$ &: $Iv_3 - Iv_4$ &$\geq 0$\\
$C_4$ &: $Iv_5$ &$\leq 25$\\
$C_5$ &: $Iv_0 - Iv_1 + Iv_4$ &$= 0$\\
$C_6$ &: $Iv_0 + Iv_4 $&$\leq 50$\\
$C_7$ &: $Iv_0 - Iv_4 $&$= 0$\\
$C_8$ &: $Iv_2 - Iv_3 - Iv_5$&$= 0$\\
$C_9$ &: $Iv_2 $&$\leq 50$\\
$C_{10}$ &: $Iv_3 - Iv_5 $&$= 0$
\end{tabular}
\end{center}

A kapcsolódó célfüggvény:
$$1\cdot Iv_1 + -1\cdot Iv_2 + 1\cdot Iv_3 + -1\cdot Iv_4\Rightarrow \min$$

Az LP feladat megoldható és a kapott megoldás:
\begin{center}
\begin{tabular}{rll}
&$Iv_0$ &: $20.0$\\
&$Iv_1$ &: $40.0$\\
&$Iv_2$ &: $40.0$\\
&$Iv_3$ &: $20.0$\\
&$Iv_4$ &: $20.0$\\
&$Iv_5$ &: $20.0$\\
&$Cost$ & $= 0.0$
\end{tabular}
\end{center}
Tehát a mintarendszerben nincs szükség belső pufferre. 
Ha lecsökkentjük a második tranzíció folyamerősségét, az akkor nem kapunk érvényes megoldást. \\
Infeasible
\begin{center}
\begin{tabular}{rll}
&$Iv_0$ &: $10.0$\\
&$Iv_1$ &: $20.0$\\
&$Iv_2$ &: $20.0$\\
&$Iv_3$ &: $10.0$\\
&$Iv_4$ &: $10.0$\\
&$Iv_5$ &: $10.0$
\end{tabular}
\end{center}