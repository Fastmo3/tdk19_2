\documentclass[a4paper]{article}

% Set margins
\usepackage[hmargin=2.5cm, vmargin=3cm]{geometry}

\frenchspacing

% Language packages
\usepackage[utf8]{inputenc}
\usepackage[T1]{fontenc}
\usepackage[magyar]{babel}

% AMS
\usepackage{amssymb,amsmath}

% Graphic packages
\usepackage{graphicx}

% Colors
\usepackage{color}
\usepackage[usenames,dvipsnames]{xcolor}

% Listings
\usepackage{listings}

% Question
\newenvironment{question}[1]
{\noindent\textcolor{OliveGreen}{$\circ$ \textit{#1}}

\smallskip

\color{Gray}

}{\bigskip}

% Task
\newenvironment{task}[1]
{\noindent\textcolor{RoyalBlue}{$\circ$ \textit{#1}}

\smallskip

\color{Gray}

}{\bigskip}

% Notification
\newenvironment{notification}[1]
{\noindent\textcolor{Peach}{$\circ$ \textit{#1}}

\smallskip

\color{Gray}

}{\bigskip}

% Problem
\newenvironment{problem}[1]
{\noindent\textcolor{OrangeRed}{$\circ$ \textit{#1}}

\smallskip

\color{Gray}

}{\bigskip}

% Solution
\newenvironment{solution}
{\color{RoyalBlue}}{\bigskip}

% Comment
\newenvironment{comment}
{\color{Green}}{\bigskip}

\lstset{% setup listings
        framexleftmargin=16pt,
        framextopmargin=6pt,
        framexbottommargin=6pt, 
        frame=single, rulecolor=\color{black}
}

\begin{document}

\begin{center}
   \huge \textbf{Üzelti folyamatok}
\end{center}

\vskip 0.3cm

\section{Folyamként való modellezés}

Az üzleti folyamatokat kezelhetjük úgy, mint bizonyos elemek (a Petri-hálók esetében tokenek) feldolgozását.
Ekkor a folyamat leírása egy olyan írányított gráf lesz, amelynek az éleihez kapacitásértékeket rendelhetünk.
Hasonló modellekkel találkozhatunk a közlekedési útvonalak, áram-, víz- és gázrendszerek esetében.

A kérdés az, hogy a topológia és az egyes kapacitások ismeretében mit tudunk elmondani az egész folyamat áteresztőképességéről, illetve hogy mely részekre lehet majd szükséges a kapacitásokat növelni ahhoz, hogy az áteresztőképesség javítható legyen.
(Az áteresztőképesség itt nekünk annak a mértékét jellemzi, hogy mennyi elemet sikerült feldolgozni.)
A kapacitásnövelés tulajdonképpen tároló (\textit{buffer}) csomópontok beiktatását jelenti.
A feladat megoldása arra ad választ, hogy egy létező folyamatot hogyan lehet javítani.

Az egyik kézenfekvő megoldás, hogy ha a problémát folyamfeladatként fogalmazzuk meg.
Az üzleti folyamat esetében kell lenni jól definiált be- és kilépési pontoknak.
Jelölje $s$ a belépési ponthoz (forrás, \textit{source}) és $t$ a kilépési ponthoz (cél, \textit{target}) tartozó csomópontokat.
A súlyokat/kapacitásokat az egyes élekhez szeretnénk rendelni.
A kapacitás jelölheti például a feldolgozott elemek másodpercenkénti számát.

A maximális folyam és a minimális vágás feladatpár megoldására van szükség, amelyek egymásnak duálisai \cite{ford2009maximal}.
Jelöljük az élekhez tartozó súlyokat $c_{ij}$-vel (\textit{capacity}), ahol az $i$ és $j$ a csomópontok indexeit jelölik, amelyeket az él összeköt.
A probléma megoldásához meg kell nézni, hogy az egyes éleken milyen mértékű folyam képes még áthaladni.
Ezek lesznek az optimalizálási feladat döntési változói, amelyeket $x_{ij}$-vel jelölünk.
Feltételezzük, hogy $x_{ij} \geq 0$.
A folyam kapacitásértékeinek figyelembevétele azt jelenti, hogy $x_{ij} \leq c_{ij}$ bármely $i, j$ esetén.

A folyamnak egy további fontos tulajdonsága, hogy egy csomópontból pontosan olyan folyamnak szabad csak kimennie, mint amilyen belement.
Ez egy tetszőleges $k$ csomópont esetén a következőképpen írható le:
\[
\displaystyle
\sum_{i} x_{i, k} = \sum_{j} x_{k, j}.
\]
Ennek az összes $k$ csomópontra teljesülnie kell.

Maximális folyamot akkor kapunk, hogy ha a rendszer az előző feltételek mellett a lehető legnagyobb $x_{i, j}$ értékeket veszi fel, tehát
\[
\displaystyle \sum_{i, j} x_{i, j} \rightarrow \max!
\]
A feltételek és a célfüggvény ismeretében ez egy lineáris programozási feladatot eredményez.

\section{Folyamatszabályozási modell}

A folyamatszabályozban megkülönböztetnek diszkrét- és folytonos modelleket \cite{bikfalvi}.
Az üzleti folyamatok esetében mindkettő használható, attól függően, hogy a konkrét bekövetkezéseket vesszük-e figyelembe, vagy pedig egy valószínűségi alapon történő, folytonos közelítést szeretnénk-e.

A diszkrét leírási mód közvetlenül visszaadja a Petri-hálós reprezentációt.

Amennyiben a folyamatok mennyiségi jellemzőit szeretnénk folytonosan vizsgálni, úgy a rendszer elemeit (ebben az esetben a Petri-háló csomópontjait), mint függvényeket kezelhetjük.

Ha ez a függvénykapcsolat egy lineáris kapcsolatot ad meg (tehát a bemenetek súlyozva jelennek meg a kimeneteken), akkor a folyamfeladathoz jutunk vissza.

A rendszerbe érkező jelek tulajdonképpen a feldolgozandó elemek (tokenek) mértékét írják le.
Ekkor a linearitás feltételezése is elegendő lehet.

Tegyük fel, hogy az üzleti folyamat bonyolultabb elemeket is tartalmazhat.
A szabályzás módja lehet például az, hogy ha a rendszerben egy adott áteresztőképességet szeretnénk tartani.
Ennek a szabályzásához 3 alapelemre van szükségünk.
\begin{itemize}
\item Legyen egy $P$ (\textit{proportional}) szabályzónk, amelyik a beérkezett tokenek számával arányosan adja a kimenetét.
\item Az $I$ (\textit{integral}) a korábban beérkezett tokenek számától teszi függővé a feldolgozás sebességét. (Ezzel így például a terheltebb feldolgozási csomópontokat modellezhetjük.)
\item A $D$ (\textit{derivative}) a tokenek mennyiségének változásával arányos.
\end{itemize}
Jelöljük $e(t)$-vel a $t$ időpontban a bemenetre érkező tokenek mennyiségét.
A $K_P$, $K_I$ és $K_D$ valós konstansok jelöljék a szabályozáshoz használt arányos, integráló és differenciáló tag súlyát.
Amennyiben egy $u(t)$ kimenetet várunk el, úgy ezeket az értékeket az alábbi összefüggés alapján kell meghatározni:
\[
u(t) =
K_P \cdot e(t) +
K_I \displaystyle \int_{0}^{t} \! e(\tau) \; \textrm{d}\tau +
K_D \dfrac{\partial e(t)}{\partial t}.
\]
Ez így egy lineáris szabályzót eredményez, mivel az ismeretlen paraméterek lineáris viszonyban vannak a bemeneti jelzéssel, annak integráljával és deriváltjával.

Ez a szabályzókör legegyszerűbb modellje (amely már az említett összetevőket figyelembe veszi).
Egy általános üzleti folyamat modellezése esetén a topológia és a csomópontokhoz tartozó függvények ismeretében lehet hasonló formában modellezni az optimalizálási problémát.

\bibliographystyle{acm}
\bibliography{flow}

\end{document}
