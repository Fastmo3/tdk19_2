%TODO Add motivation, structure, results


\chapter{Bevezetés}
Ugyan a BPEL nyelv létrejötte elsődlegesen a Web szolgáltatások területéhez kapcsolódik, a nyelv mint általános workflow leíró nyelv, más alkalmazási témakörhöz is kapcsolható. A BPEL szerepét fontosságát, jól mutatja az a tény is,, hogy  amint a bevezetésben is láthattuk, igen gazdag irodalom található az egyes alkalmazási területekről és speciális szabvány kiegészítésekről.  A  BPEL aktualitását jelzi, hogy a megvalósító motorok köre is folyamatosan bővül. Ugyan már lassan 15 év eltelt a szabvány BPEL bevezetése óta, a meglévő nagyobb rendszerek (Oracle BPEL Process Manager, IBM WebSphere Process Server, Microsoft BizTalk Server, SAP SAP Exchange Infrastructure ) alternatívájaként  most is jelennek meg új végrehajtó motor implementációk. A Wikipédia forrása szerint \cite{wikiBpelList} a közelmúltban az alábbi szabad szotver implementációk születtek:
\begin{center}
\begin{tabular}{|c|c|c|c|}
\hline
\textbf{Termék neve} & \textbf{Fejlesztő} & \textbf{Megjelenés éve} & \textbf{Licensz}\\
JBPM & JBoss & 2016 & Apache\\
\hline
Apache ODE & ASF & 2016 & Apache\\
\hline
Activiti & Alfresco & 2014 & Apache\\
\hline
\end{tabular}
\end{center}

Ugyan napjainkra már több BPEL motor elérhető és használatos, ennek ellenére a BPEL szerkesztők és különösen a BPEL validációs rendszerek köre igen szegényes. Ezen tapasztalatokból kiindulva a dolgozat célja egy olyan BPEL validációs rendszer elkészítése, amely a BPEL rendszerek egyik fontos tulajdonságát, a terhelés korlátosságát (bounded model)   vizsgálja. A korlátosság azt jelzi, hogy minden csomópontban véges számú terhelés, feladat intenzitás jelentkezik bármely időpontban. Ha a rendszer nem teljesíti ezt a kritériumot, akkor túlcsordul valamely megmunkáló/tároló helyen a rendszer. Az elemzés során a korlátosság ténye mellett, a korlát értékei is fontos vizsgálandó jellemző. 

A meglévő tervezői rendszerekben legtöbbször szimulációval történik a főbb paraméterek, a korlátosság vizsgálata. Ezen megközelítésnek rendszerint két problémája van: az vizsgálat teljessége (azaz valóban minden lehetséges esetet áttekintettük-e) illetve a végrehajtási idő (a szimulációk futtatása hosszabb időt is igénybe vehet).  A dolgozatban a BPEL folyamatok Petri-háló alapú vizsgálatát végzem el. A Petri-háló alapú reprezentáció egy elfogadott és többek által alkalmazott megközelítés. A kidolgozott rendszer inputként egy  BPEL modell leírását várja és kimenetként az elemzés eredményét illetve a folyamatok nyomkövetését  adja vissza. A dolgozat főbb eredményei:
\begin{itemize}
\item BPEL folyamatok Petri háló formalizmusra történő konverziója
\item LP alapú végesség vizsgálat a Petri hálón
\item folyamatok grafikus nyomon követése, szimuláció 
\end{itemize}

