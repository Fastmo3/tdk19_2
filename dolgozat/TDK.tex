\documentclass[12pt,a4paper]{book}
\usepackage[utf8]{inputenc}
\usepackage[magyar]{babel}
\usepackage[T1]{fontenc}
\usepackage{amsmath}
\usepackage{amsfonts}
\usepackage{amssymb}
\usepackage[left=2cm,right=2cm,top=2cm,bottom=2cm]{geometry}
\date{\vspace{-5ex}}

%TODO a petri háló írása legyen konzisztens

\title{BPEL folyamatok Petri-háló alapú reprezentációja és szimulációja}

\begin{document}

\maketitle

\chapter{Bevezetés}

A dolgozat célja egy működő Petri háló  megjelenítő, szerkesztő %(és futtató?)
alkalmazás fejlesztése, és bemutatása. %TODO: (folyt)

\chapter{BPEL folyamatok}

A BPEL (Buisness Process Execution Language)üzleti folyamatok végrehajtó nyelve.
Az OASIS által kezelt XML alapú szabványt használ. 

\chapter{Petri-hálók és alkalmazásaik}

A Petri-háló egy matematikai leírómodell elosztott rendszerek bemutatására.
A modellt Carl Adam Petri készítette.
A modell nagyon hasonlít a programozók körében elterjedt folyamat ábrára.
A háló irányított élekből, helyekből és átmenetekből (\textsl{mint elemek}) áll.
Az élek csak két különböző típusú elem között állhatnak.
A helyeken pontok, ún. tokenek állhatnak.
A tokenek csak diszkrét számban fordulhatnak elő egy helyen, és a token átvitele atomi folyamat, azaz nem félbeszakítható.
A tokenek elláthatóak attribútummal is ilyen esetben a tokeneket "kiszínezzük" és színezett petri hálóról beszélünk. (ld. 2.2.) %TODO (LINK!)

Formálisan  a petri háló egy \((P, T, F)\) rendezett hármas, ahol 
\begin{itemize}
\item $P$ egy véges \textsl{hely} halmaz ,
\item $T$ egy véges átmenet halmaz,
\item $F\subseteq (P\times T\cup T\times P)$ pedig egy folyamat reláció
\end{itemize}

\section{Színezett Petri-hálók}

Konkurrens folyamatok modellezése

\chapter{Az üzleti folyamatok elemeinek leképzése}

%TODO Ide kellene felsorolni, és részletesen leírni, hogy a BPEL egyes elemeinek milyen Petri-háló feleltethető meg.

\chapter{Szimulációs keretrendszer}

%TODO Be kell mutatni a C# nyelvű alkalmazást.

\section{Elvárások az alkalmazással szemben}

%TODO Itt kellene röviden áttekinteni az alkalmazással szemben támasztott követelményeket.

\section{Az alkalmazás felépítése}

%TODO Osztály és blokkdiagramok formájában be kellene mutatni, hogy milyen fő elemekből épül fel az alkalmazás.

\section{C\# implementáció}

%TODO Meg kellene mutatni, hogy milyen API és újrahasznosítható elemek készültek el.

\section{Tesztelés, tapasztalatok}

%TODO Itt kifejezetten az alkalmazás szemszögéből (nem pedig üzleti folyamatokra vonatkozóan) kellene bemutatni az alkalmazást.

\chapter{Komplex folyamatok vizsgálata}

\section{Első példa üzleti folyamat}

%TODO A címet majd nyilván át kell írni. Itt szerepelne egy bonyolultabb üzleti folyamat, és a konverzió eredménye.

\section{Második példa üzleti folyamat}

%TODO Hasonló az előző szakaszhoz, csak másik példával.

\chapter{Összegzés}

%TODO Leírni a dolgozatban elért eredményeket, és a további terveket!

\chapter{Hivatkozások}

%TODO A felhasznált hivatkozásokat már az elején célszerű összegyűjteni!

\end{document}
