\documentclass[12pt,a4paper]{book}
\usepackage[utf8]{inputenc}
\usepackage[magyar]{babel}
\usepackage[T1]{fontenc}
\usepackage{amsmath}
\usepackage{amsfonts}
\usepackage{amssymb}
\usepackage{listings} %TODO configure for c#
\usepackage[left=2cm,right=2cm,top=2cm,bottom=2cm]{geometry}
\date{\vspace{-5ex}}

%TODO a petri háló írása legyen konzisztens a C# ahol lehet zenei # el legyen hivatkozva az az "official"
%TODO a tranzició rövid vagy hosszú i?


\title{BPEL folyamatok Petri-háló alapú reprezentációja és szimulációja}

\begin{document}

\maketitle
\newpage
\tableofcontents

\chapter{Bevezetés}

A dolgozat célja egy működő Petri háló  megjelenítő, szerkesztő %(és futtató?)
alkalmazás fejlesztése, és bemutatása, mindezt úgy, hogy a generált háló BPEL (WS-BPEL) elemekből automatikusan generálódjon, illetve megfelelő helyeken felhasználói inputot megadva a szimulációt folytatni tudja.  %TODO: (folyt)

\chapter{BPEL folyamatok}

A BPEL (Buisness Process Execution Language)üzleti folyamatok végrehajtó nyelve.
Az OASIS által kezelt XML alapú szabványt használ. 
A dokumentum felépítésében egy XML dokumentum, mely a WS-BPEL szabvány szerint validált. A BPEL programok szerkezetét célszerű egy mintával áttekinteni, a jobb megértéshez. %TODO link: http://docs.oasis-open.org/wsbpel/2.0/OS/wsbpel-v2.0-OS.html chapter 5 section 1
Vegyük a következő példát.  \textsl{Adott egy online rendelést felvevő cég. A cég egy automata segítségével generál számlákat A számla az ár kiszámítása, a futár kiválasztása, és a szükséges termelés ütemezése után kerül kiállításra.} A lépéseket az alábbi ábra mutatja be: %TODO insert fig1.png 
Az ábrán a téglalapok egy rész processzt jelentenek. Az egy blokkban különállóak pedig konkurens proceszeket. A szaggatott vonal szekvenciát jelöl, míg a teli/sima pedig vezérlő linkek, a konkurens processzek szinkronizációját, várakoztatását lehet velük megoldani. Az ábra nem képez átíratot, csak mint egy standard érthető vizualizáció segíti a megértést. 

A program következő része egy WSDL szabvány ami a portot adja meg a processz számára:
\newpage

\begin{verbatim}
<wsdl:definitions
   targetNamespace="http://manufacturing.org/wsdl/purchase"
   xmlns:sns="http://manufacturing.org/xsd/purchase"
   xmlns:pos="http://manufacturing.org/wsdl/purchase"
   xmlns:wsdl="http://schemas.xmlsoap.org/wsdl/"
   xmlns:plnk="http://docs.oasis-open.org/wsbpel/2.0/plnktype"
   xmlns:xsd="http://www.w3.org/2001/XMLSchema">
   
   <wsdl:types>
      <xsd:schema>
         <xsd:import namespace="http://manufacturing.org/xsd/purchase"
          schemaLocation="http://manufacturing.org/xsd/purchase.xsd" />
      </xsd:schema>
   </wsdl:types> 

   <wsdl:message name="POMessage">
      <wsdl:part name="customerInfo" type="sns:customerInfoType" />
      <wsdl:part name="purchaseOrder" type="sns:purchaseOrderType" />
   </wsdl:message>

   <wsdl:message name="InvMessage">
      <wsdl:part name="IVC" type="sns:InvoiceType" />
   </wsdl:message>

   <wsdl:message name="orderFaultType">
     <wsdl:part name="problemInfo" element="sns:OrderFault"/>
   </wsdl:message>

   <wsdl:message name="shippingRequestMessage">
      <wsdl:part name="customerInfo" element="sns:customerInfo" />
   </wsdl:message>

   <wsdl:message name="shippingInfoMessage">
      <wsdl:part name="shippingInfo" element="sns:shippingInfo" />
   </wsdl:message>

   <wsdl:message name="scheduleMessage">
      <wsdl:part name="schedule" element="sns:scheduleInfo" />
   </wsdl:message> 

   <!-- portTypes supported by the purchase order process -->

   <wsdl:portType name="purchaseOrderPT">
      <wsdl:operation name="sendPurchaseOrder">
         <wsdl:input message="pos:POMessage" />
         <wsdl:output message="pos:InvMessage" />
         <wsdl:fault name="cannotCompleteOrder"
            message="pos:orderFaultType" />
      </wsdl:operation>
   </wsdl:portType>
   
   <wsdl:portType name="invoiceCallbackPT">
      <wsdl:operation name="sendInvoice">
         <wsdl:input message="pos:InvMessage" />
      </wsdl:operation>
   </wsdl:portType>

   <wsdl:portType name="shippingCallbackPT">
      <wsdl:operation name="sendSchedule">
         <wsdl:input message="pos:scheduleMessage" />
      </wsdl:operation>
   </wsdl:portType>

   <!-- portType supported by the invoice services -->

   <wsdl:portType name="computePricePT">
      <wsdl:operation name="initiatePriceCalculation">
         <wsdl:input message="pos:POMessage" />
      </wsdl:operation>      
      
      <wsdl:operation name="sendShippingPrice">
         <wsdl:input message="pos:shippingInfoMessage" />
      </wsdl:operation>
   </wsdl:portType>

   <!-- portType supported by the shipping service -->

   <wsdl:portType name="shippingPT">
      <wsdl:operation name="requestShipping">
         <wsdl:input message="pos:shippingRequestMessage" />
         <wsdl:output message="pos:shippingInfoMessage" />
         <wsdl:fault name="cannotCompleteOrder"
            message="pos:orderFaultType" />
      </wsdl:operation>
   </wsdl:portType>

   <!-- portType supported by the production scheduling process -->

   <wsdl:portType name="schedulingPT">
      <wsdl:operation name="requestProductionScheduling">
         <wsdl:input message="pos:POMessage" />
      </wsdl:operation>
      
      <wsdl:operation name="sendShippingSchedule">
         <wsdl:input message="pos:scheduleMessage" />
      </wsdl:operation>
   </wsdl:portType> 

   <plnk:partnerLinkType name="purchasingLT">
      <plnk:role name="purchaseService"
         portType="pos:purchaseOrderPT" />
   </plnk:partnerLinkType> 

   <plnk:partnerLinkType name="invoicingLT">
      <plnk:role name="invoiceService"
         portType="pos:computePricePT" />
      <plnk:role name="invoiceRequester"
         portType="pos:invoiceCallbackPT" />
   </plnk:partnerLinkType> 

   <plnk:partnerLinkType name="shippingLT">
      <plnk:role name="shippingService"
         portType="pos:shippingPT" />
      <plnk:role name="shippingRequester"
         portType="pos:shippingCallbackPT" />
   </plnk:partnerLinkType> 

   <plnk:partnerLinkType name="schedulingLT">
      <plnk:role name="schedulingService"
         portType="pos:schedulingPT" />
   </plnk:partnerLinkType>
</wsdl:definitions>
\end{verbatim}

A szükséges portok és linkek definiálása után most a rendelés processzét definiáljuk hasonlóképp. 
\begin{verbatim}
<process name="purchaseOrderProcess"
   targetNamespace="http://example.com/ws-bp/purchase"
   xmlns="http://docs.oasis-open.org/wsbpel/2.0/process/executable"
   xmlns:lns="http://manufacturing.org/wsdl/purchase"> 
   
   <documentation xml:lang="EN">
      A simple example of a WS-BPEL process for handling a purchase
      order.
   </documentation> 

   <partnerLinks>
      <partnerLink name="purchasing"
         partnerLinkType="lns:purchasingLT" myRole="purchaseService" />
      <partnerLink name="invoicing" partnerLinkType="lns:invoicingLT"
         myRole="invoiceRequester" partnerRole="invoiceService" />
      <partnerLink name="shipping" partnerLinkType="lns:shippingLT"
         myRole="shippingRequester" partnerRole="shippingService" />
      <partnerLink name="scheduling"
         partnerLinkType="lns:schedulingLT"
         partnerRole="schedulingService" />
   </partnerLinks> 

   <variables>
      <variable name="PO" messageType="lns:POMessage" />
      <variable name="Invoice" messageType="lns:InvMessage" />
      <variable name="shippingRequest"
         messageType="lns:shippingRequestMessage" />
      <variable name="shippingInfo"
         messageType="lns:shippingInfoMessage" />
      <variable name="shippingSchedule"
         messageType="lns:scheduleMessage" />
   </variables> 

   <faultHandlers>
      <catch faultName="lns:cannotCompleteOrder"
         faultVariable="POFault"
         faultMessageType="lns:orderFaultType">
         <reply partnerLink="purchasing"
            portType="lns:purchaseOrderPT"
            operation="sendPurchaseOrder" variable="POFault"
            faultName="cannotCompleteOrder" />
      </catch>
   </faultHandlers> 

   <sequence>
      <receive partnerLink="purchasing" portType="lns:purchaseOrderPT"
         operation="sendPurchaseOrder" variable="PO"
         createInstance="yes">
         <documentation>Receive Purchase Order</documentation>
      </receive>

      <flow>
         <documentation>
            A parallel flow to handle shipping, invoicing and
            scheduling
         </documentation>
         <links>
            <link name="ship-to-invoice" />
            <link name="ship-to-scheduling" />
         </links>
         <sequence>
            <assign>
               <copy>
                  <from>$PO.customerInfo</from>
                  <to>$shippingRequest.customerInfo</to>
               </copy>
            </assign>
            <invoke partnerLink="shipping" portType="lns:shippingPT"
               operation="requestShipping"
               inputVariable="shippingRequest"
               outputVariable="shippingInfo">
               <documentation>Decide On Shipper</documentation>
               <sources>
                  <source linkName="ship-to-invoice" />
               </sources>
            </invoke>
            <receive partnerLink="shipping"
               portType="lns:shippingCallbackPT"
               operation="sendSchedule" variable="shippingSchedule">
               <documentation>Arrange Logistics</documentation>
               <sources>
                  <source linkName="ship-to-scheduling" />
               </sources>
            </receive>
         </sequence>
         <sequence>
            <invoke partnerLink="invoicing"
               portType="lns:computePricePT"
               operation="initiatePriceCalculation"
               inputVariable="PO">
               <documentation>
                  Initial Price Calculation
               </documentation>
            </invoke>
            <invoke partnerLink="invoicing"
               portType="lns:computePricePT"
               operation="sendShippingPrice"
               inputVariable="shippingInfo">
               <documentation>
                  Complete Price Calculation
               </documentation>
               <targets>
                  <target linkName="ship-to-invoice" />
               </targets>
            </invoke>
            <receive partnerLink="invoicing"
               portType="lns:invoiceCallbackPT"
               operation="sendInvoice" variable="Invoice" />
         </sequence>
         <sequence>
            <invoke partnerLink="scheduling"
               portType="lns:schedulingPT"
               operation="requestProductionScheduling"
               inputVariable="PO">
               <documentation>
                  Initiate Production Scheduling
               </documentation>
            </invoke>
            <invoke partnerLink="scheduling"
               portType="lns:schedulingPT"
               operation="sendShippingSchedule"
               inputVariable="shippingSchedule">
               <documentation>
                  Complete Production Scheduling
               </documentation>
               <targets>
                  <target linkName="ship-to-scheduling" />
               </targets>
            </invoke>
         </sequence>
      </flow>
      
      <reply partnerLink="purchasing" portType="lns:purchaseOrderPT"
         operation="sendPurchaseOrder" variable="Invoice">
         <documentation>Invoice Processing</documentation>
      </reply>
   </sequence>
</process>
\end{verbatim}
 
 A kódban szereplő \texttt{<partnerLinks>} tartalmaz mindent (így közvetetten mindenkit) amik kapcsolatba kerül a processzel. Az elnevezés tükrözi a résztvevő partit, valamint a résztvevő feladatát, szándékát. A \texttt{<variables>} a változókat tartalmazza, míg a \texttt{<faultHandlers>} a hibakezelőket. A hibakezelés egy try-catch-finally "hibakezelőblokk" helyett az XML mentalitását tükröző módon kerül lekezelésre, a handlerek által. A kód többi része a processz standard definíciójához tartozik. A példák alapján elmondható, hogy a program a következő struktúra szerint épül fel.
\begin{itemize}
\item \textbf{Definíció: } a processz neve, névtere és különféle sémahívások, majd bővítmények, importok
\item \textbf{PartnerLinkek: } A megfelelő partnerek hozzáadása attribútumokkal. 
\item \textbf{Változók}
\item \textbf{Hibakezelők}
\item\textbf{Eseménykezelők}
\end{itemize}

\chapter{Petri-hálók és alkalmazásaik}
\section{Az alap Petri-hálók}
A Petri-háló egy matematikai leírómodell elosztott rendszerek bemutatására.
A modellt Carl Adam Petri készítette.
A modell nagyon hasonlít a programozók körében elterjedt folyamat ábrára.
A háló irányított élekből, helyekből és átmenetekből (\textsl{mint elemek}) áll.
Az élek csak két különböző típusú elem között állhatnak.
A helyeken pontok, ún. tokenek állhatnak.
A tokenek csak diszkrét számban fordulhatnak elő egy helyen, és a token átvitele atomi folyamat, azaz nem félbeszakítható.
A tokenek elláthatóak attribútummal is ilyen esetben a tokeneket "kiszínezzük" és színezett petri hálóról beszélünk. (ld. 2.2.) %TODO (LINK!)

%TODO cite: https://www.abhishekhalder.org/PetriNetReport.pdf__vq9vUXcOjS%2BawepkKcMLeAA64c19da3fb4d67754c3fe7eba8ce1187
Az alap Petri háló egy biparit, irányított és súlyozott multigráf $PN(P,T,A,W,S)$, ahol 
\begin{itemize}
\item $P=\{ p_1,p_2,\ldots ,p_N \}:$ a helyek véges halmaza ,
\item $T=\{ t_1,t_2,\ldots ,t_M\}:$ egy véges tranzició halmaz,
\item $P\cap T = \emptyset$
\item $A \subseteq P\times T \cup T\times P:$ az élek halmaza,
\item $W: F\Rightarrow N^+:$ az élsúlyok halmaza
\item $S: P\Rightarrow N^+:$ a kezdőállapot.
\end{itemize}

\section{Színezett Petri-hálók}

Az elemi színezett háló felírható, mint egy oly struktúra, ami: $CPN(P,T,A,\Sigma ,V,C,G,E,S)$, ahol 
\begin{itemize}
\item $P=\{ p_1,p_2,\ldots ,p_N \}:$ a helyek véges halmaza ,
\item $T=\{ t_1,t_2,\ldots ,t_M\}:$ egy véges tranzició halmaz,
\item $A \subseteq P\times T \cup T\times P:$ az élek halmaza,
\item $\Sigma:$ a színek halmazainak halmaza, 
\item $V:$ a változók halmaza, ahol $\forall v\in V:$ változóhoz egy $Type[v] \in \Sigma $ típus rendelhető,
\item $C: P\rightarrow \Sigma :$ a helyekhez színeket rendelő függvény,
\item $G: T\rightarrow EXPR_V:$ az egyes tranzíciókhoz kapcsolódó validációs, ellenőrzési kifejezés (logikai értékű)
\item $E: A\rightarrow EXPR_V:$ z  egyes élekhez kapcsolódó kifejezés, amely a kapcsolódó hely színhalmazához tartozó értéket vehet fel
\item $S: P\Rightarrow N^+:$ a kezdőállapot.
\end{itemize}

Adott $CPN(P,T,A,\Sigma ,V,C,G,E,S)$ színezett hálóhoz az alábbi kezelő funkciók köthetőek: 
\begin{itemize}
\item $M(p):$ a jelölő (marker) függvény, melynek értéke a $p$ helyhez kapcsolódó tokenek halmaza. Színezett Petri háló esetén az $M(p)$ elemek színeinek illeszkedni kell a $C(p)$ színhalma
\item $M_0(p):$  helyek induló tokenkészlete
\item $Var(t):$ a tranzíciók viselkedését leíró változók halmaza
\item $b(v):$ a adott v változó értékét megadó kifejezés, ahol $b(v) \in Type[v]$
\end{itemize}

Egy adott $t$ tranzíció esetén a $Var(t)$ kifejezés a tranzícióhoz rendelt változók együttese, ahol a változók a $G(t)$ vagy $E$(a: t-hez kötődő él) kifejezésekben szerepelnek.
\begin{equation*}
Var(t)=\begin{cases}
\{n,d\} &\text{if } t=SendPacket\\
\{n,d,success\} &\text{if } t= TransmitPacket\\
\{n,d,k,data\} &\text{if } t=ReceivePacket\\
\{n,success\} &\text{if } t=TrancmitAck\\
\{n,k\} &\text{if }t=ReceiveAck
\end{cases}
\end{equation*}

A hálóban egy tranzíció akkor engedélyezett (ready), ha minden bemenő helyeknél a kívánt tokenszám megtalálható.   Jelölt hálók esetében:
$$M'(p)=M(p)-I(p,t)+O(p,t): \forall p\in P,$$ ahol 
\begin{itemize}
\item $I:F\Rightarrow N^+:$  bejövő áram intenzitás
\item $O:F\Rightarrow N^+:$ kimenő áram intenzitás

\end{itemize}
A hierarchikus CPN rendszerben az átláthatóság növelése érdekében összefogó modulokat is lehet alkalmazni. Egy modul más elemi egységek együttese, konténere. %TODO Insert "rendszerséma 3 modullal.png" & 2A receiver belső szerkezete.png"

A moduloknál fontos szerepet kapnak az átadó helyek, melyeken keresztül a tokenek bejöhetnek a modulba illetve kiléphetnek a modulból. Az ilyen port jellegű helyek lehetnek bemeneti portok (IN) illetve kimeneti portok (OUT).  

A CPN rendszerek egyik hasznos tulajdonsága, hogy lehetőséget adnak a felépített modell formális ellenőrzésére, validálására és értékelésére. A formális ellenőrzés egyik leggyakoribb eszköze az állapottér (state space)  modell, ahol az állapottér egy olyan irányított gráf, melyben a csomópontok a háló egy lehetséges  M(CPN) jelölési állapota. Azaz a háló struktúrája rögzített, de az egyes elemeknél a tokenek és változók halmaza, azok állapota változhat. A véges állapottér modellt rendszerint szimulációkal állítják elő. 

Az állapottér modellből kiindulva további elemzésekre ad lehetőséget a komponens gráf modell (SCC graph:  strongly-connected-component graph) formalizmus. Az  SCC gráfból a rendszer általános viselkedési szabályaira lehet következtetni. Az SCC gráf olyan gráf, melynek csomópontjai  az állapottér azon diszjunkt részhalmazai, ahol egy részhalmaz bármely két elemére igaz, hogy az egyik elem  elérhető a másikból. 

Az elemzések során az alábbi főbb tulajdonságok elemzésére szoktak kitérni:
\begin{itemize}
\item Reachability Properties
\item Boundedness Properties
\item Home Properties 
\item Liveness Properties
\item Fairness Properties
\end{itemize}
%TODO insert mintaállapottér 1-2-3.png here


\chapter{Az üzleti folyamatok elemeinek leképzése}

%TODO Ide kellene felsorolni, és részletesen leírni, hogy a BPEL egyes elemeinek milyen Petri-háló feleltethető meg.

%TODO Megnézni, hogy az egyes elemek esetében milyen alternatívák lennének a leképzésre!
\section{A leképzés menete}
A leképzéskor elsődlegesen az aktivitás elemei kerülnek a figyelem középpontjába, a séma és különböző definíciók csak másodlagos helyet kapnak. Ez azért lényeges, mert a séma nem egy dinamikus folyamatot ír le, hanem az adott folyamat és eleminek tulajdonságát. Mivel a Petri-háló egy folyamat személtetésére lett létrehozva ezért főleg az aktív elemek a szignifikánsak a konverzió során, tehát leképzés során az XML validációhoz szükséges statikus elemeket ignoráljuk.  Az alábbi elemeknek feleltetünk meg egy egy részhálót:
\section{\texttt{<receive>}}
\texttt{<receive>} A recieve egy megfelelő üzenet után engedi a folyamatot továbbhaladni, így várakoztatáshoz használható. A \texttt{<receive>} teljes terjedelmében:\\
\begin{verbatim}
<receive partnerLink="NCName"
   portType="QName"?
   operation="NCName"
   variable="BPELVariableName"?
   createInstance="yes|no"?
   messageExchange="NCName"?
   standard-attributes>
   standard-elements
   <correlations>?
      <correlation set="NCName" initiate="yes|join|no"? />+
   </correlations>
   <fromParts>?
      <fromPart part="NCName" toVariable="BPELVariableName" />+
   </fromParts>
</receive>
\end{verbatim}
A hálóban ezt egy tranzicióval könnyedén megoldhatjuk, hisz csak egy specifikus msg token kell a továbblépéshez, és a többit addig az előző helyen parkoltatja. 

\section{\texttt{<reply>}}
\texttt{<reply>} A reply egy üzenetküldő elem, ami \texttt{<receive>; <onMessage>;<onEvent>} események után léphet akcióba. 
\begin{verbatim}
<reply partnerLink="NCName"
   portType="QName"?
   operation="NCName"
   variable="BPELVariableName"?
   faultName="QName"?
   messageExchange="NCName"?
   standard-attributes>
   standard-elements
   <correlations>?
      <correlation set="NCName" initiate="yes|join|no"? />+
   </correlations>
   <toParts>?
      <toPart part="NCName" fromVariable="BPELVariableName" />+
   </toParts>
</reply>
\end{verbatim}
A lekezelése az előző példával analóg módon, annyi különbséggel, hogy a tokenek nem parkolnak, hanem tovább mennek és a tranzíció csak akkor generál új tokent ha üzenetet kap egy ágról.

\section{\texttt{<invoke>}}
\texttt{<invoke>} Egy Bpel vagy épp egy webszolgáltatás meghívására szolgál és definiálja a szolgáltatás feladatát is. 
\begin{verbatim}
<invoke partnerLink="NCName"
   portType="QName"?
   operation="NCName"
   inputVariable="BPELVariableName"?
   outputVariable="BPELVariableName"?
   standard-attributes>
   standard-elements
   <correlations>?
      <correlation set="NCName" initiate="yes|join|no"?
         pattern="request|response|request-response"? />+
   </correlations>
   <catch faultName="QName"?
      faultVariable="BPELVariableName"?
      faultMessageType="QName"?
      faultElement="QName"?>*
      activity
   </catch>
   <catchAll>?
      activity
   </catchAll>
   <compensationHandler>?
      activity
   </compensationHandler>
   <toParts>?
      <toPart part="NCName" fromVariable="BPELVariableName" />+
   </toParts>
   <fromParts>?
      <fromPart part="NCName" toVariable="BPELVariableName" />+
   </fromParts>
</invoke>
\end{verbatim}
%TODO ebben nem vbagyok biztos hogy hogy lehet. Átad egy start tokent egy részhálónak??

\section{\texttt{<assign>}}
\texttt{<assign>} Egy változó értékadására szolgáló esemény. Ellentétben egy imperatív értékadással egy assign blokkban bármennyi értékadás, másolás történhet, amíg azt a kliens kezelni tudja, így logikailag egy egységbe zárja a műveleteket.  
\begin{verbatim}
<assign validate="yes|no"? standard-attributes>
   standard-elements
   (
   <copy keepSrcElementName="yes|no"? ignoreMissingFromData="yes|no"?>
      from-spec
      to-spec
   </copy>
   |
   <extensionAssignOperation>
      assign-element-of-other-namespace
   </extensionAssignOperation>
   )+
</assign>
\end{verbatim}
Az assign nagyon egszerűen átírható egy tranzicióra ami a megfelelő tokenek színét módosítja. 

\section{\texttt{<validate>}}
\texttt{<validate>} Egy sémára validálja az XML (BPEL) file-t. 
\begin{verbatim}
<validate variables="BPELVariableNames" standard-attributes>
   standard-elements
</validate>
\end{verbatim}
\texttt{<throw>}Egy rész processzen belül fault generálására szolgál. 
\begin{verbatim}
<throw faultName="QName"
   faultVariable="BPELVariableName"?
   standard-attributes>
   standard-elements
</throw>
\end{verbatim}
Nagyon egyszerűen egy fault tokent generáló tranzició komponens. Explicit hálórésze nincs, hanem a megfelelő inputtokenek megléte vagy hiánya generálja egy tranzició során. 

\section{\texttt{<wait>}}
\texttt{<wait>} Időre vonatkoztatva várakoztat. Például 5000 tick vagy 14:00:23 (hh:mm:ss)
\begin{verbatim} 
<wait standard-attributes>
   standard-elements
   (
   <for expressionLanguage="anyURI"?>duration-expr</for>
   |
   <until expressionLanguage="anyURI"?>deadline-expr</until>
   )
</wait>
\end{verbatim}
Megadható egy részhálóval ami valójában egy oszcillátor és a megfelelő iteráció után folytat tokent küld. 

\section{\texttt{<empty>}}
\texttt{<empty>} No-op (no operations) event szinkronizációra szolgál.
\begin{verbatim}
<empty standard-attributes>
   standard-elements
</empty>
\end{verbatim}
Beiktatható egy semleges tranzició és hely.

\section{\texttt{<sequence>} }
\texttt{<sequence>} Sorozatot ad meg.
\begin{verbatim}
<sequence standard-attributes>
   standard-elements
   activity+
</sequence>
\end{verbatim} Egyszerűen csak tranziciók és helyek összefűzése. 

\section{\texttt{<if>}}
\texttt{<if>} Standard kétirányú elágazás. Logikai XPATH kifejezést vár. 
\begin{verbatim}
<if standard-attributes>
   standard-elements
   <condition expressionLanguage="anyURI"?>bool-expr</condition>
   activity
   <elseif>*
      <condition expressionLanguage="anyURI"?>bool-expr</condition>
      activity
   </elseif>
   <else>?
      activity
   </else>
</if>
\end{verbatim}
Egy tranzició, mely tokenek függvényében más felé küldi tovább, vagy generál tokeneket. Analóg módon egy Swithc Case elágazás is definiálható vele.

\section{\texttt{<while>}}
\texttt{<while>} While loop. Végre hajt amíg az iterációs feltétel igaznak értékelődik ki. 
\begin{verbatim}
<while standard-attributes>
   standard-elements
   <condition expressionLanguage="anyURI"?>bool-expr</condition>
   activity
</while>
\end{verbatim}
Egy tranzició, mely token függvényében a folyamat egy korábbi pontjára csatol vissza, vagy ép egy későbbire, a feltétel hamis logikai állapota esetén. A feltétel persze egy színes token jelenléte, vagy tokenek száma is lehet. 

\section{\texttt{<repeatUntil>}}
\texttt{<repeatUntil>} Egy do-while ciklusnak feleltethető annyi kivétellel, hogy akkor enged tovább, ha a feltétel igaz. 
\begin{verbatim}
<repeatUntil standard-attributes>
   standard-elements
   activity
   <condition expressionLanguage="anyURI"?>bool-expr</condition>
</repeatUntil>
\end{verbatim}
Az előzővel analóg módon megadható

\section{\texttt{<forEach>}}
\texttt{<forEach>} A kezdeti változó és a végváltozó különbsége +1 szer iteráltatja a gyerek elemet. Megadható párhuzamos feldolgozás is. Egy Complete condition segítségével megadható egy break utasítás ami kilép a forEachből. 
\begin{verbatim}
<forEach counterName="BPELVariableName" parallel="yes|no"
   standard-attributes>
   standard-elements
   <startCounterValue expressionLanguage="anyURI"?>
      unsigned-integer-expression
   </startCounterValue>
   <finalCounterValue expressionLanguage="anyURI"?>
      unsigned-integer-expression
   </finalCounterValue>
   <completionCondition>?
      <branches expressionLanguage="anyURI"?
         successfulBranchesOnly="yes|no"?>?
         unsigned-integer-expression
      </branches>
   </completionCondition>
   <scope ...>...</scope>
</forEach>
\end{verbatim} Egyszerű loop utasítás, azonban párhuzamosítás esetén a részhálóból megfelelő példányszámot generáltatunk. 

\section{\texttt{<pick>}}
\texttt{<pick>}Üzenetek várására vagy timeout eseményre figyel. Ezek bármelyike a szubprocessz végrehajtásához vezet. 
\begin{verbatim}
<pick createInstance="yes|no"? standard-attributes>
   standard-elements
   <onMessage partnerLink="NCName"
      portType="QName"?
      operation="NCName"
      variable="BPELVariableName"?
      messageExchange="NCName"?>+
      <correlations>?
         <correlation set="NCName" initiate="yes|join|no"? />+
      </correlations>
      <fromParts>?
         <fromPart part="NCName" toVariable="BPELVariableName" />+
      </fromParts>
      activity
   </onMessage>
   <onAlarm>*
      (
      <for expressionLanguage="anyURI"?>duration-expr</for>
      |
      <until expressionLanguage="anyURI"?>deadline-expr</until>
      )
      activity
   </onAlarm>
</pick>
\end{verbatim}

\section{\texttt{<flow>}}
\texttt{<flow>} konkurens elemek deklarálására szolgál. Linkek segítségével megadható függőségi viszony a gyerekek között. 
\begin{verbatim}
<flow standard-attributes>
   standard-elements
   <links>?
      <link name="NCName" />+
   </links>
   activity+
</flow>
\end{verbatim} 

\section{\texttt{<scope>}}
\texttt{<scope>} A gyerek elemek scope-ját lehet vele szabályozni.
\begin{verbatim}
<scope isolated="yes|no"? exitOnStandardFault="yes|no"?
   standard-attributes>
   standard-elements
   <partnerLinks>?
      ... see above under <process> for syntax ...
   </partnerLinks>
   <messageExchanges>?
      ... see above under <process> for syntax ...
   </messageExchanges>
   <variables>?
      ... see above under <process> for syntax ...
   </variables>
   <correlationSets>?
      ... see above under <process> for syntax ...
   </correlationSets>
   <faultHandlers>?
      ... see above under <process> for syntax ...
   </faultHandlers>
   <compensationHandler>?
      ...
   </compensationHandler>
   <terminationHandler>?
      ...
   </terminationHandler>
   <eventHandlers>?
      ... see above under <process> for syntax ...
   </eventHandlers>
   activity
</scope>
\end{verbatim}
Nem generál új elemet, csak a láthatósági, azaz visszacsatolási elemeket adja meg. 

\section{\texttt{<compensateScope>}}
\texttt{<compensateScope>}
\begin{verbatim}
<compensateScope target="NCName" standard-attributes>
   standard-elements
</compensateScope>
\end{verbatim}

\section{\texttt{<compensate>}}
\texttt{<compensate>}
\begin{verbatim}
<compensate standard-attributes>
   standard-elements
</compensate>
\end{verbatim}

\section{\texttt{<rethrow>}}
\texttt{<rethrow>}
\begin{verbatim}
<rethrow standard-attributes>
   standard-elements
</rethrow>
\end{verbatim}

\section{\texttt{<extensionActivity>}}
\texttt{<extensionActivity>}
\begin{verbatim}
<extensionActivity>
   <anyElementQName standard-attributes>
      standard-elements
   </anyElementQName>
</extensionActivity>
\end{verbatim}


\chapter{Szimulációs keretrendszer}

%TODO Be kell mutatni a C# nyelvű alkalmazást.

\section{Elvárások az alkalmazással szemben}

%TODO Itt kellene röviden áttekinteni az alkalmazással szemben támasztott követelményeket.
Az alkalmazás legfőbb feladata egy konverzió, BPEL és Petri-háló között. Ebből adódóan minden BPEL elemet le kell tudnia kezelni, illetve az azok közti összefüggéseket feltérképezni, és az összefüggés halmazból egy petri hálót előállítani. A hálót tudni kell megjeleníteni, valamint a hálón belüli mozgásokat rajzolni, és szükségesség esetén input file-t vagy felhasználói inputot kezelni. Ha a háló nem hozható létre, akkor azt tudatnia kell,és lehetőség szerint rövid indoklással alátámasztania. 

\section{Az alkalmazás felépítése}

%TODO Osztály és blokkdiagramok formájában be kellene mutatni, hogy milyen fő elemekből épül fel az alkalmazás.
Az alkalmazás logikailag a következő fő részekből áll:
\begin{itemize}
\item I/O module: Beolvassa az XML dokumentumot és felparseolja. Szükség esetén menti a kész hálót.
\item Conversion module: Átkonvertálja  a beolvasott dokumentumot.
\item Data Structure module: A saját típusú Petri elemeket kezeli, és adatszerkezeti implementációt tartalmaz. 
\item UI talker: A UI ra illeszti a megfelelő input mezőt, az abba felvitt értéket átadja a feldolgozó egységnek.
\item Graphics module: Az MSGL libraryre épül. Feladata a gráf rajzolása és megjelenítése animációval együtt. 
\item Computing module: A háló animációjához végzi a szükséges számításokat és időzítéseket. 
\end{itemize}

\section{C\# implementáció}

%TODO Meg kellene mutatni, hogy milyen API és újrahasznosítható elemek készültek el.

\section{Tesztelés, tapasztalatok}

%TODO Itt kifejezetten az alkalmazás szemszögéből (nem pedig üzleti folyamatokra vonatkozóan) kellene bemutatni az alkalmazást.

\chapter{Komplex folyamatok vizsgálata}

\section{Első példa üzleti folyamat}

%TODO A címet majd nyilván át kell írni. Itt szerepelne egy bonyolultabb üzleti folyamat, és a konverzió eredménye.

\section{Második példa üzleti folyamat}

%TODO Hasonló az előző szakaszhoz, csak másik példával.

\chapter{A hálón végezhető elemzések}
\section{Háló korlátosság és puffer kapacitási ellenőrzés}

A háló egy adott helye akkor tekinthető korlátos (bounded) helynek, ha bármely jelölésnél a tokenek száma az adott helynél nem megy egy adott korlát fölé. A Petri háló korlátos, ha minden helye korlátos hely.
A háló korlátossága az egyik leggyakoribb és legfontosabb minőségi jellemzője a Petri hálóknak. 

A háló alap tulajdonságainak , beleértve a korlátosságának az elemzésére több módszer is létezik, melyek közül kiemelhető a
\begin{itemize}
\item komponens / elérhetőségi gráf elemzése (SCC)
\item komponens / elérhetőségi gráf elemzése (SCC)
\item dekompozíciós módszerek 
\end{itemize}

A mátrix reprezentáció esetén transzformációs mátrixok segítségével írják fel a Petri háló dinamikáját. Az alapstruktúra az ú.n. incidencia M mátrixban kerül megadásra, melynek elemei az alábbi jelentéssel bírnak: $A_{ij}=a^+_{ij}-a^-_{ij}$, ahol 
\begin{itemize}
\item $a^+_{ij}: $ az élerősség az i. tranzícióból a j. kimeneti hely felé
\item $a^-_{ij}$ az élerősség az i. tranzícióhoz a j. bemeneti hely felől.
\end{itemize}

A mátrix alapvetően a tokenek számának a változását mutatja az egyes tranzició átmenetek esetére. A  Petri háló működési alapegyenlete a következő alakban adható meg: 
$$M_k=M_{k-1}+ Au_k$$
ahol $M_k$ jelöli a háló markereinek (tokenek) státuszát a $k.$ lépésben. Az $u$ vektor a helyek tüzelési státuszt írja le. 

A fenti modellen alapuló elérhetőség vizsgálatok felhasználhatóak a korlátosság elemzésére is. LINK!% [ Petri Nets: Properties, Analysis and Appl kat ions, http://people.disim.univaq.it/adimarco/teaching/bioinfo15/paper.pdf]  
A kapcsolódó  egyenletek hatékony,. lineáris programozási megoldását vizsgálta a LINK!  
%[JEAN B. LASSERRE PHILIPPE MAHEY Using linear programming in Petri net analysis :http://www.numdam.org/article/RO_1989__23_1_43_0.pdf] 
dolgozat is.

%TODO itt van egy kép, nem tudom lehet e direktben cite-olni????

\section{Saját modell}

Modellünkben korlátosságnak egy folyam-gráf megközelítését dolgoztuk ki.  A hálóban a következő típusú helyeket definiáljuk:
\begin{itemize}
\item forrás hely
\item nyelő hely
\item köztes hely
\end{itemize}
Feltesszük, hogy csak a köztes helyeken lehet tokeneket tárolni, csak ott vannak pufferek. A hálóban az élekhez egy $I_x$ token áramlás erősséget definiálunk, ahol x jelöli az él indexét. A forrás helyekhez egy $Q_x$ forrás erősség indexet adunk meg. A hálóban az alábbi kapacitás korlátokat vezetjük be:
\begin{itemize}
\item $C_x:$ az x. tranzíció maximális erőssége 
\item $C_y:$ az y. nyelő maximális folyam erőssége 
\end{itemize}

A tranzícióknál a bemenő élek vonatkozásában kétféle működési módot értelmezünk:
\begin{itemize}
\item AND-mód: akkor van tüzelés, ha minden bejövő élnél megvan az al elvárt tokenszám, van szinkron
\item OR-mód: akkor van tüzelés, ha megjelenik valamely bemeneten egy token, nincs szinkron  
\end{itemize}
A hálóban az alábbi megkötések élnek a folyamerősségekre
\begin{itemize}
\item forrás helyek esetén: $\sum_y I_y=Q_x$ ($y$: kimenő élek)
\item nyelő helyek esetén: $\sum_y I_y \leq C_x$ ($y$: bejövő élek)
\item belső helyek esetén: 
\item$\sum_{y(ki)} I_y\leq \sum_{y(be)} I_y$
\item tranziciók esetén: 
\begin{itemize}
\item $\sum_y I_y\leq C_x$ ($y:$ bejövő élek)
\item $\sum_{y(ki)} I_y = \sum_{y(be)} I_y$
\item $\forall$ kimenő $x,y$ élre $I_x=I_y$
\end{itemize}
\end{itemize}
az AND típusú tranzakciók esetén még ezen felül teljesül, hogy $\forall$ bejövő $x,y$ élre:\\
$I_x = I_y$.

Az egyes belső helyeken a pufferbe áramló tokenek  eredő intentitása:

$$F= \sum_{x(\text{belső hely})} \left( \sum_{y(\text{x bejövő él})} I_y - \sum_{y(\text{x kimenő él})} I_y \right)$$
Az F függvény 0 értéke esetén nincs szükség belső pufferre.

A fenti feladatot egy LP programozási feladatnak is tekinthető, ahol a változók az élek $I_x$ nem negatív intenzitásai és a célfüggvény:
$$F\Rightarrow \min$$ alakú.

\section{Számítási implementációja}
\begin{verbatim}
import networkx as nx
import matplotlib.pyplot as plt
import pulp as p

class cl_place:
    def __init__(self):
        self.id = -1
        self.inputs = []
        self.outputs = []
        self.tokens = []
        self.Q = 0
        self.border = 0


class cl_transition:
    def __init__(self):
        self.id = -1
        self.inputs = []
        self.outputs = []
        self.C = 0
        self.mode = ' '

class cl_link:
    def __init__(self):
        self.id = -1
        self.input = -1
        self.output = -1
        self.alfa = 0
        self.inner = 0

class cl_petri_net:
    def __init__(self,Plist,Tlist, Elist):
        self.NP = len(Plist)
        self.NT = len(Tlist)
        self.NN = self.NP + self.NT
            
        self.places = []
        for i in range(self.NP):
            po = cl_place()
            po.id = i + 1
            po.Q = Plist[i]
            if Plist[i] > 0:
                po.border = 1
            if Plist[i] < 0:
                po.border = 2            
            self.places.append(po)
            
       
        self.transitions = []
        for i in range(len(Tlist)):
            tr = cl_transition()
            tr.id = self.NP + 1 + i
            tr.C = Tlist[i][0]
            tr.mode = Tlist[i][1]
            self.transitions.append(tr)
        
        self.nodes_dict = dict()
        for pp in self.places:
            self.nodes_dict[pp.id] = pp
        for tt in self.transitions:
            self.nodes_dict[tt.id] = tt
        
        self.links = []
        for i in range(len(Elist)):
            ll = cl_link()
            ll.id = i
            ll.input = Elist[i][0]
            ll.output = Elist[i][1]
            if ll.input > self.NP:
                db = 0
                for j in range(len(Elist)):
                    if Elist[j][0] == ll.input:
                        db = db + 1
                ll.alfa = db
                ll.inner = 1
                for j in range(len(self.places)):
                    if self.places[j].id == ll.output:
                        if self.places[j].border != 0:
                            ll.inner = 0
            else:
                db = 0
                for j in range(len(Elist)):
                    if Elist[j][1] == ll.output:
                        db = db + 1
                ll.alfa = db
                ll.inner = 1
                for j in range(len(self.places)):
                    if self.places[j].id == ll.input:
                        if self.places[j].border != 0:
                            ll.inner = 0
            
            self.links.append(ll)
            print (ll.id," : ", ll.input,"->" , ll.output, " : ", ll.alfa)
        
        self.edges_dict = dict()
        for ee in self.links:
            self.edges_dict[ee.id] = ee
        
        for pp in self.places:
            for ee in self.links:
                if ee.input == pp.id:
                    pp.outputs.append(ee.id)
                if ee.output == pp.id:
                    pp.inputs.append(ee.id)
        for tr in self.transitions:
            for ee in self.links:
                if ee.input == tr.id:
                    tr.outputs.append(ee.id)
                if ee.output == tr.id:
                    tr.inputs.append(ee.id)
    
    
    def draw_net(self):
        graph = []
        for ll in self.links:
            graph.append((ll.input, ll.output))
        ncols = []
        mp = 0
        for ll in self.links:
            fnd = 0
            for i in range(mp):
                if ncols[i] == ll.input:
                    fnd = 1
            if fnd == 0:
                ncols.append(ll.input)
                mp = mp + 1
            fnd = 0
            for i in range(mp):
                if ncols[i] == ll.output:
                    fnd = 1
            if fnd == 0:
                ncols.append(ll.output)
                mp = mp + 1
        for i in range (len(ncols)):
            if ncols[i] > self.NP:
                if self.nodes_dict[ncols[i]].mode == 'A':
                    ncols[i] = 'lime'
                else:
                    ncols[i] = 'olivedrab'
            else:
                ncols[i] = 'gold'
            
        labs = dict()
        for x in self.places:
            labs[x.id] = str(x.id) + ":" + str(x.Q)
        for x in self.transitions:
            labs[x.id] = str(x.id) + ":" + str(x.C)

        G=nx.DiGraph()
        for edge in graph:
            G.add_edge(edge[0], edge[1])
        graph_pos = nx.circular_layout(G)
        #graph_pos = nx.spring_layout(G)
    
        nx.draw_networkx_nodes(G, graph_pos,node_size=800, node_color = ncols, node_shape = 's' )
        nx.draw_networkx_edges(G, graph_pos,arrowsize = 30)
        nx.draw_networkx_labels(G, graph_pos, font_size=12, font_family='sans-serif',
        labels = labs)
        plt.show()

    def test_cap (self):
        prob = p.LpProblem("Petri net Problem",p.LpMinimize)

        tr_items = [e.id for e in self.links]
        tr_vars = p.LpVariable.dicts("Iv",tr_items,lowBound=0,cat='Continuous')
        print (tr_vars)
        
        costs = dict()
        for i in tr_items:
            costs[i] = 0
        for pp in self.places:
            if pp.border == 0:
                for ll in pp.inputs:
                    costs[ll] = costs[ll] + 1
                for ll in pp.outputs:
                    costs[ll] = costs[ll] - 1

        prob += p.lpSum([costs[i]*tr_vars[i] for i in tr_items])
        #print ("costs")
        #print (costs)
        
        #print ("weights")
        for pl in self.places:
            wgts = dict()
            cnts = 0
            for i in tr_items:
                wgts[i] = 0
            #print (wgts)
            #print (cnts)
            if pl.border == 0:
                for e in pl.inputs:
                    wgts[e] = wgts[e] + 1
                for e in pl.outputs:
                    wgts[e] = wgts[e] - 1
                cnts = 0
                prob += p.lpSum([wgts[i]*tr_vars[i] for i in tr_items]) >= cnts
            if pl.border == 1:
                for e in pl.outputs:
                    wgts[e] = wgts[e] + 1
                cnts = pl.Q
                prob += p.lpSum([wgts[i]*tr_vars[i] for i in tr_items]) == cnts
            if pl.border == 2:
                for e in pl.inputs:
                    wgts[e] = wgts[e] + 1
                cnts = -pl.Q
                prob += p.lpSum([wgts[i]*tr_vars[i] for i in tr_items]) <= cnts

        for tr in self.transitions:
            wgts = dict()
            cnts = 0
            for i in tr_items:
                wgts[i] = 0
            for e in tr.inputs:
                wgts[e] = wgts[e] + 1
            for e in tr.outputs:
                wgts[e] = wgts[e] - 1
            prob += p.lpSum([wgts[i]*tr_vars[i] for i in tr_items]) == 0
            
            for i in tr_items:
                wgts[i] = 0
            for e in tr.inputs:
                wgts[e] = wgts[e] + 1
            cnts = tr.C
            prob += p.lpSum([wgts[i]*tr_vars[i] for i in tr_items]) <= cnts

            if tr.mode == 'A':
                for e in range(1,len(tr.inputs)):
                    e1 = tr.inputs[0]
                    e2 = tr.inputs[e]
                    for i in tr_items:
                        wgts[i] = 0
                    wgts[e1] =  1
                    wgts[e2] =  -1
                    prob += p.lpSum([wgts[i]*tr_vars[i] for i in tr_items]) == 0
                for e in range(1,len(tr.outputs)):
                    e1 = tr.outputs[0]
                    e2 = tr.outputs[e]
                    for i in tr_items:
                        wgts[i] = 0
                    wgts[e1] =  1
                    wgts[e2] =  -1
                    prob += p.lpSum([wgts[i]*tr_vars[i] for i in tr_items]) == 0
            if tr.mode == 'O':
                for e in range(1,len(tr.outputs)):
                    e1 = tr.outputs[0]
                    e2 = tr.outputs[e]
                    for i in tr_items:
                        wgts[i] = 0
                    wgts[e1] =  1
                    wgts[e2] =  -1
                    prob += p.lpSum([wgts[i]*tr_vars[i] for i in tr_items]) == 0
                    
            
        print ("----------------")
        print (prob)
        prob.solve()
        print ("===============")
        print (p.LpStatus[prob.status])
        QV = 0
        for var in tr_vars:
            QV = QV + tr_vars[var].varValue*costs[var]
            print (tr_vars[var],":",tr_vars[var].varValue)
        print ("Cost=", QV)
\end{verbatim}
\section{Mintafeladat}
Vegyünk egy 4 helyből és 2 tranzícióból álló rendszert. A helyekből egy nyelő, egy forrás és kettő belső hely. A rendszerben 6 él van az ábrán megadott módon.  A gráfban a sárga elem a helyeket, zöld a tranzíciókat jelöli. A csomópont elemben az első jel a hely kódja, a második a kapcsolódó kapacitás érték. A modellben a kisebb indexű tranzíció AND tulajdonságú, a másik OR tulajdonságú. 

A rendszerben 6 (nem negatív) változó jelenik meg:  $\{0: Iv_0, 1: Iv_1, 2: Iv_2, 3: Iv_3, 4: Iv_4, 5: Iv_5\}$

Az élek indexelése:
\begin{align*}
0  :  1 \rightarrow 5  \\
1  :  5 \rightarrow 2  \\
2  :  2 \rightarrow 6  \\
3  :  6 \rightarrow 3  \\
4  :  3 \rightarrow 5  \\
5  :  6 \rightarrow 4 
\end{align*}

A rendszerhez az alábbi egyenlőtlenségek kapcsolódnak:

\begin{center}
\begin{tabular}{rll}
$C_1$ &: $Iv_0$ &$= 20$ \\
$C_2$ &: $Iv_1 - Iv_2$ &$\geq 0$\\
$C_3$ &: $Iv_3 - Iv_4$ &$\geq 0$\\
$C_4$ &: $Iv_5$ &$\leq 25$\\
$C_5$ &: $Iv_0 - Iv_1 + Iv_4$ &$= 0$\\
$C_6$ &: $Iv_0 + Iv_4 $&$\leq 50$\\
$C_7$ &: $Iv_0 - Iv_4 $&$= 0$\\
$C_8$ &: $Iv_2 - Iv_3 - Iv_5$&$= 0$\\
$C_9$ &: $Iv_2 $&$\leq 50$\\
$C_{10}$ &: $Iv_3 - Iv_5 $&$= 0$
\end{tabular}
\end{center}

A kapcsolódó célfüggvény:
$$1\cdot Iv_1 + -1\cdot Iv_2 + 1\cdot Iv_3 + -1\cdot Iv_4\Rightarrow \min$$

Az LP feladat megoldható és a kapott megoldás:
\begin{center}
\begin{tabular}{rll}
&$Iv_0$ &: $20.0$\\
&$Iv_1$ &: $40.0$\\
&$Iv_2$ &: $40.0$\\
&$Iv_3$ &: $20.0$\\
&$Iv_4$ &: $20.0$\\
&$Iv_5$ &: $20.0$\\
&$Cost$ & $= 0.0$
\end{tabular}
\end{center}
Tehát a mintarendszerben nincs szükség belső pufferre. 
Ha lecsökkentjük a második tranzíció folyamerősségét, az akkor nem kapunk érvényes megoldást. \\
Infeasible
\begin{center}
\begin{tabular}{rll}
&$Iv_0$ &: $10.0$\\
&$Iv_1$ &: $20.0$\\
&$Iv_2$ &: $20.0$\\
&$Iv_3$ &: $10.0$\\
&$Iv_4$ &: $10.0$\\
&$Iv_5$ &: $10.0$
\end{tabular}
\end{center}
\chapter{Összegzés}

%TODO Leírni a dolgozatban elért eredményeket, és a további terveket!

\chapter{Hivatkozások}

%TODO A felhasznált hivatkozásokat már az elején célszerű összegyűjteni!

\end{document}
