\documentclass[12pt,a4paper]{book}
\usepackage[utf8]{inputenc}
\usepackage[magyar]{babel}
\usepackage[T1]{fontenc}
\usepackage{amsmath}
\usepackage{amsfonts}
\usepackage{amssymb}
\usepackage{listings} %TODO configure for c#
\usepackage[left=2cm,right=2cm,top=2cm,bottom=2cm]{geometry}
\date{\vspace{-5ex}}

%TODO a petri háló írása legyen konzisztens a C# ahol lehet zenei # el legyen hivatkozva az az "official"
%TODO a tranzició rövid vagy hosszú i?


\title{BPEL folyamatok Petri-háló alapú reprezentációja és szimulációja}

\begin{document}

%\include{cimlap}
\newpage
\tableofcontents
%TODO Add motivation, structure, results


\chapter{Bevezetés}
Ugyan a BPEL nyelv létrejötte elsődlegesen a Web szolgáltatások területéhez kapcsolódik, a nyelv mint általános workflow leíró nyelv, más alkalmazási témakörhöz is kapcsolható. A BPEL szerepét fontosságát, jól mutatja az a tény is,, hogy  amint a bevezetésben is láthattuk, igen gazdag irodalom található az egyes alkalmazási területekről és speciális szabvány kiegészítésekről.  A  BPEL aktualitását jelzi, hogy a megvalósító motorok köre is folyamatosan bővül. Ugyan már lassan 15 év eltelt a szabvány BPEL bevezetése óta, a meglévő nagyobb rendszerek (Oracle BPEL Process Manager, IBM WebSphere Process Server, Microsoft BizTalk Server, SAP SAP Exchange Infrastructure ) alternatívájaként  most is jelennek meg új végrehajtó motor implementációk. A Wikipédia forrása szerint \cite{wikiBpelList} a közelmúltban az alábbi szabad szotver implementációk születtek:
\begin{center}
\begin{tabular}{|c|c|c|c|}
\hline
\textbf{Termék neve} & \textbf{Fejlesztő} & \textbf{Megjelenés éve} & \textbf{Licensz}\\
JBPM & JBoss & 2016 & Apache\\
\hline
Apache ODE & ASF & 2016 & Apache\\
\hline
Activiti & Alfresco & 2014 & Apache\\
\hline
\end{tabular}
\end{center}

Ugyan napjainkra már több BPEL motor elérhető és használatos, ennek ellenére a BPEL szerkesztők és különösen a BPEL validációs rendszerek köre igen szegényes. Ezen tapasztalatokból kiindulva a dolgozat célja egy olyan BPEL validációs rendszer elkészítése, amely a BPEL rendszerek egyik fontos tulajdonságát, a terhelés korlátosságát (bounded model)   vizsgálja. A korlátosság azt jelzi, hogy minden csomópontban véges számú terhelés, feladat intenzitás jelentkezik bármely időpontban. Ha a rendszer nem teljesíti ezt a kritériumot, akkor túlcsordul valamely megmunkáló/tároló helyen a rendszer. Az elemzés során a korlátosság ténye mellett, a korlát értékei is fontos vizsgálandó jellemző. 

A meglévő tervezői rendszerekben legtöbbször szimulációval történik a főbb paraméterek, a korlátosság vizsgálata. Ezen megközelítésnek rendszerint két problémája van: az vizsgálat teljessége (azaz valóban minden lehetséges esetet áttekintettük-e) illetve a végrehajtási idő (a szimulációk futtatása hosszabb időt is igénybe vehet).  A dolgozatban a BPEL folyamatok Petri-háló alapú vizsgálatát végzem el. A Petri-háló alapú reprezentáció egy elfogadott és többek által alkalmazott megközelítés. A kidolgozott rendszer inputként egy  BPEL modell leírását várja és kimenetként az elemzés eredményét illetve a folyamatok nyomkövetését  adja vissza. A dolgozat főbb eredményei:
\begin{itemize}
\item BPEL folyamatok Petri háló formalizmusra történő konverziója
\item LP alapú végesség vizsgálat a Petri hálón
\item folyamatok grafikus nyomon követése, szimuláció 
\end{itemize}


%TODO Add bpel control elements, 10 example from core std /w pics, annot, and bpel desgn pictograms. Non core elements only w/ annotation

\chapter{A BPEL és folyamatainak bemutatása} 

A BPEL (Buisness Process Execution Language)üzleti folyamatok végrehajtó nyelve \cite{saraswathi2013oracle}.
Az OASIS által kezelt XML alapú szabványt használ. 
A dokumentum felépítésében egy XML dokumentum, mely a WS-BPEL szabvány szerint validált. A BPEL programok szerkezetét célszerű egy mintával áttekinteni, a jobb megértéshez. %TODO link: http://docs.oasis-open.org/wsbpel/2.0/OS/wsbpel-v2.0-OS.html chapter 5 section 1
Vegyük a következő példát.  \textsl{Adott egy online rendelést felvevő cég. A cég egy automata segítségével generál számlákat A számla az ár kiszámítása, a futár kiválasztása, és a szükséges termelés ütemezése után kerül kiállításra.} A lépéseket az alábbi ábra mutatja be: %TODO insert fig1.png 
Az ábrán a téglalapok egy rész processzt jelentenek. Az egy blokkban különállóak pedig konkurens proceszeket. A szaggatott vonal szekvenciát jelöl, míg a teli/sima pedig vezérlő linkek, a konkurens processzek szinkronizációját, várakoztatását lehet velük megoldani. Az ábra nem képez átíratot, csak mint egy standard érthető vizualizáció segíti a megértést. 

A program következő része egy WSDL szabvány ami a portot adja meg a processz számára. %TODO link to appendix
\newpage


 
 A kódban szereplő \texttt{<partnerLinks>} tartalmaz mindent (így közvetetten mindenkit) amik kapcsolatba kerül a processzel. Az elnevezés tükrözi a résztvevő partit, valamint a résztvevő feladatát, szándékát. A \texttt{<variables>} a változókat tartalmazza, míg a \texttt{<faultHandlers>} a hibakezelőket. A hibakezelés egy try-catch-finally "hibakezelőblokk" helyett az XML mentalitását tükröző módon kerül lekezelésre, a handlerek által. A kód többi része a processz standard definíciójához tartozik. A példák alapján elmondható, hogy a program a következő struktúra szerint épül fel.
\begin{itemize}
\item \textbf{Definíció: } a processz neve, névtere és különféle sémahívások, majd bővítmények, importok
\item \textbf{PartnerLinkek: } A megfelelő partnerek hozzáadása attribútumokkal. 
\item \textbf{Változók}
\item \textbf{Hibakezelők}
\item\textbf{Eseménykezelők}
\end{itemize}

\chapter{Petri-hálók és alkalmazásaik}
\section{Az alap Petri-hálók}
A Petri-háló egy matematikai leírómodell elosztott rendszerek bemutatására.
A modellt Carl Adam Petri készítette.
A modell nagyon hasonlít a programozók körében elterjedt folyamat ábrára.
A háló irányított élekből, helyekből és átmenetekből (\textsl{mint elemek}) áll.
Az élek csak két különböző típusú elem között állhatnak.
A helyeken pontok, ún. tokenek állhatnak.
A tokenek csak diszkrét számban fordulhatnak elő egy helyen, és a token átvitele atomi folyamat, azaz nem félbeszakítható.
A tokenek elláthatóak attribútummal is ilyen esetben a tokeneket "kiszínezzük" és színezett petri hálóról beszélünk. (ld. 2.2.) %TODO (LINK!)

%TODO cite: https://www.abhishekhalder.org/PetriNetReport.pdf__vq9vUXcOjS%2BawepkKcMLeAA64c19da3fb4d67754c3fe7eba8ce1187
Az alap Petri háló egy biparit, irányított és súlyozott multigráf $PN(P,T,A,W,S)$, ahol 
\begin{itemize}
\item $P=\{ p_1,p_2,\ldots ,p_N \}:$ a helyek véges halmaza ,
\item $T=\{ t_1,t_2,\ldots ,t_M\}:$ egy véges tranzició halmaz,
\item $P\cap T = \emptyset$
\item $A \subseteq P\times T \cup T\times P:$ az élek halmaza,
\item $W: F\Rightarrow N^+:$ az élsúlyok halmaza
\item $S: P\Rightarrow N^+:$ a kezdőállapot.
\end{itemize}

\section{Színezett Petri-hálók}

Az elemi színezett háló felírható, mint egy oly struktúra, ami: $CPN(P,T,A,\Sigma ,V,C,G,E,S)$, ahol 
\begin{itemize}
\item $P=\{ p_1,p_2,\ldots ,p_N \}:$ a helyek véges halmaza ,
\item $T=\{ t_1,t_2,\ldots ,t_M\}:$ egy véges tranzició halmaz,
\item $A \subseteq P\times T \cup T\times P:$ az élek halmaza,
\item $\Sigma:$ a színek halmazainak halmaza, 
\item $V:$ a változók halmaza, ahol $\forall v\in V:$ változóhoz egy $Type[v] \in \Sigma $ típus rendelhető,
\item $C: P\rightarrow \Sigma :$ a helyekhez színeket rendelő függvény,
\item $G: T\rightarrow EXPR_V:$ az egyes tranzíciókhoz kapcsolódó validációs, ellenőrzési kifejezés (logikai értékű)
\item $E: A\rightarrow EXPR_V:$ z  egyes élekhez kapcsolódó kifejezés, amely a kapcsolódó hely színhalmazához tartozó értéket vehet fel
\item $S: P\Rightarrow N^+:$ a kezdőállapot.
\end{itemize}

Adott $CPN(P,T,A,\Sigma ,V,C,G,E,S)$ színezett hálóhoz az alábbi kezelő funkciók köthetőek: 
\begin{itemize}
\item $M(p):$ a jelölő (marker) függvény, melynek értéke a $p$ helyhez kapcsolódó tokenek halmaza. Színezett Petri háló esetén az $M(p)$ elemek színeinek illeszkedni kell a $C(p)$ színhalma
\item $M_0(p):$  helyek induló tokenkészlete
\item $Var(t):$ a tranzíciók viselkedését leíró változók halmaza
\item $b(v):$ a adott v változó értékét megadó kifejezés, ahol $b(v) \in Type[v]$
\end{itemize}

Egy adott $t$ tranzíció esetén a $Var(t)$ kifejezés a tranzícióhoz rendelt változók együttese, ahol a változók a $G(t)$ vagy $E$(a: t-hez kötődő él) kifejezésekben szerepelnek.
\begin{equation*}
Var(t)=\begin{cases}
\{n,d\} &\text{if } t=SendPacket\\
\{n,d,success\} &\text{if } t= TransmitPacket\\
\{n,d,k,data\} &\text{if } t=ReceivePacket\\
\{n,success\} &\text{if } t=TrancmitAck\\
\{n,k\} &\text{if }t=ReceiveAck
\end{cases}
\end{equation*}

A hálóban egy tranzíció akkor engedélyezett (ready), ha minden bemenő helyeknél a kívánt tokenszám megtalálható.   Jelölt hálók esetében:
$$M'(p)=M(p)-I(p,t)+O(p,t): \forall p\in P,$$ ahol 
\begin{itemize}
\item $I:F\Rightarrow N^+:$  bejövő áram intenzitás
\item $O:F\Rightarrow N^+:$ kimenő áram intenzitás

\end{itemize}
A hierarchikus CPN rendszerben az átláthatóság növelése érdekében összefogó modulokat is lehet alkalmazni. Egy modul más elemi egységek együttese, konténere. %TODO Insert "rendszerséma 3 modullal.png" & 2A receiver belső szerkezete.png"

A moduloknál fontos szerepet kapnak az átadó helyek, melyeken keresztül a tokenek bejöhetnek a modulba illetve kiléphetnek a modulból. Az ilyen port jellegű helyek lehetnek bemeneti portok (IN) illetve kimeneti portok (OUT).  

A CPN rendszerek egyik hasznos tulajdonsága, hogy lehetőséget adnak a felépített modell formális ellenőrzésére, validálására és értékelésére. A formális ellenőrzés egyik leggyakoribb eszköze az állapottér (state space)  modell, ahol az állapottér egy olyan irányított gráf, melyben a csomópontok a háló egy lehetséges  M(CPN) jelölési állapota. Azaz a háló struktúrája rögzített, de az egyes elemeknél a tokenek és változók halmaza, azok állapota változhat. A véges állapottér modellt rendszerint szimulációkal állítják elő. 

Az állapottér modellből kiindulva további elemzésekre ad lehetőséget a komponens gráf modell (SCC graph:  strongly-connected-component graph) formalizmus. Az  SCC gráfból a rendszer általános viselkedési szabályaira lehet következtetni. Az SCC gráf olyan gráf, melynek csomópontjai  az állapottér azon diszjunkt részhalmazai, ahol egy részhalmaz bármely két elemére igaz, hogy az egyik elem  elérhető a másikból. 

Az elemzések során az alábbi főbb tulajdonságok elemzésére szoktak kitérni:
\begin{itemize}
\item Reachability Properties
\item Boundedness Properties
\item Home Properties 
\item Liveness Properties
\item Fairness Properties
\end{itemize}
%TODO insert mintaállapottér 1-2-3.png here


\include{bproc}
\chapter{Szimulációs és validációs keretrendszer}
\section{Elvárások az alkalmazással szemben}

%TODO Itt kellene röviden áttekinteni az alkalmazással szemben támasztott követelményeket.
Az alkalmazás legfőbb feladata egy konverzió, BPEL és Petri-háló között. Ebből adódóan minden BPEL elemet le kell tudnia kezelni, illetve az azok közti összefüggéseket feltérképezni, és az összefüggés halmazból egy petri hálót előállítani. A hálót tudni kell megjeleníteni, valamint a hálón belüli mozgásokat rajzolni, és szükségesség esetén input file-t vagy felhasználói inputot kezelni. Ha a háló nem hozható létre, akkor azt tudatnia kell,és lehetőség szerint rövid indoklással alátámasztania. 

\section{Az alkalmazás felépítése}

%TODO Osztály és blokkdiagramok formájában be kellene mutatni, hogy milyen fő elemekből épül fel az alkalmazás.
Az alkalmazás logikailag a következő fő részekből áll:
\begin{itemize}
\item I/O module: Beolvassa az XML dokumentumot és felparseolja. Szükség esetén menti a kész hálót.
\item Conversion module: Átkonvertálja  a beolvasott dokumentumot.
\item Data Structure module: A saját típusú Petri elemeket kezeli, és adatszerkezeti implementációt tartalmaz. 
\item UI talker: A UI ra illeszti a megfelelő input mezőt, az abba felvitt értéket átadja a feldolgozó egységnek.
\item Graphics module: Az MSGL libraryre épül. Feladata a gráf rajzolása és megjelenítése animációval együtt. 
\item Computing module: A háló animációjához végzi a szükséges számításokat és időzítéseket. 
\end{itemize}


\section{Első példa üzleti folyamat}

%TODO A címet majd nyilván át kell írni. Itt szerepelne egy bonyolultabb üzleti folyamat, és a konverzió eredménye.

\section{Második példa üzleti folyamat}

%TODO Hasonló az előző szakaszhoz, csak másik példával.

\include{analytics}
\chapter{Összegzés}
Összegzés

A BPEL szabvány az üzleti folyamatok szabványos leírására szolgál.  A dolgozat témaköre a BPEL nyelven létrehozott üzleti folyamatok modellezése  és elemzése a Petri-háló alapú formalizmus segítségével. A kidolgozott mintarendszer inputként egy BPEL modell leírását várja és kimenetként az elemzés eredményét illetve a folyamatok nyomkövetését adja vissza.

A TDK munka keretében az alábbi eredményeket értem el:
\begin{itemize}
\item BPEL folyamatok Petri háló formalizmusra történő konverziója,
\item LP alapú végesség vizsgálat a Petri hálón,
\item folyamatok grakus nyomon követése, szimuláció.
\end{itemize}

A projekt következő lépéseként a modell logisztikai folyamatokra történő  adaptálását végezzük el. A továbbfejlesztett szimulációs motorba több új elosztás alapú generációs modult is el fogok készíteni. Az eredményeket egy cikk publikációjában kívánjuk megjeleníteni.

%TODO A felhasznált hivatkozásokat már az elején célszerű összegyűjteni!
\bibliographystyle{acm}
\bibliography{tdk}

\end{document}
