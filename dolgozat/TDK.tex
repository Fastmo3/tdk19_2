\documentclass[12pt,a4paper]{book}
\usepackage[utf8]{inputenc}
\usepackage[magyar]{babel}
\usepackage[T1]{fontenc}
\usepackage{amsmath}
\usepackage{amsfonts}
\usepackage{amssymb}

%\usepackage{URL}

\usepackage{multirow}
\usepackage{listings} %TODO configure for c#
\usepackage{cpp}
\usepackage{python}

\usepackage[left=2cm,right=2cm,top=2cm,bottom=2cm]{geometry}
\date{\vspace{-5ex}}

\usepackage{hyperref}

\usepackage{graphics}

%TODO a petri háló írása legyen konzisztens a C# ahol lehet zenei # el legyen hivatkozva az az "official"
%TODO a tranzició rövid vagy hosszú i?


\title{BPEL üzleti folyamatmodellek korlátosság vizsgálata Petri-háló reprezentációval}

\begin{document}
%TODO re-include cover with images
%\begin{center}
\begin{tabular*}{\hsize}{@{}c@{\extracolsep{\fill}}c@{\extracolsep{\fill}}c@{}}
\multirow{2}[4]*{\epsfig{file=images/ME_logo.eps,height=2truecm}}& \textcolor{blue}{\large\bfseries MISKOLCI EGYETEM}&\multirow{2}[4]*{\epsfig{file=images/gepesz_logo.eps,height=1.6truecm}}\\
&\textcolor{blue}{\large\bfseries GÉPÉSZMÉRNÖKI ÉS INFORMATIKAI KAR}&\\
\end{tabular*}
\end{center}

\vglue 2.5truecm %függleges helykihagyás

\pagestyle{empty}

%A szakdolgozat címe, akár több sorban is
{\LARGE
\begin{center}
\textcolor{blue}{\Large\bfseries TDK DOLGOZAT}
\end{center}}

\vspace*{2.5truecm}
\begin{center}
% (Tag Based Document Management)
\LARGE\bfseries BPEL üzleti folyamatmodellek korlátosság vizsgálata Petri-háló reprezentációval
\end{center}
\vglue 2cm

{\large
\begin{center}
\begin{tabular}{c}
{\bfseries Hornyák Bence}\\
Programtervező informatikus BSc
\end{tabular}
\end{center}
\vglue 3cm
\begin{center}
\textbf{Konzulens:}
\end{center}
\medskip
\begin{center}
\begin{tabular}{ccc}
\textbf{Dr. Kovács László} \\
tanszékvezető, egyetemi docens \\
Általános Informatikai Tanszék \\
\end{tabular}
\end{center}
\vfill
{\large

\begin{center}
\textbf{\textsc{Miskolc, 2019}}
\end{center}}

\newpage
\newpage
\tableofcontents

%TODO Add motivation, structure, results


\chapter{Bevezetés}
Ugyan a BPEL nyelv létrejötte elsődlegesen a Web szolgáltatások területéhez kapcsolódik, a nyelv mint általános workflow leíró nyelv, más alkalmazási témakörhöz is kapcsolható. A BPEL szerepét fontosságát, jól mutatja az a tény is,, hogy  amint a bevezetésben is láthattuk, igen gazdag irodalom található az egyes alkalmazási területekről és speciális szabvány kiegészítésekről.  A  BPEL aktualitását jelzi, hogy a megvalósító motorok köre is folyamatosan bővül. Ugyan már lassan 15 év eltelt a szabvány BPEL bevezetése óta, a meglévő nagyobb rendszerek (Oracle BPEL Process Manager, IBM WebSphere Process Server, Microsoft BizTalk Server, SAP SAP Exchange Infrastructure ) alternatívájaként  most is jelennek meg új végrehajtó motor implementációk. A Wikipédia forrása szerint \cite{wikiBpelList} a közelmúltban az alábbi szabad szotver implementációk születtek:
\begin{center}
\begin{tabular}{|c|c|c|c|}
\hline
\textbf{Termék neve} & \textbf{Fejlesztő} & \textbf{Megjelenés éve} & \textbf{Licensz}\\
JBPM & JBoss & 2016 & Apache\\
\hline
Apache ODE & ASF & 2016 & Apache\\
\hline
Activiti & Alfresco & 2014 & Apache\\
\hline
\end{tabular}
\end{center}

Ugyan napjainkra már több BPEL motor elérhető és használatos, ennek ellenére a BPEL szerkesztők és különösen a BPEL validációs rendszerek köre igen szegényes. Ezen tapasztalatokból kiindulva a dolgozat célja egy olyan BPEL validációs rendszer elkészítése, amely a BPEL rendszerek egyik fontos tulajdonságát, a terhelés korlátosságát (bounded model)   vizsgálja. A korlátosság azt jelzi, hogy minden csomópontban véges számú terhelés, feladat intenzitás jelentkezik bármely időpontban. Ha a rendszer nem teljesíti ezt a kritériumot, akkor túlcsordul valamely megmunkáló/tároló helyen a rendszer. Az elemzés során a korlátosság ténye mellett, a korlát értékei is fontos vizsgálandó jellemző. 

A meglévő tervezői rendszerekben legtöbbször szimulációval történik a főbb paraméterek, a korlátosság vizsgálata. Ezen megközelítésnek rendszerint két problémája van: az vizsgálat teljessége (azaz valóban minden lehetséges esetet áttekintettük-e) illetve a végrehajtási idő (a szimulációk futtatása hosszabb időt is igénybe vehet).  A dolgozatban a BPEL folyamatok Petri-háló alapú vizsgálatát végzem el. A Petri-háló alapú reprezentáció egy elfogadott és többek által alkalmazott megközelítés. A kidolgozott rendszer inputként egy  BPEL modell leírását várja és kimenetként az elemzés eredményét illetve a folyamatok nyomkövetését  adja vissza. A dolgozat főbb eredményei:
\begin{itemize}
\item BPEL folyamatok Petri háló formalizmusra történő konverziója
\item LP alapú végesség vizsgálat a Petri hálón
\item folyamatok grafikus nyomon követése, szimuláció 
\end{itemize}

 %ch1-Introduction
%TODO Add bpel control elements, 10 example from core std /w pics, annot, and bpel desgn pictograms. Non core elements only w/ annotation

\chapter{A BPEL és folyamatainak bemutatása} 

A BPEL (Buisness Process Execution Language)üzleti folyamatok végrehajtó nyelve \cite{saraswathi2013oracle}.
Az OASIS által kezelt XML alapú szabványt használ. 
A dokumentum felépítésében egy XML dokumentum, mely a WS-BPEL szabvány szerint validált. A BPEL programok szerkezetét célszerű egy mintával áttekinteni, a jobb megértéshez. %TODO link: http://docs.oasis-open.org/wsbpel/2.0/OS/wsbpel-v2.0-OS.html chapter 5 section 1
Vegyük a következő példát.  \textsl{Adott egy online rendelést felvevő cég. A cég egy automata segítségével generál számlákat A számla az ár kiszámítása, a futár kiválasztása, és a szükséges termelés ütemezése után kerül kiállításra.} A lépéseket az alábbi ábra mutatja be: %TODO insert fig1.png 
Az ábrán a téglalapok egy rész processzt jelentenek. Az egy blokkban különállóak pedig konkurens proceszeket. A szaggatott vonal szekvenciát jelöl, míg a teli/sima pedig vezérlő linkek, a konkurens processzek szinkronizációját, várakoztatását lehet velük megoldani. Az ábra nem képez átíratot, csak mint egy standard érthető vizualizáció segíti a megértést. 

A program következő része egy WSDL szabvány ami a portot adja meg a processz számára. %TODO link to appendix
\newpage


 
 A kódban szereplő \texttt{<partnerLinks>} tartalmaz mindent (így közvetetten mindenkit) amik kapcsolatba kerül a processzel. Az elnevezés tükrözi a résztvevő partit, valamint a résztvevő feladatát, szándékát. A \texttt{<variables>} a változókat tartalmazza, míg a \texttt{<faultHandlers>} a hibakezelőket. A hibakezelés egy try-catch-finally "hibakezelőblokk" helyett az XML mentalitását tükröző módon kerül lekezelésre, a handlerek által. A kód többi része a processz standard definíciójához tartozik. A példák alapján elmondható, hogy a program a következő struktúra szerint épül fel.
\begin{itemize}
\item \textbf{Definíció: } a processz neve, névtere és különféle sémahívások, majd bővítmények, importok
\item \textbf{PartnerLinkek: } A megfelelő partnerek hozzáadása attribútumokkal. 
\item \textbf{Változók}
\item \textbf{Hibakezelők}
\item\textbf{Eseménykezelők}
\end{itemize}
%ch2-BPEL-introduction, 
\chapter{Petri-hálók és alkalmazásaik}
\section{Az alap Petri-hálók}
A Petri-háló egy matematikai leírómodell elosztott rendszerek bemutatására.
A modellt Carl Adam Petri készítette.
A modell nagyon hasonlít a programozók körében elterjedt folyamat ábrára.
A háló irányított élekből, helyekből és átmenetekből (\textsl{mint elemek}) áll.
Az élek csak két különböző típusú elem között állhatnak.
A helyeken pontok, ún. tokenek állhatnak.
A tokenek csak diszkrét számban fordulhatnak elő egy helyen, és a token átvitele atomi folyamat, azaz nem félbeszakítható.
A tokenek elláthatóak attribútummal is ilyen esetben a tokeneket "kiszínezzük" és színezett petri hálóról beszélünk. (ld. 2.2.) %TODO (LINK!)

%TODO cite: https://www.abhishekhalder.org/PetriNetReport.pdf__vq9vUXcOjS%2BawepkKcMLeAA64c19da3fb4d67754c3fe7eba8ce1187
Az alap Petri háló egy biparit, irányított és súlyozott multigráf $PN(P,T,A,W,S)$, ahol 
\begin{itemize}
\item $P=\{ p_1,p_2,\ldots ,p_N \}:$ a helyek véges halmaza ,
\item $T=\{ t_1,t_2,\ldots ,t_M\}:$ egy véges tranzició halmaz,
\item $P\cap T = \emptyset$
\item $A \subseteq P\times T \cup T\times P:$ az élek halmaza,
\item $W: F\Rightarrow N^+:$ az élsúlyok halmaza
\item $S: P\Rightarrow N^+:$ a kezdőállapot.
\end{itemize}

\section{Színezett Petri-hálók}

Az elemi színezett háló felírható, mint egy oly struktúra, ami: $CPN(P,T,A,\Sigma ,V,C,G,E,S)$, ahol 
\begin{itemize}
\item $P=\{ p_1,p_2,\ldots ,p_N \}:$ a helyek véges halmaza ,
\item $T=\{ t_1,t_2,\ldots ,t_M\}:$ egy véges tranzició halmaz,
\item $A \subseteq P\times T \cup T\times P:$ az élek halmaza,
\item $\Sigma:$ a színek halmazainak halmaza, 
\item $V:$ a változók halmaza, ahol $\forall v\in V:$ változóhoz egy $Type[v] \in \Sigma $ típus rendelhető,
\item $C: P\rightarrow \Sigma :$ a helyekhez színeket rendelő függvény,
\item $G: T\rightarrow EXPR_V:$ az egyes tranzíciókhoz kapcsolódó validációs, ellenőrzési kifejezés (logikai értékű)
\item $E: A\rightarrow EXPR_V:$ z  egyes élekhez kapcsolódó kifejezés, amely a kapcsolódó hely színhalmazához tartozó értéket vehet fel
\item $S: P\Rightarrow N^+:$ a kezdőállapot.
\end{itemize}

Adott $CPN(P,T,A,\Sigma ,V,C,G,E,S)$ színezett hálóhoz az alábbi kezelő funkciók köthetőek: 
\begin{itemize}
\item $M(p):$ a jelölő (marker) függvény, melynek értéke a $p$ helyhez kapcsolódó tokenek halmaza. Színezett Petri háló esetén az $M(p)$ elemek színeinek illeszkedni kell a $C(p)$ színhalma
\item $M_0(p):$  helyek induló tokenkészlete
\item $Var(t):$ a tranzíciók viselkedését leíró változók halmaza
\item $b(v):$ a adott v változó értékét megadó kifejezés, ahol $b(v) \in Type[v]$
\end{itemize}

Egy adott $t$ tranzíció esetén a $Var(t)$ kifejezés a tranzícióhoz rendelt változók együttese, ahol a változók a $G(t)$ vagy $E$(a: t-hez kötődő él) kifejezésekben szerepelnek.
\begin{equation*}
Var(t)=\begin{cases}
\{n,d\} &\text{if } t=SendPacket\\
\{n,d,success\} &\text{if } t= TransmitPacket\\
\{n,d,k,data\} &\text{if } t=ReceivePacket\\
\{n,success\} &\text{if } t=TrancmitAck\\
\{n,k\} &\text{if }t=ReceiveAck
\end{cases}
\end{equation*}

A hálóban egy tranzíció akkor engedélyezett (ready), ha minden bemenő helyeknél a kívánt tokenszám megtalálható.   Jelölt hálók esetében:
$$M'(p)=M(p)-I(p,t)+O(p,t): \forall p\in P,$$ ahol 
\begin{itemize}
\item $I:F\Rightarrow N^+:$  bejövő áram intenzitás
\item $O:F\Rightarrow N^+:$ kimenő áram intenzitás

\end{itemize}
A hierarchikus CPN rendszerben az átláthatóság növelése érdekében összefogó modulokat is lehet alkalmazni. Egy modul más elemi egységek együttese, konténere. %TODO Insert "rendszerséma 3 modullal.png" & 2A receiver belső szerkezete.png"

A moduloknál fontos szerepet kapnak az átadó helyek, melyeken keresztül a tokenek bejöhetnek a modulba illetve kiléphetnek a modulból. Az ilyen port jellegű helyek lehetnek bemeneti portok (IN) illetve kimeneti portok (OUT).  

A CPN rendszerek egyik hasznos tulajdonsága, hogy lehetőséget adnak a felépített modell formális ellenőrzésére, validálására és értékelésére. A formális ellenőrzés egyik leggyakoribb eszköze az állapottér (state space)  modell, ahol az állapottér egy olyan irányított gráf, melyben a csomópontok a háló egy lehetséges  M(CPN) jelölési állapota. Azaz a háló struktúrája rögzített, de az egyes elemeknél a tokenek és változók halmaza, azok állapota változhat. A véges állapottér modellt rendszerint szimulációkal állítják elő. 
\newpage
Az állapottér modellből kiindulva további elemzésekre ad lehetőséget a komponens gráf modell (SCC graph:  strongly-connected-component graph) formalizmus. Az  SCC gráfból a rendszer általános viselkedési szabályaira lehet következtetni. Az SCC gráf olyan gráf, melynek csomópontjai  az állapottér azon diszjunkt részhalmazai, ahol egy részhalmaz bármely két elemére igaz, hogy az egyik elem  elérhető a másikból. 

Az elemzések során az alábbi főbb tulajdonságok elemzésére szoktak kitérni:
\begin{itemize}
\item Reachability Properties
\item Boundedness Properties
\item Home Properties 
\item Liveness Properties
\item Fairness Properties
\end{itemize}
\section{Saját modell,\textsl{(kiegészítés a színezett esethez)}}
A színezett Petri-hálók esetén több különböző típusú tokenek élnek a rendszerben. A kapacitás vizsgálatnál ekkor az egyes tranzícióknál eltérő lehet a kapacitás korlát (a maximális folyam erősség) a különböző típusú tokenek esetén. Emiatt külön kell vizsgálni az egyes típusok folyam erősségét, nem lehet összevonni őket. 
A színezett hálóban az élekhez $I^c_x$ token áramlás erősségeket definiálunk, ahol x jelöli az él indexét és c a színkód. A forrás helyekhez $Q^c_x$ forrás erősség indexeket adunk meg a különböző c színekre vonatkozólag. A hálóban az alábbi kapacitás korlátokat vezetjük be:
\begin{itemize}
\item $C^C_x$: az x. tranzíció maximális erőssége a c szín esetén 
\item $C^C_y$: az y. nyelő maximális folyam erőssége a c szín esetén
\end{itemize}
A hálóban az alábbi megkötések élnek a folyamerősségekre:
\begin{itemize}
\item forrás helyek (x) esetén: $\forall c$ színre: $\sum_{y\text{ kimenő élek}}I^C_Y = Q^C_x$
\item nyelő helyek (x) esetén: $\forall c$ színre: $\sum_{y\text{ bejövő élek}}I^C_y \leq C^C_x$
\item belső helyek esetén: $\forall c$ színre: $\sum_{y\text{ kimenő élek}} I^C_y \leq \sum_{y\text{ bejövő élek}}$
\item tranzíciók (x) esetén:
$$\forall c \text{ színre} \sum_{y\text{ kimenő élek}} I^C_y == \sum_{y\text{ bejövő élek}} I^C_y $$
$$\sum_{c \text{ színek}}\left( \frac{1}{C^C_x} \left( \sum_{y \text{ bejövő élek}} I^C_y \right) \right) \leq 1$$
$$\forall c \text{ színre:} \forall \text{ kimenő} (y,z) \text{élre: } I^C_y=I^C_z$$
\item az AND típusú tranzakciók esetén még ezen felül teljesül, hogy 	$\forall c$ színre: $\forall$ bejövő $(y,z)$ élre $I^C_y=I^C_z$
\end{itemize}
Az egyes belső helyeken a bufferbe áramló tokenek  eredő intenzitása:
$$F=\sum_{c\text{ színek}} \left( \sum_{x\text{ belső hely}}\left( \sum_{y\text{ bejövő élek }x\text{-nél}}I^C_y - \sum_{y \text{ kimenő helyek }x\text{-nél}} I^C_y \right) \right)$$
Az F függvény 0 értéke esetén nincs szükség belső bufferre. 
A fenti feladat egy LP programozási feladatnak is tekinthető, ahol a változók az élek $I_x$ nem negatív intenzitásai és a célfüggvény: $F \rightarrow \min$ alakú. 

\section{A validációs számítás algoritmusa}
A hálót leíró struktúra három alappilléren nyugszik:
\begin{enumerate}
\item helyek
\item tranziciók
\item élek
\end{enumerate}
A helyek esetén az alábbi attribútumokat tárolja a rendszer:
\begin{itemize}
\item id : azonosító kód
\item inputs : bejövő élek
\item outputs : kimenő élek
\item tokens : tárolt tokenek
\item Q : forrás intenzitás
\item border : pozíció jelző, belső vagy határ pozíció
\end{itemize}
        
A tranzíciók jellemzői:
\begin{itemize}
\item id : azonosító kód 
\item inputs : bejövő élek
\item outputs : kimenő élek
\item C : feldolgozási intenzitás
\item mode : működési mód (AND, OR) 
\end{itemize}

Az élek attribútumai:
\begin{itemize}
\item id : azonosító kód 
\item input : induló elem
\item output : cél elem
\item alfa : az él kapacitás jelzője
\item inner : él típusa, belső vagy határ
\end{itemize}

A kapacitás vizsgálatot végző rutin az alábbi tevékenységeket hajtja végre.
\begin{verbatim}
     	// új LP feladat inicializása
        prop = Init_LpProblem(Minimize)
	// változók inicializálása
        tr_vars = LpVariable("Iv",tr_items,lowBound=0,cat='Continuous')


// ciklus a helyekre
        for pp in places:
		// ha belső hely
	if pp.border == 0
		// együttható aktualizálása
		costs[ll] = costs[ll] +/- 1
	// a célfüggvény meghatározása
        Init_lpSum([costs[i]*tr_vars[i] for i in tr_items])

	// ciklus a helyekre
        for pp in places:
	// ha belső pont
if pl.border == 0:
                for e in pl.inputs:
                    wgts[e] = wgts[e] + 1
                for e in pl.outputs:
                    wgts[e] = wgts[e] - 1
                cnts = 0
// az egyenlőtlenség rendszer együtthatóinak meghatározása 
               Init_lpSum([wgts[i]*tr_vars[i] for i in tr_items]) >= cnts
		
// ha forrás pont:
            if pl.border == 1:
                for e in pl.outputs:
                    wgts[e] = wgts[e] + 1
                cnts = pl.Q
// az egyenlőtlenség rendszer együtthatóinak meghatározása 
                Init_lpSum([wgts[i]*tr_vars[i] for i in tr_items]) == cnts

// ha nyelő pont:
            if pl.border == 2:
                for e in pl.inputs:
                    wgts[e] = wgts[e] + 1
                cnts = -pl.Q
// az egyenlőtlenség rendszer együtthatóinak meghatározása 
                Init_.lpSum([wgts[i]*tr_vars[i] for i in tr_items]) <= cnts

	// ciklus a transition elemekre
        for tr in self.transitions:
            for e in tr.inputs:
                wgts[e] = wgts[e] + 1
            for e in tr.outputs:
                wgts[e] = wgts[e] - 1
// az egyenlőtlenség rendszer együtthatóinak meghatározása 
            Init_lpSum([wgts[i]*tr_vars[i] for i in tr_items]) == 0

            for e in tr.inputs:
                wgts[e] = wgts[e] + 1
            cnts = tr.C
// az egyenlőtlenség rendszer együtthatóinak meghatározása 
            Init_lpSum([wgts[i]*tr_vars[i] for i in tr_items]) <= cnts

		// AND működési modell, szinkronitás
            if tr.mode == 'AND':
                for e in range(1,len(tr.inputs)):
// az egyenlőtlenség rendszer együtthatóinak meghatározása 

                    Init_lpSum([wgts[i]*tr_vars[i] for i in tr_items]) == 0
                for e in range(1,len(tr.outputs)):
// az egyenlőtlenség rendszer együtthatóinak meghatározása 
                    Init_lpSum([wgts[i]*tr_vars[i] for i in tr_items]) == 0

		// OR működési modell, tetszőleges beérkezés
            if tr.mode == 'OR':
                for e in range(1,len(tr.outputs)):
                    e1 = tr.outputs[0]
                    e2 = tr.outputs[e]
                    wgts[e1] =  1
                    wgts[e2] =  -1
                    Init_p.lpSum([wgts[i]*tr_vars[i] for i in tr_items]) == 0
                    
        // LP feladat megoldása
        prob.solve()

	// eredmény kiíratása
	
        prob.print()
\end{verbatim}         

%ch3 Petri net definition+intorduction
\chapter{Az üzleti folyamatok elemeinek leképzése}

%TODO Ide kellene felsorolni, és részletesen leírni, hogy a BPEL egyes elemeinek milyen Petri-háló feleltethető meg.

%TODO Megnézni, hogy az egyes elemek esetében milyen alternatívák lennének a leképzésre!
\section{A leképzés menete}
A leképzéskor elsődlegesen az aktivitás elemei kerülnek a figyelem középpontjába, a séma és különböző definíciók csak másodlagos helyet kapnak. Ez azért lényeges, mert a séma nem egy dinamikus folyamatot ír le, hanem az adott folyamat és eleminek tulajdonságát. Mivel a Petri-háló egy folyamat személtetésére lett létrehozva ezért főleg az aktív elemek a szignifikánsak a konverzió során, tehát leképzés során az XML validációhoz szükséges statikus elemeket ignoráljuk.  Az alábbi elemeknek feleltetünk meg egy egy részhálót:
\section{\texttt{<receive>}}
\texttt{<receive>} A recieve egy megfelelő üzenet után engedi a folyamatot továbbhaladni, így várakoztatáshoz használható. A \texttt{<receive>} teljes terjedelmében:\\
\begin{verbatim}
<receive partnerLink="NCName"
   portType="QName"?
   operation="NCName"
   variable="BPELVariableName"?
   createInstance="yes|no"?
   messageExchange="NCName"?
   standard-attributes>
   standard-elements
   <correlations>?
      <correlation set="NCName" initiate="yes|join|no"? />+
   </correlations>
   <fromParts>?
      <fromPart part="NCName" toVariable="BPELVariableName" />+
   </fromParts>
</receive>
\end{verbatim}
A hálóban ezt egy tranzicióval könnyedén megoldhatjuk, hisz csak egy specifikus msg token kell a továbblépéshez, és a többit addig az előző helyen parkoltatja. 

\section{\texttt{<reply>}}
\texttt{<reply>} A reply egy üzenetküldő elem, ami \texttt{<receive>; <onMessage>;<onEvent>} események után léphet akcióba. 
\begin{verbatim}
<reply partnerLink="NCName"
   portType="QName"?
   operation="NCName"
   variable="BPELVariableName"?
   faultName="QName"?
   messageExchange="NCName"?
   standard-attributes>
   standard-elements
   <correlations>?
      <correlation set="NCName" initiate="yes|join|no"? />+
   </correlations>
   <toParts>?
      <toPart part="NCName" fromVariable="BPELVariableName" />+
   </toParts>
</reply>
\end{verbatim}
A lekezelése az előző példával analóg módon, annyi különbséggel, hogy a tokenek nem parkolnak, hanem tovább mennek és a tranzíció csak akkor generál új tokent ha üzenetet kap egy ágról.

\section{\texttt{<invoke>}}
\texttt{<invoke>} Egy Bpel vagy épp egy webszolgáltatás meghívására szolgál és definiálja a szolgáltatás feladatát is. 
\begin{verbatim}
<invoke partnerLink="NCName"
   portType="QName"?
   operation="NCName"
   inputVariable="BPELVariableName"?
   outputVariable="BPELVariableName"?
   standard-attributes>
   standard-elements
   <correlations>?
      <correlation set="NCName" initiate="yes|join|no"?
         pattern="request|response|request-response"? />+
   </correlations>
   <catch faultName="QName"?
      faultVariable="BPELVariableName"?
      faultMessageType="QName"?
      faultElement="QName"?>*
      activity
   </catch>
   <catchAll>?
      activity
   </catchAll>
   <compensationHandler>?
      activity
   </compensationHandler>
   <toParts>?
      <toPart part="NCName" fromVariable="BPELVariableName" />+
   </toParts>
   <fromParts>?
      <fromPart part="NCName" toVariable="BPELVariableName" />+
   </fromParts>
</invoke>
\end{verbatim}
%TODO ebben nem vbagyok biztos hogy hogy lehet. Átad egy start tokent egy részhálónak??

\section{\texttt{<assign>}}
\texttt{<assign>} Egy változó értékadására szolgáló esemény. Ellentétben egy imperatív értékadással egy assign blokkban bármennyi értékadás, másolás történhet, amíg azt a kliens kezelni tudja, így logikailag egy egységbe zárja a műveleteket.  
\begin{verbatim}
<assign validate="yes|no"? standard-attributes>
   standard-elements
   (
   <copy keepSrcElementName="yes|no"? ignoreMissingFromData="yes|no"?>
      from-spec
      to-spec
   </copy>
   |
   <extensionAssignOperation>
      assign-element-of-other-namespace
   </extensionAssignOperation>
   )+
</assign>
\end{verbatim}
Az assign nagyon egszerűen átírható egy tranzicióra ami a megfelelő tokenek színét módosítja. 

\section{\texttt{<validate>}}
\texttt{<validate>} Egy sémára validálja az XML (BPEL) file-t. 
\begin{verbatim}
<validate variables="BPELVariableNames" standard-attributes>
   standard-elements
</validate>
\end{verbatim}
\texttt{<throw>}Egy rész processzen belül fault generálására szolgál. 
\begin{verbatim}
<throw faultName="QName"
   faultVariable="BPELVariableName"?
   standard-attributes>
   standard-elements
</throw>
\end{verbatim}
Nagyon egyszerűen egy fault tokent generáló tranzició komponens. Explicit hálórésze nincs, hanem a megfelelő inputtokenek megléte vagy hiánya generálja egy tranzició során. 

\section{\texttt{<wait>}}
\texttt{<wait>} Időre vonatkoztatva várakoztat. Például 5000 tick vagy 14:00:23 (hh:mm:ss)
\begin{verbatim} 
<wait standard-attributes>
   standard-elements
   (
   <for expressionLanguage="anyURI"?>duration-expr</for>
   |
   <until expressionLanguage="anyURI"?>deadline-expr</until>
   )
</wait>
\end{verbatim}
Megadható egy részhálóval ami valójában egy oszcillátor és a megfelelő iteráció után folytat tokent küld. 

\section{\texttt{<empty>}}
\texttt{<empty>} No-op (no operations) event szinkronizációra szolgál.
\begin{verbatim}
<empty standard-attributes>
   standard-elements
</empty>
\end{verbatim}
Beiktatható egy semleges tranzició és hely.

\section{\texttt{<sequence>} }
\texttt{<sequence>} Sorozatot ad meg.
\begin{verbatim}
<sequence standard-attributes>
   standard-elements
   activity+
</sequence>
\end{verbatim} Egyszerűen csak tranziciók és helyek összefűzése. 

\section{\texttt{<if>}}
\texttt{<if>} Standard kétirányú elágazás. Logikai XPATH kifejezést vár. 
\begin{verbatim}
<if standard-attributes>
   standard-elements
   <condition expressionLanguage="anyURI"?>bool-expr</condition>
   activity
   <elseif>*
      <condition expressionLanguage="anyURI"?>bool-expr</condition>
      activity
   </elseif>
   <else>?
      activity
   </else>
</if>
\end{verbatim}
Egy tranzició, mely tokenek függvényében más felé küldi tovább, vagy generál tokeneket. Analóg módon egy Swithc Case elágazás is definiálható vele.

\section{\texttt{<while>}}
\texttt{<while>} While loop. Végre hajt amíg az iterációs feltétel igaznak értékelődik ki. 
\begin{verbatim}
<while standard-attributes>
   standard-elements
   <condition expressionLanguage="anyURI"?>bool-expr</condition>
   activity
</while>
\end{verbatim}
Egy tranzició, mely token függvényében a folyamat egy korábbi pontjára csatol vissza, vagy ép egy későbbire, a feltétel hamis logikai állapota esetén. A feltétel persze egy színes token jelenléte, vagy tokenek száma is lehet. 

\section{\texttt{<repeatUntil>}}
\texttt{<repeatUntil>} Egy do-while ciklusnak feleltethető annyi kivétellel, hogy akkor enged tovább, ha a feltétel igaz. 
\begin{verbatim}
<repeatUntil standard-attributes>
   standard-elements
   activity
   <condition expressionLanguage="anyURI"?>bool-expr</condition>
</repeatUntil>
\end{verbatim}
Az előzővel analóg módon megadható

\section{\texttt{<forEach>}}
\texttt{<forEach>} A kezdeti változó és a végváltozó különbsége +1 szer iteráltatja a gyerek elemet. Megadható párhuzamos feldolgozás is. Egy Complete condition segítségével megadható egy break utasítás ami kilép a forEachből. 
\begin{verbatim}
<forEach counterName="BPELVariableName" parallel="yes|no"
   standard-attributes>
   standard-elements
   <startCounterValue expressionLanguage="anyURI"?>
      unsigned-integer-expression
   </startCounterValue>
   <finalCounterValue expressionLanguage="anyURI"?>
      unsigned-integer-expression
   </finalCounterValue>
   <completionCondition>?
      <branches expressionLanguage="anyURI"?
         successfulBranchesOnly="yes|no"?>?
         unsigned-integer-expression
      </branches>
   </completionCondition>
   <scope ...>...</scope>
</forEach>
\end{verbatim} Egyszerű loop utasítás, azonban párhuzamosítás esetén a részhálóból megfelelő példányszámot generáltatunk. 

\section{\texttt{<pick>}}
\texttt{<pick>}Üzenetek várására vagy timeout eseményre figyel. Ezek bármelyike a szubprocessz végrehajtásához vezet. 
\begin{verbatim}
<pick createInstance="yes|no"? standard-attributes>
   standard-elements
   <onMessage partnerLink="NCName"
      portType="QName"?
      operation="NCName"
      variable="BPELVariableName"?
      messageExchange="NCName"?>+
      <correlations>?
         <correlation set="NCName" initiate="yes|join|no"? />+
      </correlations>
      <fromParts>?
         <fromPart part="NCName" toVariable="BPELVariableName" />+
      </fromParts>
      activity
   </onMessage>
   <onAlarm>*
      (
      <for expressionLanguage="anyURI"?>duration-expr</for>
      |
      <until expressionLanguage="anyURI"?>deadline-expr</until>
      )
      activity
   </onAlarm>
</pick>
\end{verbatim}

\section{\texttt{<flow>}}
\texttt{<flow>} konkurens elemek deklarálására szolgál. Linkek segítségével megadható függőségi viszony a gyerekek között. 
\begin{verbatim}
<flow standard-attributes>
   standard-elements
   <links>?
      <link name="NCName" />+
   </links>
   activity+
</flow>
\end{verbatim} 

\section{\texttt{<scope>}}
\texttt{<scope>} A gyerek elemek scope-ját lehet vele szabályozni.
\begin{verbatim}
<scope isolated="yes|no"? exitOnStandardFault="yes|no"?
   standard-attributes>
   standard-elements
   <partnerLinks>?
      ... see above under <process> for syntax ...
   </partnerLinks>
   <messageExchanges>?
      ... see above under <process> for syntax ...
   </messageExchanges>
   <variables>?
      ... see above under <process> for syntax ...
   </variables>
   <correlationSets>?
      ... see above under <process> for syntax ...
   </correlationSets>
   <faultHandlers>?
      ... see above under <process> for syntax ...
   </faultHandlers>
   <compensationHandler>?
      ...
   </compensationHandler>
   <terminationHandler>?
      ...
   </terminationHandler>
   <eventHandlers>?
      ... see above under <process> for syntax ...
   </eventHandlers>
   activity
</scope>
\end{verbatim}
Nem generál új elemet, csak a láthatósági, azaz visszacsatolási elemeket adja meg. 

\section{\texttt{<compensateScope>}}
\texttt{<compensateScope>}
\begin{verbatim}
<compensateScope target="NCName" standard-attributes>
   standard-elements
</compensateScope>
\end{verbatim}

\section{\texttt{<compensate>}}
\texttt{<compensate>}
\begin{verbatim}
<compensate standard-attributes>
   standard-elements
</compensate>
\end{verbatim}

\section{\texttt{<rethrow>}}
\texttt{<rethrow>}
\begin{verbatim}
<rethrow standard-attributes>
   standard-elements
</rethrow>
\end{verbatim}

\section{\texttt{<extensionActivity>}}
\texttt{<extensionActivity>}
\begin{verbatim}
<extensionActivity>
   <anyElementQName standard-attributes>
      standard-elements
   </anyElementQName>
</extensionActivity>
\end{verbatim}

%ch4 BPEL to Petri net and annotations
\chapter{Szimulációs és validációs keretrendszer}
\section{Elvárások az alkalmazással szemben}

%TODO Itt kellene röviden áttekinteni az alkalmazással szemben támasztott követelményeket.
Az alkalmazás legfőbb feladata egy konverzió, BPEL és Petri-háló között. Ebből adódóan minden BPEL elemet le kell tudnia kezelni, illetve az azok közti összefüggéseket feltérképezni, és az összefüggés halmazból egy petri hálót előállítani. A hálót tudni kell megjeleníteni, valamint a hálón belüli mozgásokat rajzolni, és szükségesség esetén input file-t vagy felhasználói inputot kezelni. Ha a háló nem hozható létre, akkor azt tudatnia kell,és lehetőség szerint rövid indoklással alátámasztania. 

\section{Az alkalmazás felépítése}

%TODO Osztály és blokkdiagramok formájában be kellene mutatni, hogy milyen fő elemekből épül fel az alkalmazás.
Az alkalmazás logikailag a következő fő részekből áll:
\begin{itemize}
\item I/O module: Beolvassa az XML dokumentumot és felparseolja. Szükség esetén menti a kész hálót.
\item Conversion module: Átkonvertálja  a beolvasott dokumentumot.
\item Data Structure module: A saját típusú Petri elemeket kezeli, és adatszerkezeti implementációt tartalmaz. 
\item UI talker: A UI ra illeszti a megfelelő input mezőt, az abba felvitt értéket átadja a feldolgozó egységnek.
\item Graphics module: Az MSGL libraryre épül. Feladata a gráf rajzolása és megjelenítése animációval együtt. 
\item Computing module: A háló animációjához végzi a szükséges számításokat és időzítéseket. 
\end{itemize}

%ch5 The definiton and logic of the framework
\chapter{Az alkalmazás implementációja}
\section{C\# implementáció}

%TODO Meg kellene mutatni, hogy milyen API és újrahasznosítható elemek készültek el.

\section{Tesztelés, tapasztalatok}

%TODO Itt kifejezetten az alkalmazás szemszögéből (nem pedig üzleti folyamatokra vonatkozóan) kellene bemutatni az alkalmazást.
%ch6 The implememntation of the framework
\chapter{A hálón végezhető elemzések}
\section{Háló korlátosság és puffer kapacitási ellenőrzés}

A háló egy adott helye akkor tekinthető korlátos (\textit{bounded}) helynek, ha bármely jelölésnél a tokenek száma az adott helynél nem megy egy adott korlát fölé. A Petri háló korlátos, ha minden helye korlátos hely.
A háló korlátossága az egyik leggyakoribb és legfontosabb minőségi jellemzője a Petri hálóknak. 

A háló alap tulajdonságainak, beleértve a korlátosságának az elemzésére több módszer is létezik, melyek közül kiemelhető a
\begin{itemize}
\item komponens/elérhetőségi gráf elemzése (SCC),
\item dekompozíciós módszerek.
\end{itemize}

A mátrix reprezentáció esetén transzformációs mátrixok segítségével írják fel a Petri háló dinamikáját. Az alapstruktúra az ú.n. incidencia M mátrixban kerül megadásra, melynek elemei az alábbi jelentéssel bírnak: $A_{ij}=a^+_{ij}-a^-_{ij}$, ahol 
\begin{itemize}
\item $a^+_{ij}: $ az élerősség az i. tranzícióból a j. kimeneti hely felé
\item $a^-_{ij}$ az élerősség az i. tranzícióhoz a j. bemeneti hely felől.
\end{itemize}

A mátrix alapvetően a tokenek számának a változását mutatja az egyes tranzició átmenetek esetére. A  Petri háló működési alapegyenlete a következő alakban adható meg: 
$$M_k=M_{k-1}+ Au_k,$$
ahol $M_k$ jelöli a háló markereinek (tokenek) státuszát a $k.$ lépésben. Az $u$ vektor a helyek tüzelési státuszt írja le. 

A fenti modellen alapuló elérhetőség vizsgálatok felhasználhatóak a korlátosság elemzésére \cite{murata1989petri}.
A kapcsolódó  egyenletek hatékony, lineáris programozási megoldása szintén ismert \cite{lasserre1989using}.

%TODO itt van egy kép, nem tudom lehet e direktben cite-olni????

\section{Saját modell alaphálóra}

Modellünkben korlátosságnak egy folyam-gráf megközelítését dolgoztuk ki.  A hálóban a következő típusú helyeket definiáljuk:
\begin{itemize}
\item forrás hely,
\item nyelő hely,
\item köztes hely.
\end{itemize}
Feltesszük, hogy csak a köztes helyeken lehet tokeneket tárolni, csak ott vannak pufferek. A hálóban az élekhez egy $I_x$ token áramlás erősséget definiálunk, ahol x jelöli az él indexét. A forrás helyekhez egy $Q_x$ forrás erősség indexet adunk meg. A hálóban az alábbi kapacitás korlátokat vezetjük be:
\begin{itemize}
\item $C_x:$ az x. tranzíció maximális erőssége,
\item $C_y:$ az y. nyelő maximális folyam erőssége.
\end{itemize}

A tranzícióknál a bemenő élek vonatkozásában kétféle működési módot értelmezünk:
\begin{itemize}
\item AND-mód: akkor van tüzelés, ha minden bejövő élnél megvan az al elvárt tokenszám, van szinkron,
\item OR-mód: akkor van tüzelés, ha megjelenik valamely bemeneten egy token, nincs szinkron.
\end{itemize}
A hálóban az alábbi megkötések élnek a folyamerősségekre:
\begin{itemize}
\item forrás helyek esetén: $\sum_y I_y=Q_x$ ($y$: kimenő élek),
\item nyelő helyek esetén: $\sum_y I_y \leq C_x$ ($y$: bejövő élek),
\item belső helyek esetén: 
$\sum_{y(ki)} I_y\leq \sum_{y(be)} I_y$,
\item tranziciók esetén: 
\begin{itemize}
\item $\sum_y I_y\leq C_x$ ($y:$ bejövő élek),
\item $\sum_{y(ki)} I_y = \sum_{y(be)} I_y$,
\item $\forall$ kimenő $x,y$ élre $I_x=I_y$.
\end{itemize}
\end{itemize}
Az AND típusú tranzakciók esetén még ezen felül teljesül, hogy $\forall$ bejövő $x,y$ élre:\\
$I_x = I_y$.

Az egyes belső helyeken a pufferbe áramló tokenek  eredő intentitása:
$$F= \sum_{x(\text{belső hely})} \left( \sum_{y(\text{x bejövő él})} I_y - \sum_{y(\text{x kimenő él})} I_y \right)$$
Az $F$ függvény 0 értéke esetén nincs szükség belső pufferre.

A fenti feladatot egy LP programozási feladatnak is tekinthető, ahol a változók az élek $I_x$ nem negatív intenzitásai és a célfüggvény:
$$F\Rightarrow \min$$ alakú.

\section{Az alkalmazott, kibővített modell}

A színezett Petri-hálók esetén több különböző típusú tokenek élnek a rendszerben. A kapacitás vizsgálatnál ekkor az egyes tranzícióknál eltérő lehet a kapacitás korlát (a maximális folyam erősség) a különböző típusú tokenek esetén. Emiatt külön kell vizsgálni az egyes típusok folyam erősségét, nem lehet összevonni őket.

A színezett hálóban az élekhez $I^c_x$ token áramlás erősségeket definiálunk, ahol $x$ jelöli az él indexét és $c$ a színkód. A forrás helyekhez $Q^c_x$ forrás erősség indexeket adunk meg a különböző $c$ színekre vonatkozólag. A hálóban az alábbi kapacitás korlátokat vezetjük be:
\begin{itemize}
\item $C^C_x$: az $x$. tranzíció maximális erőssége a c szín esetén 
\item $C^C_y$: az $y$. nyelő maximális folyam erőssége a c szín esetén
\end{itemize}
A hálóban az alábbi megkötések élnek a folyamerősségekre:
\begin{itemize}
\item forrás helyek ($x$) esetén: $\forall c$ színre: $\sum_{y\text{ kimenő élek}}I^C_Y = Q^C_x$
\item nyelő helyek ($x$) esetén: $\forall c$ színre: $\sum_{y\text{ bejövő élek}}I^C_y \leq C^C_x$
\item belső helyek esetén: $\forall c$ színre: $\sum_{y\text{ kimenő élek}} I^C_y \leq \sum_{y\text{ bejövő élek}}$
\item tranzíciók ($x$) esetén:
$$\forall c \text{ színre} \sum_{y\text{ kimenő élek}} I^C_y == \sum_{y\text{ bejövő élek}} I^C_y $$
$$\sum_{c \text{ színek}}\left( \frac{1}{C^C_x} \left( \sum_{y \text{ bejövő élek}} I^C_y \right) \right) \leq 1$$
$$\forall c \text{ színre:} \forall \text{ kimenő} (y,z) \text{élre: } I^C_y=I^C_z$$
\item az AND típusú tranzakciók esetén még ezen felül teljesül, hogy 	$\forall c$ színre: $\forall$ bejövő $(y,z)$ élre $I^C_y=I^C_z$
\end{itemize}
Az egyes belső helyeken a bufferbe áramló tokenek  eredő intenzitása:
$$F=\sum_{c\text{ színek}} \left( \sum_{x\text{ belső hely}}\left( \sum_{y\text{ bejövő élek }x\text{-nél}}I^C_y - \sum_{y \text{ kimenő helyek }x\text{-nél}} I^C_y \right) \right).$$
Az $F$ függvény 0 értéke esetén nincs szükség belső bufferre. 
A fenti feladat egy lineáris programozási feladatnak (röviden LP) is tekinthető, ahol a változók az élek $I_x$ nem negatív intenzitásai és a célfüggvény $F \rightarrow \min$ alakú. 

\section{A validációs számítás algoritmusa}

A hálót leíró struktúra három alappilléren nyugszik: helyek, tranzíciók, élek.

A helyek esetén az alábbi attribútumokat tárolja a rendszer:
\begin{itemize}
\item \texttt{id}: az egyedi azonosító kód,
\item \texttt{inputs}: bejövő élek,
\item \texttt{outputs}: kimenő élek,
\item \texttt{tokens}: tárolt tokenek,
\item \texttt{Q}: forrás intenzitás,
\item \texttt{border}: pozíció jelző, belső vagy határ pozíció.
\end{itemize}        
A tranzíciók jellemzői:
\begin{itemize}
\item \texttt{id}: egyedi azonosító kód,
\item \texttt{inputs}: bejövő élek,
\item \texttt{outputs}: kimenő élek,
\item \texttt{C}: feldolgozási intenzitás,
\item \texttt{mode}: működési mód (AND, OR).
\end{itemize}
Az élek attribútumai:
\begin{itemize}
\item \texttt{id}: azonosító kód,
\item \texttt{input}: induló elem,
\item \texttt{output}: cél elem,
\item \texttt{alfa}: az él kapacitás jelzője,
\item \texttt{inner}: él típusa, belső vagy határ.
\end{itemize}

A kapacitás vizsgálatot végző rutin az alábbi tevékenységeket hajtja végre.
Új LP feladat létrehozása a \texttt{prop} változóba:
\begin{python}
prop = Init_LpProblem(Minimize)
\end{python}
Az optimalizálási probléma változóinak inicializálása:
\begin{python}
tr_vars = LpVariable("Iv", tr_items, lowBound=0, cat='Continuous')
\end{python}
Együtthatók meghatározása és beállítása:
\begin{python}
for pp in places:
	if pp.border == 0:
		costs[ll] = costs[ll] +/- 1
\end{python}
A célfüggvény meghatározása:
\begin{python}
Init_lpSum([costs[i] * tr_vars[i] for i in tr_items])
\end{python}
Belső pontok súlyának meghatározása:
\begin{python}
for pp in places:
    if pl.border == 0:
        for e in pl.inputs:
            wgts[e] = wgts[e] + 1
        for e in pl.outputs:
            wgts[e] = wgts[e] - 1
        cnts = 0
\end{python}
Az egyenlőtlenség rendszer együtthatóinak meghatározása:
\begin{python}
Init_lpSum([wgts[i] * tr_vars[i] for i in tr_items]) >= cnts
if pl.border == 1:
    for e in pl.outputs:
        wgts[e] = wgts[e] + 1
    cnts = pl.Q

Init_lpSum([wgts[i] * tr_vars[i] for i in tr_items]) == cnts
if pl.border == 2:
    for e in pl.inputs:
        wgts[e] = wgts[e] + 1
    cnts = -pl.Q
    
Init_.lpSum([wgts[i] * tr_vars[i] for i in tr_items]) <= cnts
for tr in self.transitions:
    for e in tr.inputs:
        wgts[e] = wgts[e] + 1
    for e in tr.outputs:
        wgts[e] = wgts[e] - 1
                
Init_lpSum([wgts[i]*tr_vars[i] for i in tr_items]) == 0
    for e in tr.inputs:
        wgts[e] = wgts[e] + 1
    cnts = tr.C
    
Init_lpSum([wgts[i] * tr_vars[i] for i in tr_items]) <= cnts

if tr.mode == 'AND':
    for e in range(1, len(tr.inputs)):
        Init_lpSum([wgts[i] * tr_vars[i] for i in tr_items]) == 0
    for e in range(1,len(tr.outputs)):
        Init_lpSum([wgts[i]*tr_vars[i] for i in tr_items]) == 0

if tr.mode == 'OR':
    for e in range(1,len(tr.outputs)):
        e1 = tr.outputs[0]
        e2 = tr.outputs[e]
        wgts[e1] =  1
        wgts[e2] =  -1
        Init_p.lpSum([wgts[i]*tr_vars[i] for i in tr_items]) == 0
\end{python}
Az optimalizálási probléma megoldása, majd az eredményeinek a kiírása:
\begin{python}
prob.solve()
prob.print()
\end{python}

\section{Mintafeladat}
Vegyünk egy 4 helyből és 2 tranzícióból álló rendszert. A helyekből egy nyelő, egy forrás és kettő belső hely. A rendszerben 6 él van az ábrán megadott módon.  A gráfban a sárga elem a helyeket, zöld a tranzíciókat jelöli. A csomópont elemben az első jel a hely kódja, a második a kapcsolódó kapacitás érték. A modellben a kisebb indexű tranzíció AND tulajdonságú, a másik OR tulajdonságú. 

A rendszerben 6 (nem negatív) változó jelenik meg:  $\{0: Iv_0, 1: Iv_1, 2: Iv_2, 3: Iv_3, 4: Iv_4, 5: Iv_5\}$

Az élek indexelése:
\begin{align*}
0  :  1 \rightarrow 5  \\
1  :  5 \rightarrow 2  \\
2  :  2 \rightarrow 6  \\
3  :  6 \rightarrow 3  \\
4  :  3 \rightarrow 5  \\
5  :  6 \rightarrow 4 
\end{align*}

A rendszerhez az alábbi egyenlőtlenségek kapcsolódnak:

\begin{center}
\begin{tabular}{rll}
$C_1$ &: $Iv_0$ &$= 20$ \\
$C_2$ &: $Iv_1 - Iv_2$ &$\geq 0$\\
$C_3$ &: $Iv_3 - Iv_4$ &$\geq 0$\\
$C_4$ &: $Iv_5$ &$\leq 25$\\
$C_5$ &: $Iv_0 - Iv_1 + Iv_4$ &$= 0$\\
$C_6$ &: $Iv_0 + Iv_4 $&$\leq 50$\\
$C_7$ &: $Iv_0 - Iv_4 $&$= 0$\\
$C_8$ &: $Iv_2 - Iv_3 - Iv_5$&$= 0$\\
$C_9$ &: $Iv_2 $&$\leq 50$\\
$C_{10}$ &: $Iv_3 - Iv_5 $&$= 0$
\end{tabular}
\end{center}

A kapcsolódó célfüggvény:
$$1\cdot Iv_1 + -1\cdot Iv_2 + 1\cdot Iv_3 + -1\cdot Iv_4\Rightarrow \min$$

Az LP feladat megoldható és a kapott megoldás:
\begin{center}
\begin{tabular}{rll}
&$Iv_0$ &: $20.0$\\
&$Iv_1$ &: $40.0$\\
&$Iv_2$ &: $40.0$\\
&$Iv_3$ &: $20.0$\\
&$Iv_4$ &: $20.0$\\
&$Iv_5$ &: $20.0$\\
&$Cost$ & $= 0.0$
\end{tabular}
\end{center}
Tehát a mintarendszerben nincs szükség belső pufferre. 
Ha lecsökkentjük a második tranzíció folyamerősségét, az akkor nem kapunk érvényes megoldást. \\
Infeasible
\begin{center}
\begin{tabular}{rll}
&$Iv_0$ &: $10.0$\\
&$Iv_1$ &: $20.0$\\
&$Iv_2$ &: $20.0$\\
&$Iv_3$ &: $10.0$\\
&$Iv_4$ &: $10.0$\\
&$Iv_5$ &: $10.0$
\end{tabular}
\end{center}
%ch7 Analytics and Validation 
\section{Első példa üzleti folyamat}

%TODO A címet majd nyilván át kell írni. Itt szerepelne egy bonyolultabb üzleti folyamat, és a konverzió eredménye.

\section{Második példa üzleti folyamat}

%TODO Hasonló az előző szakaszhoz, csak másik példával.
%ch7 Given examples for analytocs
\chapter{Összegzés}
Összegzés

A BPEL szabvány az üzleti folyamatok szabványos leírására szolgál.  A dolgozat témaköre a BPEL nyelven létrehozott üzleti folyamatok modellezése  és elemzése a Petri-háló alapú formalizmus segítségével. A kidolgozott mintarendszer inputként egy BPEL modell leírását várja és kimenetként az elemzés eredményét illetve a folyamatok nyomkövetését adja vissza.

A TDK munka keretében az alábbi eredményeket értem el:
\begin{itemize}
\item BPEL folyamatok Petri háló formalizmusra történő konverziója,
\item LP alapú végesség vizsgálat a Petri hálón,
\item folyamatok grakus nyomon követése, szimuláció.
\end{itemize}

A projekt következő lépéseként a modell logisztikai folyamatokra történő  adaptálását végezzük el. A továbbfejlesztett szimulációs motorba több új elosztás alapú generációs modult is el fogok készíteni. Az eredményeket egy cikk publikációjában kívánjuk megjeleníteni.
%ch8 summatization and end wording
%TODO A felhasznált hivatkozásokat már az elején célszerű összegyűjteni!
\bibliographystyle{acm}
\bibliography{tdk}

\chapter{Mellékletek}

\section{WSDL szabvány a BPEL számára}
\begin{verbatim}
<wsdl:definitions
   targetNamespace="http://manufacturing.org/wsdl/purchase"
   xmlns:sns="http://manufacturing.org/xsd/purchase"
   xmlns:pos="http://manufacturing.org/wsdl/purchase"
   xmlns:wsdl="http://schemas.xmlsoap.org/wsdl/"
   xmlns:plnk="http://docs.oasis-open.org/wsbpel/2.0/plnktype"
   xmlns:xsd="http://www.w3.org/2001/XMLSchema">
   
   <wsdl:types>
      <xsd:schema>
         <xsd:import namespace="http://manufacturing.org/xsd/purchase"
          schemaLocation="http://manufacturing.org/xsd/purchase.xsd" />
      </xsd:schema>
   </wsdl:types> 

   <wsdl:message name="POMessage">
      <wsdl:part name="customerInfo" type="sns:customerInfoType" />
      <wsdl:part name="purchaseOrder" type="sns:purchaseOrderType" />
   </wsdl:message>

   <wsdl:message name="InvMessage">
      <wsdl:part name="IVC" type="sns:InvoiceType" />
   </wsdl:message>

   <wsdl:message name="orderFaultType">
     <wsdl:part name="problemInfo" element="sns:OrderFault"/>
   </wsdl:message>

   <wsdl:message name="shippingRequestMessage">
      <wsdl:part name="customerInfo" element="sns:customerInfo" />
   </wsdl:message>

   <wsdl:message name="shippingInfoMessage">
      <wsdl:part name="shippingInfo" element="sns:shippingInfo" />
   </wsdl:message>

   <wsdl:message name="scheduleMessage">
      <wsdl:part name="schedule" element="sns:scheduleInfo" />
   </wsdl:message> 

   <!-- portTypes supported by the purchase order process -->

   <wsdl:portType name="purchaseOrderPT">
      <wsdl:operation name="sendPurchaseOrder">
         <wsdl:input message="pos:POMessage" />
         <wsdl:output message="pos:InvMessage" />
         <wsdl:fault name="cannotCompleteOrder"
            message="pos:orderFaultType" />
      </wsdl:operation>
   </wsdl:portType>
   
   <wsdl:portType name="invoiceCallbackPT">
      <wsdl:operation name="sendInvoice">
         <wsdl:input message="pos:InvMessage" />
      </wsdl:operation>
   </wsdl:portType>

   <wsdl:portType name="shippingCallbackPT">
      <wsdl:operation name="sendSchedule">
         <wsdl:input message="pos:scheduleMessage" />
      </wsdl:operation>
   </wsdl:portType>

   <!-- portType supported by the invoice services -->

   <wsdl:portType name="computePricePT">
      <wsdl:operation name="initiatePriceCalculation">
         <wsdl:input message="pos:POMessage" />
      </wsdl:operation>      
      
      <wsdl:operation name="sendShippingPrice">
         <wsdl:input message="pos:shippingInfoMessage" />
      </wsdl:operation>
   </wsdl:portType>

   <!-- portType supported by the shipping service -->

   <wsdl:portType name="shippingPT">
      <wsdl:operation name="requestShipping">
         <wsdl:input message="pos:shippingRequestMessage" />
         <wsdl:output message="pos:shippingInfoMessage" />
         <wsdl:fault name="cannotCompleteOrder"
            message="pos:orderFaultType" />
      </wsdl:operation>
   </wsdl:portType>

   <!-- portType supported by the production scheduling process -->

   <wsdl:portType name="schedulingPT">
      <wsdl:operation name="requestProductionScheduling">
         <wsdl:input message="pos:POMessage" />
      </wsdl:operation>
      
      <wsdl:operation name="sendShippingSchedule">
         <wsdl:input message="pos:scheduleMessage" />
      </wsdl:operation>
   </wsdl:portType> 

   <plnk:partnerLinkType name="purchasingLT">
      <plnk:role name="purchaseService"
         portType="pos:purchaseOrderPT" />
   </plnk:partnerLinkType> 

   <plnk:partnerLinkType name="invoicingLT">
      <plnk:role name="invoiceService"
         portType="pos:computePricePT" />
      <plnk:role name="invoiceRequester"
         portType="pos:invoiceCallbackPT" />
   </plnk:partnerLinkType> 

   <plnk:partnerLinkType name="shippingLT">
      <plnk:role name="shippingService"
         portType="pos:shippingPT" />
      <plnk:role name="shippingRequester"
         portType="pos:shippingCallbackPT" />
   </plnk:partnerLinkType> 

   <plnk:partnerLinkType name="schedulingLT">
      <plnk:role name="schedulingService"
         portType="pos:schedulingPT" />
   </plnk:partnerLinkType>
</wsdl:definitions>
\end{verbatim}

A szükséges portok és linkek definiálása után most a rendelés processzét definiáljuk hasonlóképp. 
\begin{verbatim}
<process name="purchaseOrderProcess"
   targetNamespace="http://example.com/ws-bp/purchase"
   xmlns="http://docs.oasis-open.org/wsbpel/2.0/process/executable"
   xmlns:lns="http://manufacturing.org/wsdl/purchase"> 
   
   <documentation xml:lang="EN">
      A simple example of a WS-BPEL process for handling a purchase
      order.
   </documentation> 

   <partnerLinks>
      <partnerLink name="purchasing"
         partnerLinkType="lns:purchasingLT" myRole="purchaseService" />
      <partnerLink name="invoicing" partnerLinkType="lns:invoicingLT"
         myRole="invoiceRequester" partnerRole="invoiceService" />
      <partnerLink name="shipping" partnerLinkType="lns:shippingLT"
         myRole="shippingRequester" partnerRole="shippingService" />
      <partnerLink name="scheduling"
         partnerLinkType="lns:schedulingLT"
         partnerRole="schedulingService" />
   </partnerLinks> 

   <variables>
      <variable name="PO" messageType="lns:POMessage" />
      <variable name="Invoice" messageType="lns:InvMessage" />
      <variable name="shippingRequest"
         messageType="lns:shippingRequestMessage" />
      <variable name="shippingInfo"
         messageType="lns:shippingInfoMessage" />
      <variable name="shippingSchedule"
         messageType="lns:scheduleMessage" />
   </variables> 

   <faultHandlers>
      <catch faultName="lns:cannotCompleteOrder"
         faultVariable="POFault"
         faultMessageType="lns:orderFaultType">
         <reply partnerLink="purchasing"
            portType="lns:purchaseOrderPT"
            operation="sendPurchaseOrder" variable="POFault"
            faultName="cannotCompleteOrder" />
      </catch>
   </faultHandlers> 

   <sequence>
      <receive partnerLink="purchasing" portType="lns:purchaseOrderPT"
         operation="sendPurchaseOrder" variable="PO"
         createInstance="yes">
         <documentation>Receive Purchase Order</documentation>
      </receive>

      <flow>
         <documentation>
            A parallel flow to handle shipping, invoicing and
            scheduling
         </documentation>
         <links>
            <link name="ship-to-invoice" />
            <link name="ship-to-scheduling" />
         </links>
         <sequence>
            <assign>
               <copy>
                  <from>$PO.customerInfo</from>
                  <to>$shippingRequest.customerInfo</to>
               </copy>
            </assign>
            <invoke partnerLink="shipping" portType="lns:shippingPT"
               operation="requestShipping"
               inputVariable="shippingRequest"
               outputVariable="shippingInfo">
               <documentation>Decide On Shipper</documentation>
               <sources>
                  <source linkName="ship-to-invoice" />
               </sources>
            </invoke>
            <receive partnerLink="shipping"
               portType="lns:shippingCallbackPT"
               operation="sendSchedule" variable="shippingSchedule">
               <documentation>Arrange Logistics</documentation>
               <sources>
                  <source linkName="ship-to-scheduling" />
               </sources>
            </receive>
         </sequence>
         <sequence>
            <invoke partnerLink="invoicing"
               portType="lns:computePricePT"
               operation="initiatePriceCalculation"
               inputVariable="PO">
               <documentation>
                  Initial Price Calculation
               </documentation>
            </invoke>
            <invoke partnerLink="invoicing"
               portType="lns:computePricePT"
               operation="sendShippingPrice"
               inputVariable="shippingInfo">
               <documentation>
                  Complete Price Calculation
               </documentation>
               <targets>
                  <target linkName="ship-to-invoice" />
               </targets>
            </invoke>
            <receive partnerLink="invoicing"
               portType="lns:invoiceCallbackPT"
               operation="sendInvoice" variable="Invoice" />
         </sequence>
         <sequence>
            <invoke partnerLink="scheduling"
               portType="lns:schedulingPT"
               operation="requestProductionScheduling"
               inputVariable="PO">
               <documentation>
                  Initiate Production Scheduling
               </documentation>
            </invoke>
            <invoke partnerLink="scheduling"
               portType="lns:schedulingPT"
               operation="sendShippingSchedule"
               inputVariable="shippingSchedule">
               <documentation>
                  Complete Production Scheduling
               </documentation>
               <targets>
                  <target linkName="ship-to-scheduling" />
               </targets>
            </invoke>
         </sequence>
      </flow>
      
      <reply partnerLink="purchasing" portType="lns:purchaseOrderPT"
         operation="sendPurchaseOrder" variable="Invoice">
         <documentation>Invoice Processing</documentation>
      </reply>
   </sequence>
</process>
\end{verbatim}
\end{document}
