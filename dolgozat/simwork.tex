\chapter{Szimulációs keretrendszer}

%TODO Be kell mutatni a C# nyelvű alkalmazást.

\section{Elvárások az alkalmazással szemben}

%TODO Itt kellene röviden áttekinteni az alkalmazással szemben támasztott követelményeket.
Az alkalmazás legfőbb feladata egy konverzió, BPEL és Petri-háló között. Ebből adódóan minden BPEL elemet le kell tudnia kezelni, illetve az azok közti összefüggéseket feltérképezni, és az összefüggés halmazból egy petri hálót előállítani. A hálót tudni kell megjeleníteni, valamint a hálón belüli mozgásokat rajzolni, és szükségesség esetén input file-t vagy felhasználói inputot kezelni. Ha a háló nem hozható létre, akkor azt tudatnia kell,és lehetőség szerint rövid indoklással alátámasztania. 

\section{Az alkalmazás felépítése}

%TODO Osztály és blokkdiagramok formájában be kellene mutatni, hogy milyen fő elemekből épül fel az alkalmazás.
Az alkalmazás logikailag a következő fő részekből áll:
\begin{itemize}
\item I/O module: Beolvassa az XML dokumentumot és felparseolja. Szükség esetén menti a kész hálót.
\item Conversion module: Átkonvertálja  a beolvasott dokumentumot.
\item Data Structure module: A saját típusú Petri elemeket kezeli, és adatszerkezeti implementációt tartalmaz. 
\item UI talker: A UI ra illeszti a megfelelő input mezőt, az abba felvitt értéket átadja a feldolgozó egységnek.
\item Graphics module: Az MSGL libraryre épül. Feladata a gráf rajzolása és megjelenítése animációval együtt. 
\item Computing module: A háló animációjához végzi a szükséges számításokat és időzítéseket. 
\end{itemize}

\section{C\# implementáció}

%TODO Meg kellene mutatni, hogy milyen API és újrahasznosítható elemek készültek el.

\section{Tesztelés, tapasztalatok}

%TODO Itt kifejezetten az alkalmazás szemszögéből (nem pedig üzleti folyamatokra vonatkozóan) kellene bemutatni az alkalmazást.
